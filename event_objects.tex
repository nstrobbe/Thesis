%%%%%%%%%%%%%%%%%%%%%%%%%%%%%%%%%%%
%%  Object identification
%%%%%%%%%%%%%%%%%%%%%%%%%%%%%%%%%%%



% The average pileup energy due to neutral hadrons is computed
% event-by-event and subtracted from the energy when computing lepton isolation and jet energy.  The
% energy subtracted is  the average pileup energy per unit area (in $\Delta\eta \times \Delta\phi$)
% times the jet area~\cite{Fastjet1, Fastjet2}.
% this corrects energy and momentum, not substructure
% TODO: move to jet and lepton sections

% 
% Missing transverse energy, which is used in the calculation of the razor variable $\mr$, is 
% defined to be the negative sum of the transverse momenta of all the particle flow objects in an
% event.  Loosely identified and isolated electrons with $\pt > 5$~\GeV and $|\eta| < 2.5$ and muons
% with $\pt > 5$\GeV and $|\eta| < 2.4$ are used both to suppress backgrounds in our signal region
%and
% in the definition of the control regions.  A tight definition of isolated leptons (electrons with
% $\pt > 10$~\GeV and $|\eta| < 2.5$ and muons with $\pt > 10$~\GeV and $|\eta| < 2.4$) defines a
% control region enriched in $\cPZ \rightarrow \ell \ell $ events, from which we estimate the
% systematic uncertainty in the predicted number of $\cPZ \rightarrow \nu \nu$ events in the signal
% region. Any electron candidates with $1.44 < |\eta| < 1.57$ are rejected since the transition
%region
% between barrel and endcap calorimeters is less well-instrumented.
% In order to suppress the decays of taus and other leptons that fail the loose selection, events
%that
% have isolated tracks with $\pt > 10$\GeV and track-primary vertex distance along the beam
%direction
% $dz < 0.05$ are rejected.

\subsubsection{Primary vertices \label{sec:object_vertex}}

We require at least one {\it good} primary vertex to be reconstructed in each event. 
This vertex should be associated with at least four charged-particle tracks. It should also lie
within 24\cm of the origin of the CMS coordinate system along the beam direction, and within 2\cm
in the plane transverse to the beam. 
These requirements, translated to the CMS nomenclature, are summarized in
Table~\ref{tab:object_vertex}.
In case there are multiple good vertices, we choose the vertex with the highest value of $\sum
\pt^2$ of associated tracks to be the leading primary vertex in the event. This vertex is
taken as a reference to reconstruct the event, e.g. to perform the track subtraction for pileup
removal, for which we use the charged hadron subtraction algorithm, as explained before.

\begin{table}[htdp]
\caption{Vertex selection criteria. \label{tab:object_vertex}}
\begin{center}
\begin{tabular}{l l}
\toprule
\texttt{\small isFake()} & $= 0$ \\
\texttt{\small ndof()} & $> 4$ \\
\texttt{\small z()} & $< 24\cm$ \\
\texttt{\small position.Rho()} & $< 2\cm$ \\
\bottomrule
\end{tabular}
\end{center}
\end{table}


\subsubsection{Jets \label{sec:object_jets}}

Most analyses are interested primarily in the quarks and gluon produced in the hard interaction, or
in the decay of heavy particles, such as top quarks or $\W$ bosons. 
However, through the process of parton showering and hadronization, the few initial quarks and
gluons turn into a multitude of hadrons. 
Hadrons from a given initial quark or gluon can usually be found close together, they form a
\textit{jet}. The proper description of jets, and the jet definitions that are used to reconstruct
them, relies on two properties: infrared, and collinear safety~\cite{Salam:2009jx}.
It is important that a jet definition returns the same set of final jets regardless of whether a
parton underwent a collinear or soft splitting. If this is not the case, i.e. the jet definition is
infrared or collinear unsafe, then one finds that divergencies in the theoretical computation of
jet cross sections do not vanish. 

A jet definition comprises two parts: the jet algorithm that defines in which order particles are
grouped together, and the recombination scheme that defines how to combine the momenta of the
to-be-merged particles. 
For the latter, the most common choice is to simply add the four-vectors of the particles, which
then gives rise to massive jets. 
For the jet algorithm there are many choices. Here I will focus solely on the anti-$k_\textrm{T}$
algorithm~\cite{antikt}, which is the default jet algorithm used by CMS.
As for most sequential recombination algorithms, one defines distances $d_{ij}$ between particles
$i$ and $j$ (or pseudojets if particles have been combined before),  and distances $d_{iB}$ between
particle $i$ and the beam.
The distance measures are in this case given by
\begin{align}
  d_{ij} &= \min \left(\frac{1}{p_{\mathrm{T,i}}^2}, \frac{1}{p_{\mathrm{T,j}}^2}\right)
\frac{\Delta R_{ij}^2}{R^2}, \\
  d_{iB} &= \frac{1}{p_{\mathrm{T,i}}^2},
\end{align}
where $\Delta R_{ij}^2 = (y_i - y_j)^2 + (\phi_i - \phi_j)^2$ and $R$ is a tuneable parameter
determining the size of the jets. The rapidity $y$ of a particle is given by,
\begin{equation}
  y = \frac{1}{2} \ln{\frac{ E + p_z }{ E - p_z }} . \label{eq:rapidity}
\end{equation}
The jet clustering proceeds by identifying the smallest of all distances. If it is a $d_{ij}$, we
recombine particles $i$ and $j$, while if it is $d_{iB}$, we move $i$ from the list of particles to
the list of final jets. All distances are then recalculated and the procedure is repeated until no
particles are left.
The anti-$k_\textrm{T}$ algorithm results in mostly circular jets, reminiscent of the older cone jet
algorithms that are no longer used because they are not infrared and collinear safe.

The input to the jet clustering are the PF candidates that pass the charged hadron subtraction. 
The clustering itself is done with the anti-$k_\textrm{T}$ algorithm with size parameter $R=0.5$
(AK5), as implemented in \textsc{FastJet 3.0.1}~\cite{Cacciari:2011ma}.
We apply the standard loose identification criteria to the resulting jets, as defined by the
requirements listed in Table~\ref{tab:object_jets}. Jets are required to not be composed of only a
single component, as this usually indicates that the reconstruction is of poor quality. 

Unfortunately, the calorimeter response to incident particles is not uniform. It is, therefore, not
straightforward to translate the measured jet energy to the true particle energy, which is what we
want to use to do our analysis. A set of jet energy scale corrections -- scalings of the
jet four-momentum depending on jet \pt and $\eta$ -- are applied to both data and
simulation in order to achieve a proper mapping to the particle level. 
Jet energy corrections within CMS are taken care of in a sequential way, each level of correction
taking care of a different effect~\cite{JEC,Chatrchyan:2011ds}. 
The uncertainties associated with the jet energy scale corrections need to be taken into account as
systematic uncertainty on any analysis result. How this will be done in the razor boost analysis is
explained in Section~\ref{sec:boost_JEC}.  

First, the residual effect from pileup is removed using the \textit{L1 corrections}. 
The effect of pileup on a given jet is quantified by the so-called offset, defined as the
difference in \pt for a reconstructed jet with added pileup and the same jet without pileup.
The effects of charged hadrons from in-time pileup have already been largely reduced by the charged
hadron subtraction method. The effect of neutral particles and out-of-time pileup is removed at this
stage using a slightly modified version of the \textit{jet area method}~\cite{Fastjet1,Fastjet2}.
This method uses the effective area of the jets, $A$, multiplied by the average energy density
in the event, $\rho$, to calculate the energy to be subtracted from the jets.
Both real and simulated jets are first corrected with a $\pt$, $\eta$, and number of primary
vertices dependent offset correction determined in simulation. For data events, an additional
data/simulation scale factor is derived from ZeroBias data to correct for remaining $\eta$ dependent
discrepancies.
Figure~\ref{fig:JEC_L1} shows the size of the energy offset for AK5 jets in the central region,
before and after the L1 corrections have been applied. A clear reduction of the overall offset is
observed. The dependence on the number of pileup interaction has also been reduced, indicating that
the L1 corrections behave properly.  

\begin{figure}[tpb]
  \centering
  \includegraphics[width=0.4\textwidth]{figures/eventreco_objects/OffMeantnpuRef_BB_ak5pfchs}
  ~
  \includegraphics[width=0.4\textwidth]{figures/eventreco_objects/OffMeantnpuRef_BB_ak5pfchsl1}
  \caption{The offset shown on the $y$-axis in these plots is defined as the difference in
transverse momentum for a reconstructed jet with added pileup and the same jet without pileup.
The left-hand side shows the offset as a function of the generated \pt of a jet before the L1
corrections have been applied, and the right-hand side shows the offset after pileup
corrections. Different markers represent different levels of pileup. Figures taken from
Ref.~\cite{JEC_plots}.
  \label{fig:JEC_L1}}
\end{figure}

Since the simulation of the detector response is very detailed, see
Section~\ref{sec:event_simulation}, the jet response in the absence of pileup is in fact very well
modelled in simulation. The bulk of the jet energy corrections will thus be derived using
simulation, and only the residual differences between data and simulation are derived directly
from the data.

The second and third level of corrections, the \textit{L2 Relative} and \textit{L3 Absolute
corrections}, are designed to make the jet response flat in $\eta$ and \pt, respectively. They are
derived from the simulation together. The L2 correction corrects a jet at arbitrary $\eta$ relative
to a jet in the central area ($|\eta|<1.3$). Once that is done, the jet energy is translated back to
the particle level, such that on average the \pt of a reconstructed jet matches that of a jet
clustered using generator level particles,
\begin{equation}
  <\pt(\mathrm{reco})_{\mathrm{corr}}> {=} <\pt(\mathrm{gen})> .
\end{equation}
These are the final corrections applied to jets from simulated events. The size of the L2 and L3
corrections as a function of jet $\eta$, for three reference \pt values, is shown in
Fig.~\ref{fig:JEC_L23} on the left-hand side. 

\begin{figure}[tpb]
  \centering
\includegraphics[height=0.22\textheight]
{figures/eventreco_objects/CorrectionVsEta_Overview_TDR_ak5pfl1_L2L3}
~
\includegraphics[height=0.22\textheight]
  {figures/eventreco_objects/ResComp_FSRcorr_residuals_Abseta_PF_DiJetData}
  \caption{[left] The size of the L2 and L3 corrections as a function of jet $\eta$ for three
reference transverse momentum values: 30\GeV (white hollow circles), 100\GeV (red squares) and
300\GeV (blue circles). 
[right] L2 Residual corrections, obtained from dijet events, as a function of $\eta$. The jet
energy scale uncertainty is shown with a yellow band, while the statistical uncertainty is shown
with a blue band.
Figures taken from Ref.~\cite{JEC_plots2}.
  \label{fig:JEC_L23}}
\end{figure}

Data events are further corrected by the \textit{L2L3 Residual} jet energy scale
\textit{corrections} to take care of the small differences between data and simulation. These
corrections are \pt and $\eta$ dependent, and only correct the relative energy scale. The absolute
energy scale was found to be well modelled in the simulation. A dedicated, data-driven approach is
employed, using data samples of dijet, $\gamma+$jet, and $\cPZ+$jet events. The L2 Residual
correction derived from dijet events is shown on the right in Fig.~\ref{fig:JEC_L23}.

After all corrections have been applied, jets to be used for analysis are required to have $\pt >
30\GeV$ and $|\eta| < 2.4$.
The AK5 jets defined in this section will be used for most aspects of
the razor boost analysis, except for the reconstruction of boosted hadronic $\W$-candidates. 
Section~\ref{sec:boost_wtag} provides details on the dedicated jet treatment that is used for $\W$
tagging.

\begin{table}[htdp]
\caption{Jet selection criteria. \label{tab:object_jets}}
\begin{center}
\begin{tabular}{l l}
\toprule
\pt & $> 30\GeV$ \\
$|\eta|$ & $< 2.4$ \\
\midrule
\texttt{\small neutralHadronEnergyFraction()} & $< 0.99$ \\
\texttt{\small neutralEmEnergyFraction()} & $< 0.99$ \\
\texttt{\small nConstituents()} & $> 1$ \\
\texttt{\small chargedHadronEnergyFraction()} & $> 0$ \\
\texttt{\small chargedMultiplicity()} & $> 0$ \\
\texttt{\small chargedEmEnergyFraction()} & $< 0.99$ \\
\bottomrule
\end{tabular}
\end{center}
\end{table}

%%%%%%%%%%%%%%%%%%%%%%%%%%%%%%%%%%%%%%%%%%%%%%%%%%%%%%%%%%%%%%%%%%%%%%%%%%%%%%%%%%%%%%%%%

\subsubsection{B-tagging \label{sec:object_btag}}

Jets originating from the hadronization of $\cPqb$ quarks can be distinguished from other jets,
initiated by gluons or light flavour quarks, due to the long lifetime, of the order of
1.5\ten{-12}\second, of the $B$ hadrons. 
The non-prompt decay of the $B$ hadrons results in a secondary vertex, displaced by hundreds of
micrometers with respect to the primary vertex of the hard interaction. This feature can be used to
identify jets as originating from a $\cPqb$ quark. 

The ability to distinguish $\cPqb$ jets is especially important for new physics searches. Many new
physics models are associated with production of third generation quarks, whereas this is more rare
in the standard model. For many searches $\cPqb$ jet tagging is an essential tool in suppressing
the background from multijet or vector boson production. 

CMS has developed several $\cPqb$ tagging algorithms\cite{btag7TeV,btag8TeV}. The combined
secondary vertex (CSV) algorithm is the most widely used, and combines information on the secondary
vertices and the impact parameter of the tracks into a likelihood based discriminant. 
 
The input to any $\cPqb$ tag algorithm are high purity tracks, which feature a good track fit with
many separate hits, have $\pt > 1\GeV$, and lie within the jet cone. 
The impact parameter of a track is defined as the distance from the primary vertex to the track at
the point of closest approach.  
Secondary vertices within jets are reconstructed using an adaptive vertex
fitter~\cite{Fruhwirth:2007hz}. The list of secondary vertices is cleaned from candidates that
share more than 65\% of their tracks with the primary vertex of the event. Vertices that are
consistent with originating from a $K^0$ decay are also removed. Properties of the secondary
vertices used in the CSV likelihood discriminant are the flight distance between secondary and
primary vertex, the flight distance significance, the vertex invariant mass, and the total \pt of
all associated tracks. The impact parameter of the tracks is also included. The shape of the CSV
discriminant, and a data versus simulation comparison, is shown on Fig.~\ref{fig:CSV_discriminant}.

In the razor boost analysis the CSV $\cPqb$ tagging algorithm will be applied on the PF jets at two
working points~\cite{BTagWP}, which are shown on Table~\ref{tab:object_btag}. 
The Loose working point (CSVL), corresponding to a misidentification rate of $\sim$10\% and
efficiency of $\sim$85\%, will be used to veto $\cPqb$ jets, whereas the Medium working point
(CSVM), corresponding to a misidentification rate of $\sim$1\% and a typical efficiency of
$\sim$70\% , is used to select $\cPqb$ jets.
The small differences in the discriminant shape, as observed from Fig.~\ref{fig:CSV_discriminant}, 
will result in a slightly different $\cPqb$ tagging efficiency in data versus simulation. Scale
factors with associated uncertainties have been derived to correct for this effect, and are
provided by the BTAG POG in CMS. Section~\ref{sec:boost_btag_sf} will cover how these
uncertainties are propagated to the final systematic uncertainties for the razor boost analysis. 

\begin{figure}[tpb]
  \centering
  \includegraphics[width=0.6\textwidth]{figures/eventreco_objects/CSV_Log_ttbar}
  \caption{ Discriminator value for the CSV discriminant for a $t\bar{t}$-enriched data sample.
  \label{fig:CSV_discriminant}}
\end{figure}

\begin{table}[htdp]
\caption{Working points for the combined secondary vertex $\cPqb$ jet tagger.
\label{tab:object_btag}}
\begin{center}
\begin{tabular}{l l}
\toprule
Working point & Discriminator value \\
\midrule
Medium & $> 0.679$ \\
Loose & $> 0.244$ \\
\bottomrule
\end{tabular}
\end{center}
\end{table}


%%%%%%%%%%%%%%%%%%%%%%%%%%%%%%%%%%%%%%%%%%%%%%%%%%%%%%%%%%%%%%%%%%%%%%%%%%%%%%%%%%%%%%%%%

\subsubsection{Muons \label{sec:object_muon}}

The razor boost analysis uses muons that are identified using two different working
points, a loose selection and a tight selection, both of which will be detailed below. 
The loose selection is used throughout most of the analysis, both for vetoing the presence of muons
in the signal region, and for selecting single muon events in the control regions. The tight
selection is used only to define a control region enriched in $\cPZ\rightarrow
\ell\bar{\ell}$ events which is used to derive a systematic uncertainty on the contribution of
$\cPZ\rightarrow \nu\bar{\nu}$ in the signal region. 

The \textbf{loose muon selection} was developed especially for events with a
large amount of hadronic activity, where the standard identification criteria were observed to lose
efficiency, resulting in less background suppression when vetoing the presence of muons. 
The details and performance of this optimized selection is documented in
Ref.~\cite{CMS-AN2011-498}. 
The main feature is the use of a so-called \textit{directional isolation}.
The isolation of a particle is a measure of how far it is from other activity in the detector. The
leptons we are interested in, those originating in the hard interaction, are usually separated from
other activity, e.g. jets. This is not the case for misidentified muons or for muons from the decay
of heavy-flavour jets. Directional isolation is designed to have a better rejection of leptons from
these heavy-flavour jet decays, and is defined as
\begin{equation}
\overrightarrow{\mathrm{ISO}}(R) \equiv \sum_{\Delta R_{i} < R} \delta_{i}^{2}\pt{}_{i} ,
\end{equation}
where the sum is over all other particles $i$ within $\Delta R_{i}<R$ of the muon direction,
and $\delta_{i}$ is the angle between particle $i$ and the $\pt$-weighted centroid position
($\delta_{c}$) of all such particles in $(\eta,\phi)$ space. That is, if $\Delta\phi_i$ and
$\Delta\eta_i$ are respectively the difference in $\phi$ and $\eta$ angles between particle $i$ and
the muon, then:
\begin{eqnarray*}
\vec{e}_{i} & \equiv & \frac{1}{\sqrt{\Delta\phi_{i}^{2}+\Delta\eta_{i}^{2}}}\left(\begin{array}{c}
\Delta\phi_{i},\\
\Delta\eta_{i}
\end{array}\right),\\
\vec{\delta}_{c} & = & \sum_{\Delta R_{i}<R}\pt{}_{i}\vec{e}_{i},\\
\delta_{i} & = &
\angle(\vec{\delta}_{c},\vec{e}_{i})=\arccos(\vec{\delta}_{c}\cdot\vec{e}_{i}/|\vec{\delta}_{c}|),
\end{eqnarray*}
where $\vec{e}_{i}$ is the unit vector specifying particle $i$'s relative location in $(\eta,\phi)$
space with respect to the considered muon, as illustrated in Fig.~\ref{fig:object_directional_iso}.
Because of the weighting by $\delta_{i}^{2}$, the value for the directional isolation tends to be
larger for muons that are near the jet core, e.g. in case of leptonic $\cPqb$ decays, compared to
the more conventional isolation definition which does not use this weighting. 

\begin{figure}[htpb]
  \centering
  \includegraphics[width=0.8\textwidth]{figures/eventreco_objects/directional_iso_cartoon}
  \caption{Illustration of ingredients used in the computation of directional isolation for a prompt
muon, denoted by a star, near some particles from a jet, denoted by points, in the $(\eta,\phi)$
plane. For prompt leptons $\delta_i$ tends to be small, especially for the high-\pt particles near
the core of the jet. Figure taken from Ref.~\cite{CMS-AN2011-498}.
  \label{fig:object_directional_iso}}
\end{figure}

Apart from the isolation, the identification criteria themselves are also altered from the standard
Loose Muon ID from the POG in order to further optimize the muon identification in environments
with large hadronic activity. 
Loose muons are reconstructed using either the global muon algorithm or the tracker-only
algorithm. 
Global muons are required to pass the {\tt GlobalMuonPromptTight} quality criteria,
and to have at least two muon chambers containing segments uniquely matched to its inner track. 
Tracker-only muons are required to pass the {\tt TMLastStationTight} criteria, which require the
muon to have compatible hits in the last muon chamber. 
All selected muons are then required to pass the selection listed in
Table~\ref{tab:object_loosemuon}. 
Some aspects of the selection depend on the muon $\pt$ and $\eta$; these are summarized in
Table~\ref{tab:object_loosemuon_cuts}.

\begin{table}[p]
\caption{Loose muon definition. }
\begin{center}
{\small
\begin{tabular}{l l}
\toprule
\pt & $> 5\GeV$ \\
$|\eta|$ & $< 2.4$ \\
\midrule
\texttt{\footnotesize innerTrack().hitPattern().numberOfLostHits()} & $\leq 1$ if $\pt < 20\GeV$ \\
                                                      & $\leq 4$ if $\pt \geq 20\GeV$ \\
$|\texttt{\footnotesize innerTrack().dxy(vertex.position())}|$ & $\pt$- and $\eta$-dependent\\
$|\texttt{\footnotesize muonBestTrack().dz(vertex.position())}|$ & $\pt$- and $\eta$-dependent\\
\midrule
$\overrightarrow{\mathrm{ISO}}(R=0.2)$ & $\pt$- and $\eta$-dependent \\
\bottomrule
\end{tabular}
}
\end{center}
\label{tab:object_loosemuon}
\end{table}

\begin{table}[p]
\caption{Details of the $\pt$ dependent thresholds employed in the loose muon selection.}
\begin{center}
  \begin{tabular}{l cccccc }
      \toprule
      Muon $\pt$  & $d_{xy} (\cm)$ & $d_{xy} (\cm)$ & $d_z (\cm)$ & $d_z (\cm)$ &
$\overrightarrow{\mathrm{ISO}}(0.2)$ &
$\overrightarrow{\mathrm{ISO}}(0.2)$ \\
      (\GeV) & Barrel & Endcap & Barrel & Endcap & Barrel & Endcap \\
      \midrule
      0 - 5          & 0.052 & 0.037 & 0.054 & 0.076 & 1.5  & 2    \\
      5 - 10         & 0.041 & 0.018 & 0.042 & 0.082 & 3    & 2.5  \\
      10 - 25        & 0.029 & 0.013 & 0.028 & 0.098 & 7    & 7.5  \\
      15 - 20        & 0.014 & 0.015 & 0.034 & 0.1   & 10.5 & 9    \\
      20 - 40        & 0.021 & 0.021 & 1     & 0.1   & 15.5 & 13.5 \\
      40 - 80        & 0.04  & 0.2   & 1     & 1     & 32.5 & 19   \\
      80 - 140       & 0.1   & 0.2   & 1     & 1     & 54.5 & 37   \\
      140 - 200      & 0.1   & 0.2   & 1     & 1     & 87   & 65.5 \\
      \bottomrule
    \end{tabular}
\end{center}
\label{tab:object_loosemuon_cuts}
\end{table}

 
The \textbf{tight muon selection} follows the recommendation from the Muon POG~\cite{MuonID}.
In addition to the identification criteria, we also require the tight muon to be isolated. 
Here we do not use directional isolation, but rather the more standard particle-based relative
isolation. 
This isolation, denoted $I_\mu$, is calculated using the PF candidates in a cone of size $\Delta R =
0.4$ around the muon. Charged-hadron candidates associated with pileup vertices are not taken into
account in the calculation of the isolation. However, they are used to estimate the remaining
contribution to the isolation coming from neutral hadrons associated with pileup. This contribution
is then subtracted. 
The isolation definition is given by:
\begin{equation}
I_\mu = \frac{I_{Charged} + \max\left(0, I_{Neutral} + I_{\gamma} - \Delta\beta\cdot
I_{Charged}^{PU}\right)}
             {\pt^\mu} , 
\label{eqn:iso}
\end{equation}
where $I_{Charged}$, $I_{Neutral}$, and $I_{\gamma}$ are computed as the sum of the \pt of the
charged hadrons, neutral hadrons and photons, respectively, in a cone of size $\Delta R = 0.4$
around the muon. The parameter $\Delta\beta$ is set to 0.5, and $I_{Charged}^{PU}$ is the estimated
contribution from pileup computed as the sum of the \pt of the charged hadrons associated with
pileup vertices.
The tight muon isolation requirement is $I_\mu < 0.15$.
A summary of the tight muon selection can be found in Table~\ref{tab:object_tightmuon}. 

\begin{table}[p]
\caption{Tight muon definition. }
\begin{center}
{\small
\begin{tabular}{l l}
\toprule
\pt & $> 10\GeV$ \\
$|\eta|$ & $< 2.4$ \\
\midrule
\texttt{\footnotesize isPFMuon()} & $= 1$ \\
\texttt{\footnotesize isGlobalMuon()} & $= 1$ \\
\texttt{\footnotesize globalTrack().normalizedChi2()} & $< 10$ \\
\texttt{\footnotesize globalTrack().hitPattern().numberOfValidMuonHits()} & $> 0$ \\
\texttt{\footnotesize track().hitPattern().trackerLayersWithMeasurement()} & $> 5$ \\
\texttt{\footnotesize innerTrack().hitPattern().numberOfValidPixelHits()} & $> 0$ \\
\texttt{\footnotesize numberOfMatchedStations()} & $> 1$ \\
$|\texttt{\footnotesize innerTrack().dxy(vertex.position())}|$ & $< 0.2\cm$ \\
$|\texttt{\footnotesize muonBestTrack().dz(vertex.position())}|$ & $< 0.5\cm$ \\
\midrule
$I_\mu =$ [\texttt{\footnotesize pfIsolationR04().sumChargedHadronPt()}& \\
\hspace{0.9cm} $+$ max(0., \texttt{\footnotesize pfIsolationR04().sumNeutralHadronPt()}  & \\
\hspace{2.7cm} $+$ \texttt{\footnotesize pfIsolationR04().sumPhotonPt()}  & \\
\hspace{2.7cm} $-$ 0.5 $\cdot$ \texttt{\footnotesize pfIsolationR04().sumPUPt()}) & \\
\hspace{0.9cm} ] / \pt & $< 0.15$ \\ 
\bottomrule
\end{tabular}
}
\end{center}
\label{tab:object_tightmuon}
\end{table}

 
%%%%%%%%%%%%%%%%%%%%%%%%%%%%%%%%%%%%%%%%%%%%%%%%%%%%%%%%%%%%%%%%%%%%%%%%%%%%%%%%%%%%%%%%%


\subsubsection{Electrons \label{sec:object_electron}}

Similar to the muon selection, we identify electrons using two different working points, a loose
selection, and a tight selection. 

The \textbf{loose electron selection} uses directional isolation as described in the previous
section, and fully documented in Ref.~\cite{CMS-AN2011-498}. A summary of the complete
loose electron selection is given in Table~\ref{tab:object_looseelectron}, with the details of
the $\pt$- and $\eta$-dependent requirements listed in Table~\ref{tab:object_looseelectron_cuts}. 

\begin{table}[p]
  \caption{Loose electron definition.}
  \begin{center}
  {\small 
    \begin{tabular}{l l l l}
      \toprule
      & Condition & Barrel & Endcap \\
      \midrule
      \pt & & $ > 5 \GeV$ & $> 5\GeV$ \\
      $|\eta|$ & & $ < 1.442$ & $1.556 - 2.5$ \\
      \midrule
      \texttt{\footnotesize gsfTrack().numberOfLostHits()} & $\pt < 20\GeV$ & $= 0$ & $= 0$ \\
      \texttt{\footnotesize gsfTrack().hitPattern().numberOfValidPixelHits()} & $\pt < 10\GeV$ &
$\geq 2$ & $\geq 1$ \\
      $|\texttt{\footnotesize gsfTrack().dz(vertex.position())}|$ & & \multicolumn{2}{l}{$\pt$- and
$\eta$-dependent}\\
      \midrule
      $\overrightarrow{\mathrm{ISO}}(R=0.3)$, calculated from charged particles only & &
\multicolumn{2}{l}{$\pt$- and $\eta$-dependent} \\
      $\overrightarrow{\mathrm{ISO}}(R=0.2)$, barrel only, calculated using all particles & &
\multicolumn{2}{l}{$\pt$- and $\eta$-dependent} \\
      \bottomrule
    \end{tabular}
    }
  \end{center}
  \label{tab:object_looseelectron} 
\end{table}


\begin{table}[p]
  \caption{Details of the $\pt$ dependent thresholds employed in the loose electron selection.}
  \begin{center}
  \begin{tabular}{ l ccccc }
      \toprule
      Electron $\pt$ & $d_z (\cm)$ & $d_z (\cm)$ &
$\overrightarrow{\mathrm{ISO}}(0.3,\textrm{charged})$ &
$\overrightarrow{\mathrm{ISO}}(0.3,\textrm{charged})$ & $\overrightarrow{\mathrm{ISO}}(0.2)$ \\
      (\GeV) & Barrel & Endcap & Barrel & Endcap & Barrel \\
      \midrule
      0 - 5          & 0.03 & 0.09 & 0.5  & 0.5  & 2    \\
      5 - 10         & 0.05 & 0.09 & 1.5  & 2.5  & 4.25 \\
      10 - 25        & 0.05 & 0.09 & 4.5  & 6.5  & 8.75 \\
      15 - 20        & 0.05 & 0.11 & 7.5  & 9    & 11   \\
      20 - 40        & 0.2  & 1    & 10   & 10.5 & 20.8 \\
      40 - 80        & 1    & 1    & 18.5 & 18.5 & 200  \\
      80 - 140       & 1    & 1    & 44   & 66.5 & 200  \\
      140 - 200      & 1    & 1    & 81.5 & 70   & 200  \\
      \bottomrule
    \end{tabular}
  \end{center}
  \label{tab:object_looseelectron_cuts}
\end{table}

The \textbf{tight electron selection} is in accordance with the recommendations of the EGamma POG
\cite{ElectronID}. A summary of the selection can be found in table~\ref{tab:object_tightelectron}.
We also require to electron to be isolated. The isolation $I_e$ is calculated using the PF
candidates in a cone of size $\Delta R = 0.3$ around the electron, and then corrected with an
estimate of the median energy from pileup as calculated with the {\tt FastJet} algorithm in a
similar way to the L1 jet corrections explained in Sec.~\ref{sec:object_jets}. 
We require that this corrected isolation, relative to the $\pt$ of the electron is less than 0.15.
\begin{equation}
I_e = \frac{ I_{Charged} + \max(0, I_{NeutralHad} + I_{\gamma} - A \rho ) }{\pt^e}, 
\end{equation}
with $A$ the effective area of the cone, and $\rho$ the average pileup density. 
Small discrepancies exist between the electron identification efficiency in data and in simulation.
Scale factors are provided to correct for this effect, see Section~\ref{sec:boost_leptonID}.

\begin{table}[htpb]
\caption{Tight electron definition. }
\begin{center}
{\small
\begin{tabular}{l l l}
\toprule
& Barrel & Endcap \\
\midrule
\pt & $> 10\GeV$ & $> 10\GeV$\\
$|\eta|$ & $< 1.442$ & $1.556 - 2.5$ \\
\midrule
$|$\texttt{\footnotesize deltaEtaSuperClusterTrackAtVtx()}$|$ & $< 0.004$ & $< 0.005$ \\
$|$\texttt{\footnotesize deltaPhiSuperClusterTrackAtVtx()}$|$ & $< 0.030$ & $< 0.020$ \\
\texttt{\footnotesize sigmaIetaIeta()} & $< 0.010$ & $< 0.030$ \\
\texttt{\footnotesize hadronicOverEm()} & $< 0.120$ & $< 0.100$ \\
1.0/\texttt{\footnotesize ecalEnergy()} - \texttt{\footnotesize eSuperClusterOverP()/ecalEnergy()} &
$< 0.050$ &
$< 0.050$ \\
\texttt{\footnotesize gsfTrack().trackerExpectedHitsInner().numberOfHits()} & $\le 0$ & $\le 0$ \\
\texttt{\footnotesize passConversionVeto()} & $= 1$ & $= 1$ \\
$|\texttt{\footnotesize innerTrack().dxy(vertex.position())}|$ & $< 0.02\cm$ & $< 0.02\cm$\\
$|\texttt{\footnotesize gsfTrack().dz(vertex.position())}|$ & $< 0.1\cm$ & $< 0.1\cm$ \\
\midrule
$I_e$ & $<0.15$ & $< 0.15$ \\
\bottomrule
\end{tabular}
}
\end{center}
\label{tab:object_tightelectron}
\end{table}

%%%%%%%%%%%%%%%%%%%%%%%%%%%%%%%%%%%%%%%%%%%%%%%%%%%%%%%%%%%%%%%%%%%%%%%%%%%%%%%%%%%%%%%%%

\subsubsection{Isolated tracks \label{sec:object_isolatedtrack}}

In order to suppress the decays of both taus and other leptons that do not pass the loose
selection, we can veto events for which an isolated track is present~\cite{CMS-AN2013-089}. 
Isolated tracks are selected from the charged PF candidates with $\pt > 10\GeV$ and
longitudinal track-primary vertex distance of $d_z < 0.05\cm$. They are required to have a
relative isolation in a cone of $\Delta R = 0.3$ of less than 0.1. 
In the razor boost analysis the isolated track veto will only be applied in the hadronic event
selections, and not in the control regions which require the presence of a lepton. 

\begin{table}[htdp]
\caption{Isolated track selection. }
\begin{center}
\begin{tabular}{l l}
\toprule
\pt & $> 10\GeV$ \\
\midrule
\texttt{charge()} & $> 0$ \\
$d_z({\rm PV, track})$ & $< 0.05\cm$ \\
$I_{\textrm{track}_i} = \frac{\sum_{j \neq i} \pt{}_j }{ \pt{}_i }$ & $< 0.1$ \\
\bottomrule
\end{tabular}
\end{center}
\label{tab:isolatedtrack}
\end{table}

\subsubsection{Missing transverse momentum \label{sec:object_met}}

The missing transverse momentum, \VEtmiss, associated with a given event is computed as the negative
vector sum of the transverse momentum of all PF candidates $i$,
\begin{equation}
  \VEtmiss = - \sum_i \ptvec^{\,i} .
\end{equation}
and its magnitude is denoted by \ETm. 

The corrections to the jet energy scale discussed above are propagated to the \VEtmiss as well. 
Within CMS this type of missing transverse momentum is known as type-1 corrected
\VEtmiss~\cite{Khachatryan:2014gga}.
\begin{equation}
  {\vec E}_{\mathrm{T}\text{,type-1}}^{\text{miss}} = 
  {\vec E}_{\mathrm{T}\text{,raw}}^{\text{miss}} + \sum_i {\vec p}^{\,i}_{\mathrm{T}\text{,raw}}
 - \sum_i {\vec p}^{\,i}_{\mathrm{T}\text{,corr}} - \sum_i \vec{\mathcal{O}}^{\,i}
\end{equation}
where $\VEtmiss{}_{\mathrm{raw}}$ is the uncorrected missing transverse energy,
$\ptvec{}_{\mathrm{raw}}$ is the uncorrected jet \pt, $\ptvec{}_{\mathrm{corr}}$ is the fully
corrected jet \pt, and $\vec{\mathcal{O}}$ is the average offset due to pileup. 
Only jets with ${\vec p}^{\,i}_{\mathrm{T}\text{,corr}} > 10 \GeV$ are included in the sum.
The average pileup offset underneath jets is included in the \ETm vector sum to ensure that the
pileup offset remains isotropic and does not cause any bias.

The missing transverse momentum is sensitive to detector malfunctions and to various
reconstruction effects that result in the mismeasurement of particles or their misidentification.
Precise calibration of all reconstructed physics objects is thus crucial for the performance of
\VEtmiss. 

No explicit selection will be placed on \ETm in the razor boost analysis selection, but it is
used in the definition of the razor variable $\rsq$, to be introduced in
Section~\ref{sec:boost_razor}.

