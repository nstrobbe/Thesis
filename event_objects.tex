%%%%%%%%%%%%%%%%%%%%%%%%%%%%%%%%%%%
%%  Object identification
%%%%%%%%%%%%%%%%%%%%%%%%%%%%%%%%%%%

\subsubsection{Vertex Selection \label{sec:object_vertex}}

We require at least one {\it good} vertex to be reconstructed in each event, according to the
definitions given in Table~\ref{tab:object_vertex}.
The vertex with the highest value of $\sum \pt$ of associated tracks is chosen as the primary vertex
in the event. This vertex is taken as a reference to reconstruct the event, e.g. to perform the
track subtraction for pileup removal, for which we use the {\tt PFNoPileUp} algorithm.

\begin{table}[htdp]
\caption{Vertex selection criteria. \label{tab:object_vertex}}
\begin{center}
\begin{tabular}{l l}
\toprule
\texttt{\small isFake()} & $= 0$ \\
\texttt{\small ndof()} & $> 4$ \\
\texttt{\small z()} & $< 24\cm$ \\
\texttt{\small position.Rho()} & $< 2\cm$ \\
\bottomrule
\end{tabular}
\end{center}
\end{table}


\subsubsection{Jet selection \label{sec:object_jets}}

Jets are clustered from the set of PF candidates with \textsc{FastJet 3.0.1}~\cite{Cacciari:2011ma}
using the anti-$k_\textrm{T}$ algorithm~\cite{antikt} with size parameter $\Delta R=0.5$ (AK5).  
Jet corrections are applied as a function of jet $\pt$ and $\eta$ to account for the residual
effects of non-uniform detector response.  
The jet energies are corrected so that, on average, they match those of simulated particle-level
jets. 

% TODO Add information on jet corrections and pileup subtraction


% The jets are corrected for pile-up effects in a two step process.  
% First charged hadron particle-flow candidates that have been associated with a pile-up vertex are
% removed from the list of particles to be clustered using the {\tt PFNoPileUp} algorithm.  
% The jets are then clustered and corrected for the L2 and L3 corrections, taking into
% account the charged-hadron removal. 
% The remaining PU energy is subtracted by applying the event-by-event quantity $\pi \rho (\Delta
% R)^2$, where $\Delta R$ is the jet size and $\rho$ is the average density from PU events, as
% computed by {\tt FastJet} using only neutral hadron particle-flow candidates.  

After all corrections are applied, jets are required to have $\pt > 30$~\GeV and $|\eta| < 2.4$.  
We also apply the standard loose identification criterion, defined by the requirements listed in
Table~\ref{tab:object_jets}. 

\begin{table}[htdp]
\caption{Jet selection \label{tab:object_jets}}
\begin{center}
\begin{tabular}{l l}
\toprule
\pt & $> 30\GeV$ \\
$|\eta|$ & $< 2.4$ \\
\midrule
\texttt{\small neutralHadronEnergyFraction()} & $< 0.99$ \\
\texttt{\small neutralEmEnergyFraction()} & $< 0.99$ \\
\texttt{\small nConstituents()} & $> 1$ \\
\texttt{\small chargedHadronEnergyFraction()} & $> 0$ \\
\texttt{\small chargedMultiplicity()} & $> 0$ \\
\texttt{\small chargedEmEnergyFraction()} & $< 0.99$ \\
\bottomrule
\end{tabular}
\end{center}
\end{table}

The AK5 jets defined here will be used for most of the razor boost analysis, except to reconstruct
boosted hadronic $\W$-candidates. For this more details can be found in
Section~\ref{sec:boost_wtag}.

\subsubsection{B-Tagging \label{sec:object_btag}}

Jets originating from the hadronization of $\cPqb$ quarks can be distinguished from other jets,
initiated by gluons or light flavor quarks, due to the long lifetime of the $\cPqb$ quark. 
The non-prompt decay of the $B$ hadrons results in a secondary vertex, displaced with respect to
the primary vertex of the hard interaction. 

% TODO add more info on b-tagging algorithm

The ability to distinguish $\cPqb$ jets is especially important for new physics searches. Many new
physics models are associated with production of third generation quarks, whereas this is more rare
in the standard model. For many searches $\cPqb$ jet tagging is an essential tool in suppressing
the background from multijet or vector boson production. 

In the razor boost analysis $\cPqb$ tagging will also be employed. We will use the combined
secondary vertex (CSV) algorithm at two working points~\cite{btag7TeV,btag8TeV,BTagWP}, which are
shown on Table~\ref{tab:object_btag}. 
The Loose working point (CSVL), corresponding to a misidentification rate of $\sim$10\% and
efficiency of $\sim$85\%, will be used to veto $\cPqb$ jets, whereas the Medium working point
(CSVM), corresponding to a misidentification rate of $\sim$1\% and a typical efficiency of
$\sim$70\% , is used to select $\cPqb$ jets. 

\begin{table}[htdp]
\caption{Working points for the combined secondary vertex $\cPqb$ jet tagger.
\label{tab:object_btag}}
\begin{center}
\begin{tabular}{l l}
\toprule
Working point & Discriminator value \\
\midrule
Medium & $> 0.679$ \\
Loose & $> 0.244$ \\
\bottomrule
\end{tabular}
\end{center}
\end{table}
% 
% As will be explained in section~\ref{sec:selection}, we define our signal and control regions
% based on the number of b-tagged jets. 
% As the b-tagged jet multiplicity distribution is not exactly the same in data as in simulation, we
% need to apply appropriate Data/MC scale factors to the simulation. These scalefactors and their
% prescription have been provided by the BTag POG \cite{BTagSF1,BTagSF2}. 
% Whenever an explicit selection based on the number of b-tagged jets is made, the btag scale
% factors are applied to the simulation. 
% For more detailed information on the scale factors and their associated uncertainties, we refer to
% section~\ref{sec:btag_uncertainties}. 

% TODO Decide where to put the scale factor information

\subsubsection{Muons \label{sec:object_muon}}

Muons are identified using two different working points, a loose selection and a tight selection,
both of which will be detailed below. 


% Currently we mainly use the loose definition in the analysis, both for vetoing, and for selecting
% single muon events for the control regions enriched in TTJets and WJets. The tight selection is
% only used to define a control region enriched in $Z\rightarrow ll$ events, from which we derive a
% systematic uncertainty on the predicted number of $Z\rightarrow\nu\nu$ events in our signal
% region.

The \textbf{loose muon selection} that will be employed was developed especially for events with a
large amount of hadronic activity, where the standard identification criteria were observed to lose
efficiency, resulting in less background suppression when vetoing the presence of muons. 
The details and performance of this optimized selection is documented in
Ref.~\cite{CMS-AN2011-498}. 
The main feature is the use of a so-called \textit{directional} isolation.
The isolation of a particle is a measure of how far it is from other activity in the detector. The
leptons we are interested in, those originating in the hard interaction, are usually separated from
other activity, e.g. jets. This is not the case for misidentified muons or for muons from the decay
of heavy-flavour jets. Directional isolation is designed to have a better rejection of leptons from
these heavy-flavour jet decays, and is defined as
\begin{equation}
\overrightarrow{\mathrm{ISO}}(R) \equiv \sum_{\Delta R_{i} < R} \delta_{i}^{2}\pt{}_{i} ,
\end{equation}
where the sum is over all other particles $i$ within $\Delta R_{i}<R$ of the muon direction,
and $\delta_{i}$ is the angle between particle $i$ and the $\pt$-weighted centroid position
($\delta_{c}$) of all such particles in $(\eta,\phi)$ space. That is, if $\Delta\phi_i$ and
$\Delta\eta_i$ are respectively the difference in $\phi$ and $\eta$ angles between particle $i$ and
the muon, then:
\begin{eqnarray*}
\vec{e}_{i} & \equiv & \frac{1}{\sqrt{\Delta\phi_{i}^{2}+\Delta\eta_{i}^{2}}}\left(\begin{array}{c}
\Delta\phi_{i},\\
\Delta\eta_{i}
\end{array}\right),\\
\vec{\delta}_{c} & = & \sum_{\Delta R_{i}<R}\pt{}_{i}\vec{e}_{i},\\
\delta_{i} & = &
\angle(\vec{\delta}_{c},\vec{e}_{i})=\arccos(\vec{\delta}_{c}\cdot\vec{e}_{i}/|\vec{\delta}_{c}|),
\end{eqnarray*}
where $\vec{e}_{i}$ is the unit vector specifying particle $i$'s relative location in $(\eta,\phi)$
space with respect to the considered muon, as illustrated in Fig.~\ref{fig:object_directional_iso}.
Because of the weighting by $\delta_{i}^{2}$, the value for the directional isolation tends to be
larger for muons that are near the jet core, e.g. in case of leptonic $\cPqb$ decays, compared to
the more convential isolation definition which does not use this weighting. 

\begin{figure}[htpb]
  \centering
  \includegraphics[width=0.8\textwidth]{figures/eventreco_objects/directional_iso_cartoon}
  \caption{Illustration of ingredients used in the computation of directional isolation for a prompt
muon, denoted by a star, near some particles from a jet, denoted by points, in the $(\eta,\phi)$
plane. For prompt leptons $\delta_i$ tends to be small, especially for the high-\pt particles near
the core of the jet. Figure taken from Ref.~\cite{CMS-AN2011-498}.
  \label{fig:object_directional_iso}}
\end{figure}

Apart from the isolation, the identication criteria themselves are also altered from the standard
Loose Muon ID from the POG in order to further optimize the muon identification in environments
with large hadronic activity. 
Loose muons are reconstructed using either the global muon algorithm or the tracker-only
algorithm. 
Global muons are required to pass the {\tt GlobalMuonPromptTight} quality criteria,
and to have at least two muon chambers containing segments uniquely matched to its inner track. 
Tracker-only muons are required to pass the {\tt TMLastStationTight} criteria, which require the
muon to have compatible hits in the last muon chamber. 
All selected muons are then required to pass the selection listed in
Table~\ref{tab:object_loosemuon}. 
Some aspects of the selection depend on the muon $\pt$ and $\eta$; these are summarized in
Table~\ref{tab:object_loosemuon_cuts}.

\begin{table}[p]
\caption{Loose muon definition. }
\begin{center}
{\small
\begin{tabular}{l l}
\toprule
\pt & $> 5\GeV$ \\
$|\eta|$ & $< 2.4$ \\
\midrule
\texttt{\footnotesize innerTrack().hitPattern().numberOfLostHits()} & $\leq 1$ if $\pt < 20\GeV$ \\
                                                      & $\leq 4$ if $\pt \geq 20\GeV$ \\
$|\texttt{\footnotesize innerTrack().dxy(vertex.position())}|$ & $\pt$- and $\eta$-dependent\\
$|\texttt{\footnotesize muonBestTrack().dz(vertex.position())}|$ & $\pt$- and $\eta$-dependent\\
\midrule
$\overrightarrow{\mathrm{ISO}}(R=0.2)$ & $\pt$- and $\eta$-dependent \\
\bottomrule
\end{tabular}
}
\end{center}
\label{tab:object_loosemuon}
\end{table}

\begin{table}[p]
\caption{Details of the $\pt$ dependent thresholds employed in the loose muon selection.}
\begin{center}
  \begin{tabular}{l cccccc }
      \toprule
      Muon $\pt$  & $d_{xy} (\cm)$ & $d_{xy} (\cm)$ & $d_z (\cm)$ & $d_z (\cm)$ &
$\overrightarrow{\mathrm{ISO}}(0.2)$ &
$\overrightarrow{\mathrm{ISO}}(0.2)$ \\
      (\GeV) & Barrel & Endcap & Barrel & Endcap & Barrel & Endcap \\
      \midrule
      0 - 5          & 0.052 & 0.037 & 0.054 & 0.076 & 1.5  & 2    \\
      5 - 10         & 0.041 & 0.018 & 0.042 & 0.082 & 3    & 2.5  \\
      10 - 25        & 0.029 & 0.013 & 0.028 & 0.098 & 7    & 7.5  \\
      15 - 20        & 0.014 & 0.015 & 0.034 & 0.1   & 10.5 & 9    \\
      20 - 40        & 0.021 & 0.021 & 1     & 0.1   & 15.5 & 13.5 \\
      40 - 80        & 0.04  & 0.2   & 1     & 1     & 32.5 & 19   \\
      80 - 140       & 0.1   & 0.2   & 1     & 1     & 54.5 & 37   \\
      140 - 200      & 0.1   & 0.2   & 1     & 1     & 87   & 65.5 \\
      \bottomrule
    \end{tabular}
\end{center}
\label{tab:object_loosemuon_cuts}
\end{table}

 
The \textbf{tight muon selection} follows the recommendation from the Muon POG~\cite{MuonID}.
In addition to the identification criteria, we also require the tight muon to be isolated. 
Here we do not use directional isolation, but rather the more standard particle-based relative
isolation. 
This isolation, denoted $I_\mu$, is calculated using the PF candidates in a cone of size $\Delta R =
0.4$ around the muon. Charged-hadron candidates associated with pileup vertexes are not taken into
account in the calculation of the isolation. However, they are used to estimate the remaining
contribution to the isolation coming from neutral hadrons associated with pileup. This contribution
is then subtracted. 
The isolation definition is given by:
\begin{equation}
I_\mu = \frac{I_{Charged} + I_{Neutral} + I_{\gamma} - \Delta\beta\cdot I_{Charged}^{PU}}
             {\pt^\mu} , 
\label{eqn:iso}
\end{equation}
where $I_{Charged}$, $I_{Neutral}$, and $I_{\gamma}$ are computed as the sum of the \pt of the
charged hadrons, neutral hadrons and photons, respectively, in a cone of size $\Delta R = 0.4$
around the muon. The parameter $\Delta\beta$ is set to 0.5, and $I_{Charged}^{PU}$ is the estimated
contribution from pileup computed as the sum of the \pt of the charged hadrons associated with
pileup vertices.
The tight muon isolation requirement is $I_\mu < 0.15$.
A summary of the tight muon selection can be found in Table~\ref{tab:object_tightmuon}. 

\begin{table}[p]
\caption{Tight muon definition. }
\begin{center}
{\small
\begin{tabular}{l l}
\toprule
\pt & $> 10\GeV$ \\
$|\eta|$ & $< 2.4$ \\
\midrule
\texttt{\footnotesize isPFMuon()} & $= 1$ \\
\texttt{\footnotesize isGlobalMuon()} & $= 1$ \\
\texttt{\footnotesize globalTrack().normalizedChi2()} & $< 10$ \\
\texttt{\footnotesize globalTrack().hitPattern().numberOfValidMuonHits()} & $> 0$ \\
\texttt{\footnotesize track().hitPattern().trackerLayersWithMeasurement()} & $> 5$ \\
\texttt{\footnotesize innerTrack().hitPattern().numberOfValidPixelHits()} & $> 0$ \\
\texttt{\footnotesize numberOfMatchedStations()} & $> 1$ \\
$|\texttt{\footnotesize innerTrack().dxy(vertex.position())}|$ & $< 0.2\cm$ \\
$|\texttt{\footnotesize muonBestTrack().dz(vertex.position())}|$ & $< 0.5\cm$ \\
\midrule
$I_\mu =$ [\texttt{\footnotesize pfIsolationR04().sumChargedHadronPt()}& \\
\hspace{0.9cm} $+$ max(0., \texttt{\footnotesize pfIsolationR04().sumNeutralHadronPt()}  & \\
\hspace{2.7cm} $+$ \texttt{\footnotesize pfIsolationR04().sumPhotonPt()}  & \\
\hspace{2.7cm} $-$ 0.5 $\cdot$ \texttt{\footnotesize pfIsolationR04().sumPUPt()}) & \\
\hspace{0.9cm} ] / \pt & $< 0.15$ \\ 
\bottomrule
\end{tabular}
}
\end{center}
\label{tab:object_tightmuon}
\end{table}

 

\subsubsection{Electrons \label{sec:object_electron}}

Similar to the muon selection, we identify electrons using two different working points, a loose
selection, and a tight selection. 

% Currently we mainly use the loose definition in the analysis, both for vetoing, and for selecting
% single electron events for the control regions enriched in TTJets and WJets. The tight selection
% is only used to define a control region enriched in $Z\rightarrow ll$ events, from which we
% derive a systematic uncertainty on the predicted number of $Z\rightarrow\nu\nu$ events in our
% signal region.


The \textbf{loose electron selection} uses directional isolation as described in the previous
section, and fully documented in Ref.~\cite{CMS-AN2011-498}. A summary of the complete
loose electron selection is given in Table~\ref{tab:object_looseelectron}, with the details of
the $\pt$- and $\eta$-dependent requirements listed in Table~\ref{tab:object_looseelectron_cuts}. 

\begin{table}[p]
  \caption{Loose electron definition.}
  \begin{center}
  {\small 
    \begin{tabular}{l l l l}
      \toprule
      & Condition & Barrel & Endcap \\
      \midrule
      \pt & & $ > 5 \GeV$ & $> 5\GeV$ \\
      $|\eta|$ & & $ < 1.442$ & $1.556 - 2.5$ \\
      \midrule
      \texttt{\footnotesize gsfTrack().numberOfLostHits()} & $\pt < 20\GeV$ & $= 0$ & $= 0$ \\
      \texttt{\footnotesize gsfTrack().hitPattern().numberOfValidPixelHits()} & $\pt < 10\GeV$ &
$\geq 2$ & $\geq 1$ \\
      $|\texttt{\footnotesize gsfTrack().dz(vertex.position())}|$ & & \multicolumn{2}{l}{$\pt$- and
$\eta$-dependent}\\
      \midrule
      $\overrightarrow{\mathrm{ISO}}(R=0.3)$, calculated from charged particles only & &
\multicolumn{2}{l}{$\pt$- and $\eta$-dependent} \\
      $\overrightarrow{\mathrm{ISO}}(R=0.2)$, barrel only, calculated using all particles & &
\multicolumn{2}{l}{$\pt$- and $\eta$-dependent} \\
      \bottomrule
    \end{tabular}
    }
  \end{center}
  \label{tab:object_looseelectron} 
\end{table}


\begin{table}[p]
  \caption{Details of the $\pt$ dependent thresholds employed in the loose electron selection.}
  \begin{center}
  \begin{tabular}{ l ccccc }
      \toprule
      Electron $\pt$ & $d_z (\cm)$ & $d_z (\cm)$ &
$\overrightarrow{\mathrm{ISO}}(0.3,\textrm{charged})$ &
$\overrightarrow{\mathrm{ISO}}(0.3,\textrm{charged})$ & $\overrightarrow{\mathrm{ISO}}(0.2)$ \\
      (\GeV) & Barrel & Endcap & Barrel & Endcap & Barrel \\
      \midrule
      0 - 5          & 0.03 & 0.09 & 0.5  & 0.5  & 2    \\
      5 - 10         & 0.05 & 0.09 & 1.5  & 2.5  & 4.25 \\
      10 - 25        & 0.05 & 0.09 & 4.5  & 6.5  & 8.75 \\
      15 - 20        & 0.05 & 0.11 & 7.5  & 9    & 11   \\
      20 - 40        & 0.2  & 1    & 10   & 10.5 & 20.8 \\
      40 - 80        & 1    & 1    & 18.5 & 18.5 & 200  \\
      80 - 140       & 1    & 1    & 44   & 66.5 & 200  \\
      140 - 200      & 1    & 1    & 81.5 & 70   & 200  \\
      \bottomrule
    \end{tabular}
  \end{center}
  \label{tab:object_looseelectron_cuts}
\end{table}

The \textbf{tight electron selection} is in accordance with the recommendations of the EGamma POG
\cite{ElectronID}. A summary of the selection can be found in table~\ref{tab:object_tightelectron}.
We also require to electron to be isolated. The isolation is calculated using the PF candidates in a
cone of size $\Delta R = 0.3$ around the electron, and then corrected with an estimate of the
median energy from pileup as calculated with the {\tt FastJet} algorithm in a similar way to the
jet corrections explained in Sec.~\ref{sec:object_jets}. We require that this corrected isolation,
relative to the $\pt$ of the electron is less than 0.15.

\begin{equation}
I_e = \frac{ I_{Charged} + \max(0, I_{NeutralHad} + I_{\gamma} - A \rho ) }{\pt^e}
\end{equation}

% TODO: add more info on the pileup correction

\begin{table}[p]
\caption{Tight electron definition. }
\begin{center}
{\small
\begin{tabular}{l l l}
\toprule
& Barrel & Endcap \\
\midrule
\pt & $> 10\GeV$ & $> 10\GeV$\\
$|\eta|$ & $< 1.442$ & $1.556 - 2.5$ \\
\midrule
$|$\texttt{\footnotesize deltaEtaSuperClusterTrackAtVtx()}$|$ & $< 0.004$ & $< 0.005$ \\
$|$\texttt{\footnotesize deltaPhiSuperClusterTrackAtVtx()}$|$ & $< 0.030$ & $< 0.020$ \\
\texttt{\footnotesize sigmaIetaIeta()} & $< 0.010$ & $< 0.030$ \\
\texttt{\footnotesize hadronicOverEm()} & $< 0.120$ & $< 0.100$ \\
1.0/\texttt{\footnotesize ecalEnergy()} - \texttt{\footnotesize eSuperClusterOverP()/ecalEnergy()} &
$< 0.050$ &
$< 0.050$ \\
\texttt{\footnotesize gsfTrack().trackerExpectedHitsInner().numberOfHits()} & $\le 0$ & $\le 0$ \\
\texttt{\footnotesize passConversionVeto()} & $= 1$ & $= 1$ \\
$|\texttt{\footnotesize innerTrack().dxy(vertex.position())}|$ & $< 0.02\cm$ & $< 0.02\cm$\\
$|\texttt{\footnotesize gsfTrack().dz(vertex.position())}|$ & $< 0.1\cm$ & $< 0.1\cm$ \\
\midrule
$I_e$ & $<0.15$ & $< 0.15$ \\
\bottomrule
\end{tabular}
}
\end{center}
\label{tab:object_tightelectron}
\end{table}


\subsubsection{Isolated tracks \label{sec:object_isolatedtrack}}

In order to suppress the decays of both taus and other leptons that do not pass the loose
selection, we can veto events for which an isolated track is present~\cite{CMS-AN2013-089}. 
Isolated tracks are selected from the charged PF candidates with $\pt > 10\GeV$ and
longitudional track-primary vertex distance of $d_z < 0.05\cm$. They are required to have a
relative isolation in a cone of $\Delta R = 0.3$ of less than 0.1. 
In the razor boost analysis the isolated track veto will only be applied in the hadronic event
selections, and not in the control regions which require the presence of a lepton. 

\begin{table}[htdp]
\caption{Isolated track selection. }
\begin{center}
\begin{tabular}{l l}
\toprule
\pt & $> 10\GeV$ \\
\midrule
\texttt{charge()} & $> 0$ \\
$d_z({\rm PV, track})$ & $< 0.05\cm$ \\
$I_{\textrm{track}_i} = \frac{\sum_{j \neq i} \pt{}_j }{ \pt{}_i }$ & $< 0.1$ \\
\bottomrule
\end{tabular}
\end{center}
\label{tab:isolatedtrack}
\end{table}

\subsubsection{Missing transverse momentum \label{sec:object_met}}

The missing transverse momentum, \ETm, associated with a given event is computed as the negative
vector sum of the transverse momentum of all PF candidates $i$,
\begin{equation}
  \ETm = - \sum_i \pt^i .
\end{equation}
The corrections to the jet energy scale discussed above are propagated to the \ETm as well. 
Within CMS this type of missing transverse momentum is know as type-1 corrected \ETm.

No explicit selection will be placed on \ETm in the razor boost analysis selection, although it is
used in the definition of the razor variable $\rsq$, to be introduced in
Section~\ref{sec:boost_razor}.

