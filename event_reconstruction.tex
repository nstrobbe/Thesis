%%%%%%%%%%%%%%%%%%%%%%%%%%%
% Event reconstruction
%%%%%%%%%%%%%%%%%%%%%%%%%%%

The CMS event reconstruction aims to provide a global event description in terms of electrons,
muons, photons, charged hadrons, and neutral hadrons. Information from all subdetectors is combined
to achieve a fully consistent picture of the event. The algorithm that implements this, is called
particle flow (PF), and is documented in Refs.~\cite{CMS-PAS-PFT-09-001,PF}. 

Particle flow relies heavily on the high resolution sicilon tracker, and high granularity ECAL. The
basic building blocks are tracks, constructed using a very efficient tracking algorithm, and
clusters of calorimeter energy deposits. 
These two elements are then linked together using a linking algorithm. The actual particle flow
algorithm uses the links to create a list of reconstructed and identified particles, that are
subsequently used for physics analysis. A description of each of these steps will be given in
Section~\ref{sec:event_reco_pf}.

The particles that are reconstructed by the PF algorithm can be further refined to suit the needs
of individual analyses or analysis groups. More stringent identification criteria are usually
required, so that the misidentification rate, and thus background rate, is substantially reduced.
All physics objects that will be used in the razor boost analysis will be listed in
Section~\ref{sec:event_objects}.

Reconstructed events are sometimes affected by spurious detector noise or reconstruction failures,
leading to anomalous amounts of missing transverse momentum, or to very high \pt jets. 
These events are filtered out by various targeted cleaning algorithms, as discussed in
Section~\ref{sec:event_cleaning}.

The PF event reconstruction is applied in the same way to data and simulation, resulting in an
overall very good agreement between them. Some quantities cannot be adequately modelled in
simulation, however. Event reweighting techniques are employed to account for discrepancies that
might influence a physics analysis. In Section~\ref{sec:event_reweighting} I will discuss the
standard event reweighting techniques that are applied in the razor boost analysis. 


\subsection{Particle flow \label{sec:event_reco_pf}}


The particle flow method reconstructs particles, the PF candidates, by combining information from
the inner tracker, the calorimeters, and the muon system.  Each PF candidate is assigned to one of
five object categories: muons, electrons, photons, charged hadrons, and neutral hadrons.  

By simultaneously using information from all subdetectors, overlaps between object collections are
removed. In a calorimeter only reconstruction, for example, photons and electrons are also
reconstructed as jets. 
Since about 65\% of the jet energy is carried by charged hadrons, inclusion of the tracker
information in the jet reconstruction results in a much improved energy resolution. For jets
originating from a $\cPqb$ quark this is even more striking, as energy carried away by muons from
the $\cPqb$ decay can be included in the jet when taking a PF approach. 

The separate components of the PF algorithm are explained below, including a discussion on how
particle flow is used to suppress pileup effects.


\subsubsection{Iterative tracking}

The tracker provides a very good momentum resolution for charged hadrons, better than the
calorimeters up to a \pt of several hundred \GeV. It also gives a precise measurement of the
direction of charged particles. For these reasons, the tracker is the cornerstone of the PF
algorithm. A tracking efficiency as close to 100\% as possible, while keeping the tracking fake rate
as low as possible, is thus of the utmost importance.
An iterative, Kalman filter based, tracking strategy~\cite{Chatrchyan:2014fea} is used to achieve
this.

The track finding algorithm starts by requiring very tight criteria for the track seeds and
reconstruction quality, leading to a moderate tracking efficiency, but with a negligibly small fake
rate. 
The hits assigned to those tracks are then removed, and the tracking cycle is repeated two times
more for the remaining hits with progressively looser track seeding criteria. The looser criteria
increase the tracking efficiency, and the fake rate is kept low because of the reduced combinatorics
resulting from the removal of hits in the previous iteration. 
During three more iterations, the constraints on the origin vertex are also relaxed, allowing
the reconstruction of tracks associated with secondary vertices. 

With the iterative tracking technique, charged particles with a \pt as low as 150\MeV, as little
as three hits, and originating more than 50\unit{cm} from the beam axis, can be reconstructed with a
fake rate of the order of 1\%. 

\subsubsection{Calorimeter clustering}

The calorimeter clustering in the PF method aims for a high detection efficiency and a separation
of nearby energy deposits. 
The calorimeters are also solely responsible for providing a measurement of the energy and direction
of neutral hadrons and photons, and should provide additional information on charged hadrons in
case the track parameters could not be determined with high precision. 
The clustering is done separately for the ECAL, HCAL and preshower, and separately for barrel and
endcaps.
The algorithm consists of three steps. 
\begin{enumerate}
  \item Cluster seeds are identified as calorimeter cells with energy deposits above a certain
threshold, and larger than their neighbouring cells. 
  \item Topological clusters are grown from the cluster seeds by joining adjacent cells that pass a
chosen minimum energy threshold related to the level of noise in the electronics. 
  \item PF clusters are constructed from the topological clusters using an iterative procedure. Each
cluster seed gives rise to one PF cluster, even when multiple seeds are part of one large
topological cluster. If this happens, the energy of each calorimeter cell is shared among all PF
clusters according to the distance between cluster and cell. During the different iterations, the
PF cluster position is computed as a weighted average of the positions of the cells. As the
cluster position changes, the distance between cluster and cell changes, and thus the energy
sharing, which prompts the recalculation of the cluster position. This continues until the cluster
positions are stable. 
\end{enumerate}

% The purpose of a clustering algorithm in the calorimeters is at least fourfold:
% (ii) separate these neutral particles from energy deposits from charged hadrons; 
% (iii) reconstruct and identify electrons and all accompanying Bremsstrahlung photons; and 


\subsubsection{Link algorithm}

A particle passing through the CMS detector can leave hits in multiple subdetectors, as was
illustrated in Fig.~\ref{fig:cms_slice}, and will thus most likely give rise to multiple PF
elements. There can be a track, one or more calorimeter clusters, or possibly a track in the muon
system. The linking algorithm is designed to link together all elements originating from a single
particle, thereby removing any possible double counting from different subsystems. The quality of a
given link is quantified by the distance between the linked elements. 

A link between a charged particle track and a calorimeter cluster is made by extrapolating the
track to the calorimeters, at a depth corresponding to the expected shower maximum. The track is
linked to a calorimeter cluster if the extrapolated track position falls within the cluster. The
link distance is defined as the distance in the $(\eta,\phi)$ plane between the extrapolated track
position and the cluster position.  
Calorimeter clusters originating from bremsstrahlung photons emitted by electrons are captured by
extrapolating tangents to a track at each intersection between the track and a tracker layer
toward the ECAL. If the extrapolated tangent position is within a cluster, the cluster is linked to
the track as well.

Links between calorimeter clusters are made when the position of a cluster in the higher granularity
calorimeter falls within the boundaries of a cluster in the less granular calorimeter. The link
distance is defined as before. 

A charged-particle track in the tracker and a muon track in the muon system are linked when a global
fit between the two tracks returns an acceptable $\chi^2$, the value of which is used as link
distance. When several of these `global muons' can be fit using a given muon track and several
tracker tracks, only the global muon that returns the smallest $\chi^2$ is retained. 


\subsubsection{Particle reconstruction and identification}

The blocks of linked elements are converted into a set of identified particles by the particle-flow
algorithm. The order of the particle reconstruction follows how clean the signature is. Muons are
reconstructed first, neutral hadrons last. The full particle reconstruction and identification
algorithm proceeds in this way for each block of linked tracks and clusters. 

\begin{enumerate}
  \item Each global muon is added to the list of PF muons if its momentum is compatible with the
momentum determined using tracker information only. The track is then removed from the block. 

  \item The algorithm then proceeds to reconstruct electrons, using a dedicated
method~\cite{Khachatryan:2015hwa}. A Gaussian-Sum Filter is used to refit the candidate electron
track, taking into account the possible energy loss by bremsstrahlung, and follow its trajectory to
the ECAL. Seeds for the time-consuming fitting procedure are chosen only from the subset of tracks
that pass certain identification criteria.
An electron is fully identified if its track matches with an ECAL cluster, and if it passes a set of
tracking and calorimeter requirements. The electron is then added to the PF electron
collection, and the associated electron track and ECAL clusters, including those from
bremsstrahlung, are removed from the block before further processing. 

  \item The remaining tracks are subject to tighter quality criteria, namely the relative
uncertainty on the transverse momentum should be smaller than the relative calorimeter energy
resolution for charged hadrons. The presence of photons and neutral hadrons will be inferred from a
detailed comparison of the track momenta and calorimeter energies. 

  \item For each HCAL cluster all associated charged hadron candidate tracks are found. If a track
traverses more than one HCAL cluster, it is assigned to the closest one. 
  The charged hadron candidate tracks associated with a given HCAL cluster are then matched with
the ECAL clusters. The closest ECAL cluster they traverse is assigned to the charged hadron
candidate. If the track passes through multiple ECAL clusters, those clusters are first ordered by
distance. They are added, one by one, to the charged hadron candidate for as long as the total
calorimetric energy is smaller than the momentum of the charged particle track. 

  \item If the total reconstructed calorimeter energy is significantly smaller than the total
charged particle momentum, there is an inconsistency, and a relaxed muon reconstruction is
performed.
Tracks that fail more stringent track quality criteria are subsequently removed as well. 

  \item If on the other hand the total track momentum is smaller than the total calorimeter energy,
then the remaining tracks are indeed consistent with stemming from charged hadrons. The tracks are
thus added to the list of PF charged hadrons, with the track momentum as charged hadron momentum.

  \item For cases where the track momentum is compatible with the calorimeter energy, the charged
hadron energy and momentum are refit, using both tracker and calorimeter information. This is of
particular interest for high \pt hadrons, where the calorimeters provide a better resolution. 
 
  \item When there is a substantial excess of calorimeter energy compared to the track momentum,
photons and possibly neutral hadrons are reconstructed. First, photons are reconstructed from the
ECAL clusters. If this cannot account for the full excess, neutral hadrons are reconstructed from
the remainder.  
 
  \item Finally, ECAL and HCAL clusters without matching tracks are reconstructed as photons and
neutral hadrons, respectively. 
\end{enumerate}

At the end of the PF sequence, we now have a list of identified particles which can be used to
reconstruct jets. The user can specify which particle types are included in the jet reconstruction. 
By default, isolated muons and electrons are not included.  


\subsubsection{Pileup mitigation techniques}

The presence of pileup causes extra energy deposits and tracks to be overlaid with those of the
hard interaction. This results in a degraded resolution, and less clean signatures.
Pileup vertices are usually separated in space from the vertex of interest. The very precise
tracker system allows these vertices to be reconstructed, and we can use the particle flow
framework to mitigate the effect of in-time pileup, using a technique called \textit{charged
hadron subtraction}~\cite{CMS-PAS-JME-14-001}, or also \texttt{PFNoPileUp}. 

Contamination from pileup events is reduced by discarding charged hadron PF candidates that are
associated to pileup vertices, prior to jet clustering and any further processing.   
The leading primary vertex of the event is the one with the largest value of $\sum
|\pt^{\mathrm{track}}|^2$. The pileup vertices are all other primary vertices for which the number
of degrees of freedom (d.o.f.) in the vertex fit is greater than four. 
Charged hadrons are assigned to a particular vertex according to the compatibility, expressed as
the $\chi^2/\mathrm{d.o.f.}$, of the track with the proto-vertex reconstructed without the
currently considered track. If $\chi^2/\mathrm{d.o.f.}<20$ for a given track-vertex combination,
the track is associated to that vertex. 
Charged hadron candidates with a track associated to a pileup vertex are removed. All other tracks,
even when not associated to a vertex, are retained. 

Since charged hadron subtraction relies on the tracker information, it can only be applied within
the tracker acceptance, $|\eta| < 2.5$. It has been shown that this technique is successful at
removing a large portion of pileup jets, in addition to removing a significant part of the pileup
contribution to jets from the hard interaction. This results in an improved energy and angular
resolution. 


% The average pileup energy due to neutral hadrons is computed
% event-by-event and subtracted from the energy when computing lepton isolation and jet energy.  The
% energy subtracted is  the average pileup energy per unit area (in $\Delta\eta \times \Delta\phi$)
% times the jet area~\cite{Fastjet1, Fastjet2}.

