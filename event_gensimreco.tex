\chapter[Event generation, simulation, reconstruction]{Event generation, simulation and
reconstruction \label{chap:event_generation}}

In the previous chapter I discussed how $\Pp\Pp$ collisions are produced by the LHC, and
how they are detected by CMS. This chapter will elaborate on the different
steps needed to actually use the collisions for physics analysis. 
First, I will explain more details about the collisions, or \textit{events}, themselves. I will
discuss the separate components an event consists of, as well as touch upon how events are described
mathematically. 

In order to understand what we observe in the data, which might contain signals of new physics, it
is important to know how Standard Model processes will appear in the detector. To achieve this we
generate those processes using Monte Carlo generation techniques, incorporating everything that is
known about how the Standard Model works. The principal options that are available to generate
events will be discussed in Section~\ref{sec:event_generation}. 
For each generated collision, we obtain a set of final state particles according to the specified
physics process. This could for example be the particles that result from the production and decay
of a top quark. 
At this stage we do not know yet how these particles would interact with the detector. That is
taken care of in a next step by the event simulation, as explained in
Section~\ref{sec:event_simulation}. An event simulator mimics how a particle, \eg an electron,
would interact with all the different detector layers, and stores the response of the detector in
the same format as the actual detector data. 

At this point the simulated data and the real data are very similar, but are stored in a raw format,
containing detector hits and energy depositions, rather than physics objects. This format is hard to
use for further analysis. The final step will thus be to perform the event reconstruction. The
purpose of event reconstruction is to convert the raw detector information, be it real or simulated,
into physical objects, such as electrons, muons, photons, charged or neutral hadrons. Each of those
objects comes with a set of defining variables, which can be very basic (e.g. \pt, $\eta$ or $\phi$)
or more complex (e.g. shower shape). The different algorithms and techniques that are used within
CMS for this purpose are detailed in Section~\ref{sec:event_reconstruction}. 

\section{What is an ``event''? \label{sec:event}}

%%%%%%%%%%%%%%%%%%%%%%%%%%%%%%%%
%% What is an event? 
%%%%%%%%%%%%%%%%%%%%%%%%%%%%%%%%

At the LHC we define an \textit{event} as everything that happens in a proton bunch crossing.
These high energy collisions are very complex, often resulting in the production of many hundreds of
particles. An illustration of this complexity is shown in Fig.~\ref{fig:event_full_event}.
A proper description of what happens is impeded by the composite nature of the proton, and by the
strong coupling constant of QCD, the quantum field theory governing hadron
interactions.
Fortunately, it turns out that the full process can be factorized into independent subprocesses,
each taking place at different energy scales and, therefore, distance~\cite{Skands:2011pf}. 


\begin{figure}[p]
  \centering
  \includegraphics[width=0.98\textwidth]{figures/eventreco_event/full_event}
  \caption{Pictorial representation of a $\Pp\Pp$ collision event.
The hard interaction (big red blob) is followed by the decay of the produced particles (small red
blobs).
Additional hard QCD radiation is produced (red) and a secondary interaction takes place (purple
blob) before the final-state partons hadronize (light green blobs) and hadrons decay (dark green
blobs). Photon radiation occurs at any stage (yellow). Figure taken from
Ref.~\cite{Gleisberg:2008ta}
  \label{fig:event_full_event}}
\end{figure}


The process that is usually of most interest is the interaction between the constituents of the two
protons that results in high \pt particles. This is referred to as the \textit{hard interaction}. 
Not every collision produces very hard particles, sometimes protons merely undergo elastic
collisions, resulting in very soft scattering products that do not pass the detection thresholds. In
general, any interaction producing some detectable particles is called a \textit{minimum bias
interaction}~\cite{Field:2012jv}. 

The initial momentum distribution of the partons involved in the hard interaction is described by
\textit{parton distribution functions} (PDF). 
Apart from the hard interaction, the other constituents of the proton can also interact. This
usually results in a spray of softer particles, the \textit{underlying event} (UE). 
Any high momentum particle involved in the collision will emit QCD radiation.
Radiation from particles before the hard interaction is called initial-state-radiation (ISR),
whereas radiation off particles produced in the collision is called final-state-radiation (FSR).

Quarks and gluons produced in the collision cannot stay free; they must hadronize in a time scale
of $\mathcal{O}(\text{10}^{\text{-23}}\second)$. These hadrons, in addition to possible produced
leptons, will then pass through the experiment where they can be detected, and used to find out
what happened in the collision itself. 
A complication for the physicists analyzing the data arises from the very high
instantaneous luminosity at the LHC. During one bunch crossing there are usually up to 20
$\Pp\Pp$ interactions, collectively referred to as \textit{pileup}. Most of these interactions
produce relatively soft particles, but they do add to the overall hadronic activity in an event,
and can obscure the interesting hard process. 
An example of how an event might look like in the CMS detector is shown in
Fig.~\ref{fig:event_display}. 

In the next subsections I will elaborate on how to describe an event in a more mathematical way,
starting from the factorization theorem. These sections are largely based on
Refs.~\cite{Campbell:2006wx,Skands:2011pf,Salam_Bautzen,Salam:2010zt,Tung:2001cv}.

\begin{figure}[p]
  \centering
  \includegraphics[width=0.9\textwidth]{figures/eventreco_event/event_display_SUS12024}
  \caption{CMS event display showing five high \pt jets, three of which are tagged as coming from a
$\cPqb$ quark. Figure from~\cite{SUS12024_event_display}.
  \label{fig:event_display}}
\end{figure}


\subsection{Factorization theorems}

The basic problem addressed by factorization theorems~\cite{Collins:1989gx} is how to calculate
cross sections for high energy processes. In general, these cross sections are a combination of
short- and long-distance contributions, and are thus not computable directly using perturbation
theory.
Factorization theorems allow us to derive predictions for these cross sections,
by separating (factorizing) long-distance from short-distance effects. 
The long-distance effects, which cannot be described using perturbation theory - and therefore
referred to as “non-perturbative” effects, 
are encapsulated in parton distribution functions
describing the momentum distribution of partons in a hadron. 
These functions cannot be computed from first principles. Therefore, their form must be extracted
from data by comparing the predictions and measurements of suitable observables, see
Section~\ref{sec:event_pdfs}. Most importantly, the same functions can be used for different
processes for which the factorization theorem holds.
The short-distance hard-scattering cross section can be calculated with perturbation theory
because the QCD coupling strength is small at short distances. 

The factorization theorem applied to the cross section $\sigma$ of a hard scattering initiated by
two hadrons $A$ and $B$, illustrated in Fig.~\ref{fig:event_hard_scatter},
can be expressed in terms of the parton distribution functions $f$, and partonic cross section
$\hat{\sigma}$:
\begin{multline}
  \sigma(s;\alpha_S,\mu_F,\mu_R) = \\ 
  \sum_{a,b} \int_0^1 dx_a \int_0^1 dx_b \,  f_{a/A}(x_a, \alpha_S, \mu_F) \cdot f_{b/B}(x_b,
\alpha_S, \mu_F) \cdot \hat{\sigma}(\hat{s};\alpha_S,\mu_F,\mu_R),
\label{eq:factorization_theorem} 
\end{multline}
 \begin{wrapfigure}{r}{0.4\textwidth}
  \centering
  \vspace{-1eM}
  \includegraphics[width=0.38\textwidth]{figures/eventreco_event/Hardscattering}
  \caption{Diagram of a hard scattering process, showing the parton distribution functions
$f$ and the partonic cross section $\hat{\sigma}$. Figure taken from Ref.~\cite{Campbell:2006wx}.
  \label{fig:event_hard_scatter}}
\end{wrapfigure}
with $f_{a/A}(x_a, \alpha_S, \mu_F)$ the probability that a parton $a$ inside
a hadron $A$ carries a momentum fraction $x_a$, where $s$ is the centre-of-mass energy of the
collision, and $\hat{s} = s x_a x_b$ the partonic centre-of-mass energy.
The strong coupling constant is denoted by $\alpha_S$, the factorization scale by $\mu_F$ and the
 renormalization scale by $\mu_R$.
The factorization scale defines the (arbitrary) boundary between what is viewed as a short-distance
versus a long-distance interaction. The renormalization scale is also an arbitrary scale, which is
needed to regulate the divergencies that appear when computing the partonic cross section in a
perturbative expansion. 
Often, the choice $\mu_F = \mu_R$ is made for convenience.
Obviously, the physical cross section cannot possibly depend on these scales, which are artifacts of
calculations. Therefore, we expect an accurate prediction of the cross section to be insensitive to
these scales. However, when making computations,
%The left-hand side of the equation is in reality independent of the arbitrary choices for $\mu_F$
%and $\mu_R$. When making computations, 
a dependence can be introduced because we cannot
compute the partonic cross sections up to all orders in $\alpha_S$. 

The validity of this factorization theorem can be proven mathematically for certain classes of
processes (it is only approximately true for many other processes), but can also be understood
intuitively in the context of the parton model. 
Hadrons are viewed as composite objects, made up of partons held together by their interactions in
a virtual partonic state.
Let's consider how a hadron-hadron scattering at high energy and momentum transfer looks like in the
centre-of-mass frame. The hadrons appear Lorentz contracted in the direction of the collision, and
their internal interactions are time dilated. The higher the centre-of-mass energy, the longer the
lifetime of any virtual partonic state will be, and the shorter the time needed for a parton of one
hadron to cross the other hadron. At high enough energy, the time needed to traverse the hadron
will be much shorter than the lifetime of any partonic state. Each parton inside the hadron can
thus be viewed as carrying a definite fraction $\xi$ of the hadron's momentum in the centre-of-mass
frame, and so it makes sense to talk about the partons interacting rather than the hadrons.  
Therefore, the interactions of the partons inside a hadron, which occur at time-dilated time
scales before or after the hard scattering, cannot interfere with the interaction of a parton
from one hadron with a parton from the other hadron. 
The cross section for hadron scattering may thus be computed by
combining probabilities, rather than amplitudes, and factorization is reached. 




\subsection{Parton distribution functions \label{sec:event_pdfs}}

A key ingredient to the computation of any cross section at the LHC, is the set of parton
distribution functions describing the momentum distributions of partons within the proton, as
is visible from Eq.~\ref{eq:factorization_theorem}. 
Physically, PDFs express the fact that hadrons are composite objects, with a time-dependent
structure. The PDFs themselves are not physical observables, but rather a more fundamental quantity
derived from the actual physical observables such as structure functions, which can be measured
in e.g. deep-inelastic scattering processes. 
Parton distribution functions can be extracted from this data, but only within a specific
factorization scheme, order by order in perturbation theory. 
At leading order they have a very simple physical interpretation: if the PDF for a given particle
species $p$ is given by $p(x,Q^2)$, then $p(x,Q^2) dx$ is the probability that a probe of
virtuality $Q^2$ will find a particle of flavour $p$ inside the proton, with a
momentum fraction between $x$ and $x + dx$ of the full proton momentum. 
At higher orders, the PDFs no longer have a clear probabilistic interpretation. 

Parton distribution functions satisfy sum rules, governed by the valence content of the
hadrons. For a proton we find for the PDFs of the $u$, $d$,
and $s$ (anti)quarks: 
\begin{align}
  \int_0^1 dx \left( u(x,Q^2) - \bar{u}(x,Q^2)\right) &= 2, \\[-2pt]
  \int_0^1 dx \left( d(x,Q^2) - \bar{d}(x,Q^2)\right) &= 1, %\\[-2pt]
%  \int_0^1 dx \left( s(x,Q^2) - \bar{s}(x,Q^2)\right) &= 0, 
\end{align}
\begin{equation}
%  \int_0^1 dx \left( u(x,Q^2) - \bar{u}(x,Q^2)\right) &= 2, \\[-2pt]
%  \int_0^1 dx \left( d(x,Q^2) - \bar{d}(x,Q^2)\right) &= 1, \\[-2pt]
  \int_0^1 dx \left( s(x,Q^2) - \bar{s}(x,Q^2)\right) = 0, 
\end{equation}
while we also need the momentum weighted sum of the PDFs of all particle species to sum to unity, in
order to satisfy momentum conservation, 
\begin{equation}
  \int_0^1 dx \, x \left( g(x,Q^2) + \sum_i [u_i(x,Q^2) + \bar{u}_i(x,Q^2)]\right) = 1 ,
\end{equation}
where $g(x,Q^2)$ is the gluon PDF, and $i$ runs over all quark flavours. 

Looking back to Eq.~\ref{eq:factorization_theorem}, we note that the parton distribution
functions depend on the chosen factorization scale. The dependence of the PDFs on the scale $Q^2$
is described by the DGLAP equations, which can be viewed as renormalization group equations in
analogy with those for the running coupling constant. 
The DGLAP equations, and thus the PDF evolution, are governed by the so-called splitting functions,
$P_{ab}$ , that model the rate for a particle of type $a$ to undergo a collinear splitting to
produce a particle of type $b$. 

Parton distribution functions are obtained from global fits to a wide variety of data from
many experiments, among which are measurements of deep-inelastic scattering at HERA, and Drell-Yan
or inclusive jet production at the Tevatron and the LHC. 
Since there is only a partial kinematic overlap between these data and the region in $(x,Q^2)$ space
where we want to use the PDFs, for example to model the production of supersymmetric particles,
the DGLAP evolution is essential for the successful prediction of PDFs in the LHC domain. 
The splitting functions are now known up to NNLO precision, which reduced the uncertainties in the
evolution dramatically, from 30\% down to about 2\%.

As illustration of the PDF scale dependence, we show in Fig.~\ref{fig:NNPDF} the full
set of parton distribution functions for two $Q^2$ scales, as derived by the \textsc{NNPDF}
collaboration~\cite{Ball:2012cx}. 
The gluon PDF is seen to dominate for small momentum fractions, and this domination increases as the
scale increases. This simply means that as we probe the proton with higher energy, i.e. to smaller
length scales, we will find more and more gluons. 

\begin{figure}[tpb]
  \centering
  \includegraphics[width=0.8\textwidth]{figures/eventreco_event/nnpdf23_nnlo_allpdfs}
  \caption{Parton distribution functions for different parton species inside the proton for two
values for the momentum transfer, as obtained by the \textsc{NNPDF}
collaboration~\cite{NNPDF_website,Ball:2012cx}. 
  \label{fig:NNPDF}}
\end{figure}

Most global analyses, such as the one performed by the \textsc{CTEQ} or \textsc{MSTW}
collaborations, use a generic form for the parameterization of the quark and gluon distributions at
some reference value $Q_0$, usually chosen in the range $1-2\GeV$:
\begin{equation}
  f(x,Q_0) = A_0 x^{A_1} (1-x)^{A_2} P(x; A_3, \ldots) .
\end{equation}
The parameter $A_1$ is associated with small-$x$ behaviour, while $A_2$ is associated with
large $x$. These two factors are, in general, not sufficient to describe the quark or gluon
distribution functions. The term $P(x; A_3, \ldots)$ is a smooth function, depending on one or more
parameters, that is introduced to add more flexibility to the PDF parameterization. The various PDF
collaborations usually make different choices for the form of $P(x)$. 
The coefficients $A_i$ are then usually determined by comparing theoretical predictions with the
data using $\chi^2$ fits. 
The \textsc{NNPDF} collaboration uses a different approach, and parameterizes $f(x,Q_0)$ by a
neural network. 
Once the PDFs are determined for the reference value $Q_0$, they are generated for the full
$(x,Q^2)$ plane using the DGLAP evolution equations. 

Apart from having an estimate for the nominal values of the PDFs in a given kinematic range, it is
also important to understand the uncertainties, especially for the gluon PDF, which is the hardest
to access experimentally, and is constrained mostly by the $Q^2$ evolution of the quark PDFs. 
A common method of estimating parton distribution uncertainties is to compare different published
parton distributions. This poses a problem since most published PDF sets adopt similar
assumptions such that the differences between these sets do not fully capture the uncertainties that
actually exist. Several techniques exists that remedy this, and they are used by the PDF
collaborations to publish a proper set of uncertainties with each PDF set.  



\subsection{Hard interaction \label{sec:event_hard_interaction}}

% mention following things:
% - complications arising from MPI
% - make link to section on matrix element generators which will contain more info on the practical
% implementation


The partonic scattering cross section describes the hard interaction, and contains all the
short-range effects. For the interaction between two partons $a$ and $b$, resulting in final state
$F$ plus anything else ($X$), it can be written as
\begin{equation}
  \hat{\sigma}_{ab\rightarrow F+X} = \frac{1}{2\hat{s}_{ab}} | \mathcal{M}_{ab\rightarrow
F+X}|^2(\Phi_F, \mu_F, \mu_R), 
\end{equation}
with $\hat{s} = (p_a + p_b)^2$ the usual Mandelstam variable, and 
$|\mathcal{M}|^2$ the matrix element squared for the process $a b \rightarrow F+X$, appropriately
summed and averaged over the relevant helicities and colours. 
The matrix element depends on the final state phase space $\Phi_F$, and should be evaluated at the
factorization scale $\mu_F$ and renormalization scale $\mu_R$.

The partonic cross section can be expanded in a perturbative series in the strength of the
QCD coupling constant $\alpha_S$, 
\begin{equation}
  \hat{\sigma} = \hat{\sigma}_0 + \hat{\sigma}_1 \alpha_S + \hat{\sigma}_2 \alpha_S^2 + \ldots
\end{equation}
The first couple of terms in the perturbative expansion are the terms that so-called
\textit{fixed-order predictions} deal with. They are conceptually quite simple; it is easy to state
which contributions are included, and by including further orders in the expansion one can expect to
see improvement in the accuracy of the predictions. 
At leading order we can still compute many inclusive cross sections by hand, although this is often
automated, by computing all the relevant tree-level Feynman diagrams and integrating over the
appropriate phase space. 
At next-to-leading order we can distinguish between two sets of extra contributions to the
originally considered process: the real emissions resulting in extra quarks or gluons in the final
state, and the virtual loops which do not change the number of final state particles, but do impact
the cross section. 

It is important to note that the complexity of the computations increases mostly with the number of
extra loops, rather than the actual order in $\alpha_S$. 
Tree-level diagrams can be calculated up to quite high final-state multiplicities, $\sim\,$10,
while one-loop diagrams have only been used for processes with up to 3 or sometimes 4 final-state
particles, and two-loop diagrams are available only for $2 \rightarrow 1$ type processes, such as
$\Pp\Pp \rightarrow W$.
When going to higher orders in the perturbative series, it also becomes more and more tricky to
properly combine, i.e. cancel, divergencies between 2-loops, 1-loop and tree-level diagrams. 
Examples of tree-level diagrams that become divergent is anything produced in association with
extra quarks or gluons which could become soft or collinear. These divergencies must be cancelled
by the corresponding loop divergencies, otherwise unitarity is violated. 
In practice this is not always easy to do, especially when experimental cuts need to be applied. 
The standard technique to deal with this issue is through a \textit{subtraction procedure} which
introduces suitable counterterms, adding them to the real diagrams, and subtracting them from the
loops, hereby removing the divergencies from the calculation.


Even though the switch from LO to NLO predictions introduces some technical complications, it is
still worthwhile to do so, where possible, because of the reduced uncertainties. 
At NLO, the dependence on the factorization and renormalization scales is much smaller, as this
relies on the missing higher order terms, which for NLO contain an extra factor $\alpha_S$, and are
thus smaller.

The strength of the NLO correction is often encapsulated in a so-called NLO k-factor, which is
defined as the ratio of the NLO cross section to the LO cross section. K-factors for many processes
can be as large as 1.5, much larger than the 10\% effect one would expect from considering only the
extra factor of $\alpha_S$. The reason for this is that the terms accompanying that factor of
$\alpha_S$ can be quite large. 
The calculated k-factors can vary for different kinematic regimes within the same process, so care
needs to be taken when attempting to scale a LO cross section obtained for some particular corner of
phase space. 


As explained in the introduction of this chapter, we also need to generate full events for
which we can simulate the detector response, rather than only computing inclusive cross sections.
This chain often starts by generating events for the hard process only, of course taking into
account the parton distribution functions as well. 
Until recently, event generators based on the perturbative calculation of matrix elements could only
generate events up to LO. 
With the release of \textsc{MG5\_aMC@NLO}~\cite{Alwall:2014hca}, the automated generation of events
at NLO precision is now possible for almost any Standard Model process.
More details on how this is done in practice are presented in
Section~\ref{sec:event_matrix_element_generators}.






\section{Event generation \label{sec:event_generation}}

%%%%%%%%%%%%%%%%%%%%%%%%%%%%
%% Event generation 
%%%%%%%%%%%%%%%%%%%%%%%%%%%%

% add info on Madgraph, pythia, jet matching etc

\subsection{Matrix element generators}

\subsection{Parton shower and hadronization}

\subsection{Jet matching}


\section{Event simulation \label{sec:event_simulation}}

%%%%%%%%%%%%%%%%%%%%%%%%%%%%
%% Event simulation
%%%%%%%%%%%%%%%%%%%%%%%%%%%%

Simulation of the CMS detector response is done in one of two ways: a `full' simulation
(FullSim) which is time-consuming but very accurate, or a `fast' simulation (FastSim), which is much
faster, but for which approximations have been made. The following subsections will explain the
basic concepts and use cases for both of these options. 

\subsection{CMS Full Simulation using Geant4 \label{subsec:fullsim}}

The purpose of the CMS full simulation~\cite{Banerjee:2007zz,Banerjee:2011zzc,Banerjee:2012ge} is to
provide a very accurate description of how particles interact with the CMS detector. These simulated
event samples can then be used to understand and demonstrate the power of analysis methods which
will later be applied to the real data. They can also be used to derive calibrations, efficiencies
and resolutions for high level physics objects in case the available data are not sufficient

The inputs to the detector simulation are the hadronized particles from the event generator. These
particles and their four-vectors are passed to the simulation software in the \textsc{HepMC}
format~\cite{Dobbs:2001ck}. 
The particles are then propagated through the CMS detector using the \GEANTfour toolkit~\cite{G4}.
\GEANTfour contains a large collection of electromagnetic and hadronic physics processes describing
the interaction of particles with material, and the resulting energy loss. Examples of this are
bremsstrahlung, photon conversion, nuclear interactions, multiple scattering, and showering. In
case new particles are produced in these interactions, they are also propagated further. 

A key component for the success of the simulation is the precise implementation of the detector
geometry, material budget, and magnetic field strength. Not only the active layers need to be
accounted for, but also the cooling, cabling, and support structures. 
Choices must be made for the level of detail to include in the simulation geometry in order to
optimize computation speed versus the correctness of the simulation.
Within the CMS software framework there is a single, common implementation of the full detector
geometry for both the simulation and the reconstruction, ensuring full compatibility.

The accuracy of the detector implementation has been thoroughly tested with dedicated test beam
data, cosmic muon data, and early LHC collision data. Overall, very good agreement was
found, and adjustments were made where necessary. 

Pileup interactions are also added at this stage of the processing chain. A library of simulated
hits of minimum bias events is prepared beforehand, and then used to overlay a number of extra
interactions onto the signal event according to a specified pileup scenario. 
As the bunch crossing time is shorter than the time needed for particles of a given event to fully
traverse the CMS detector, we also need to take into account \textit{out-of-time pileup}, which is
the effect of previous or subsequent bunch-crossings on the current event. Out-of-time pileup is
modelled by modifying the timing of the detector hits when overlaying a minimum bias interaction. 

The next step is the conversion of all energy depositions, from signal and pileup events, in the
sensitive detector volumes to electronic signals. Electronic noise is also included during this
\textit{digitization} step.
The output of the digitization is simulated data in a format identical to that of real collision
data read directly from the detector. The L1 and HLT decisions, and the objects used to arrive at
them, are also included in the simulated data. 
From this point onwards the simulated events go through the same reconstruction steps as the real
data, as will be explained in Section~\ref{sec:event_reconstruction}. 





\subsection{CMS Fast Simulation \label{subsec:fastsim}}

% explain what approximations are made, why it is necessary, especially in SUSY searches


The CMS fast simulation~\cite{fastsim,Rahmat:2012fs} is a faster alternative to the full simulation
explained above. 
It is intended to be used for physics analyses that require the generation of many
samples to span a wide phase-space region, e.g. the vast SUSY simplified model scans. 
A set of approximations is made, resulting in a speed
increase of about a factor twenty. 

The interactions simulated in FastSim are electron bremsstrahlung, photon conversion, charged
particle energy loss by ionization, charged particle multiple scattering, nuclear interactions, and
electron, photon and hadron showering. The various CMS subsystems are modelled in different ways,
with various levels of approximation. 

The tracking detector is the most complicated CMS subsystem, and the one for which the most
approximations are made, including for the geometry.
The tracker is modelled by 30 thin nested cylinders, and is assumed to be made of pure silicon,
uniformly distributed over each layer. The thickness of each layer in terms of interaction lengths
was tuned to reproduce the number of bremsstrahlung photons above a certain threshold as observed in
FullSim.
Charged particles are propagated between two detector surfaces according to the magnetic field, and
experience multiple scattering and energy loss by ionization. The intersections between the
trajectories and each tracker layer define the position of the simulated hits that are then
converted into reconstructed hits with a certain efficiency, which is determined from FullSim. 

The showers of electrons and photons which hit the electromagnetic calori\-meter are simulated as if
the ECAL were a homogeneous medium. This is a reasonable approximation because the ECAL crystals
are organized to have almost no gaps in between them. 
Electrons and photons at rapidity values not covered by the electromagnetic calorimeter ($|\eta| >
3$) are propagated directly to the forward hadron calorimeter.

Charged and neutral hadrons are propagated to the start of the ECAL, HCAL and HF after their
interactions with the tracker layers. Their energy response is derived from the full
simulation. First the energy is smeared according to energy resolution measured in FullSim for
single pions. Then, this smeared energy is distributed in the calorimeters using parameterized
longitudinal shower profiles. 

Muons propagate through the tracker, the calorimeters, the solenoid, and the muon chambers. 
Both muons coming directly from the main interaction vertex and those produced inside the
tracker from the decay of another particle are included. 
The calorimeter response is treated similarly to that of charged hadrons. In the muon systems the
only processes that are taken into account are multiple scattering and energy loss by ionization.

The reconstruction of FastSim tracker hits into tracks is done in a different, faster, way compared
to FullSim or data. The truth information is used to group hits together that come from the same
particle. Consequently, FastSim does not have fake tracks. Two hits in the same place are also not
merged to be one reconstructed hit, as would be the case for the real detector readout. 
Despite these simplifications, good agreement between FastSim and FullSim track reconstruction is
observed. 





\section{Event reconstruction \label{sec:event_reconstruction}}

%%%%%%%%%%%%%%%%%%%%%%%%%%%
% Event reconstruction
%%%%%%%%%%%%%%%%%%%%%%%%%%%

The CMS event reconstruction aims to provide a global event description in terms of electrons,
muons, photons, charged hadrons, and neutral hadrons. Information from all subdetectors is combined
to achieve a fully consistent picture of the event. The algorithm that implements this, is called
particle flow (PF), and is documented in Refs.~\cite{CMS-PAS-PFT-09-001,PF}. 

Particle flow relies heavily on the high resolution sicilon tracker, and high granularity ECAL. The
basic building blocks are tracks, constructed using a very efficient tracking algorithm, and
clusters of calorimeter energy deposits. 
These two elements are then linked together using a linking algorithm. The actual particle flow
algorithm uses the links to create a list of reconstructed and identified particles, that are
subsequently used for physics analysis. A description of each of these steps will be given in
Section~\ref{sec:event_reco_pf}.

The particles that are reconstructed by the PF algorithm can be further refined to suit the needs
of individual analyses or analysis groups. More stringent identification criteria are usually
required, so that the misidentification rate, and thus background rate, is substantially reduced.
All physics objects that will be used in the razor boost analysis will be listed in
Section~\ref{sec:event_objects}.

Reconstructed events are sometimes affected by spurious detector noise or reconstruction failures,
leading to anomalous amounts of missing transverse momentum, or to very high \pt jets. 
These events are filtered out by various targeted cleaning algorithms, as discussed in
Section~\ref{sec:event_cleaning}.

The PF event reconstruction is applied in the same way to data and simulation, resulting in an
overall very good agreement between them. Some quantities cannot be adequately modelled in
simulation, however. Event reweighting techniques are employed to account for discrepancies that
might influence a physics analysis. In Section~\ref{sec:event_reweighting} I will discuss the
standard event reweighting techniques that are applied in the razor boost analysis. 


\subsection{Particle flow \label{sec:event_reco_pf}}


The particle flow method reconstructs particles, the PF candidates, by combining information from
the inner tracker, the calorimeters, and the muon system.  Each PF candidate is assigned to one of
five object categories: muons, electrons, photons, charged hadrons, and neutral hadrons.  

By simultaneously using information from all subdetectors, overlaps between object collections are
removed. In a calorimeter only reconstruction, for example, photons and electrons are also
reconstructed as jets. 
Since about 65\% of the jet energy is carried by charged hadrons, inclusion of the tracker
information in the jet reconstruction results in a much improved energy resolution. For jets
originating from a $\cPqb$ quark this is even more striking, as energy carried away by muons from
the $\cPqb$ decay can be included in the jet when taking a PF approach. 

The separate components of the PF algorithm are explained below, including a discussion on how
particle flow is used to suppress pileup effects.


\subsubsection{Iterative tracking}

The tracker provides a very good momentum resolution for charged hadrons, better than the
calorimeters up to a \pt of several hundred \GeV. It also gives a precise measurement of the
direction of charged particles. For these reasons, the tracker is the cornerstone of the PF
algorithm. A tracking efficiency as close to 100\% as possible, while keeping the tracking fake rate
as low as possible, is thus of the utmost importance.
An iterative, Kalman filter based, tracking strategy~\cite{Chatrchyan:2014fea} is used to achieve
this.

The track finding algorithm starts by requiring very tight criteria for the track seeds and
reconstruction quality, leading to a moderate tracking efficiency, but with a negligibly small fake
rate. 
The hits assigned to those tracks are then removed, and the tracking cycle is repeated two times
more for the remaining hits with progressively looser track seeding criteria. The looser criteria
increase the tracking efficiency, and the fake rate is kept low because of the reduced combinatorics
resulting from the removal of hits in the previous iteration. 
During three more iterations, the constraints on the origin vertex are also relaxed, allowing
the reconstruction of tracks associated with secondary vertices. 

With the iterative tracking technique, charged particles with a \pt as low as 150\MeV, as little
as three hits, and originating more than 50\unit{cm} from the beam axis, can be reconstructed with a
fake rate of the order of 1\%. 

\subsubsection{Calorimeter clustering}

The calorimeter clustering in the PF method aims for a high detection efficiency and a separation
of nearby energy deposits. 
The calorimeters are also solely responsible for providing a measurement of the energy and direction
of neutral hadrons and photons, and should provide additional information on charged hadrons in
case the track parameters could not be determined with high precision. 
The clustering is done separately for the ECAL, HCAL and preshower, and separately for barrel and
endcaps.
The algorithm consists of three steps. 
\begin{enumerate}
  \item Cluster seeds are identified as calorimeter cells with energy deposits above a certain
threshold, and larger than their neighbouring cells. 
  \item Topological clusters are grown from the cluster seeds by joining adjacent cells that pass a
chosen minimum energy threshold related to the level of noise in the electronics. 
  \item PF clusters are constructed from the topological clusters using an iterative procedure. Each
cluster seed gives rise to one PF cluster, even when multiple seeds are part of one large
topological cluster. If this happens, the energy of each calorimeter cell is shared among all PF
clusters according to the distance between cluster and cell. During the different iterations, the
PF cluster position is computed as a weighted average of the positions of the cells. As the
cluster position changes, the distance between cluster and cell changes, and thus the energy
sharing, which prompts the recalculation of the cluster position. This continues until the cluster
positions are stable. 
\end{enumerate}

% The purpose of a clustering algorithm in the calorimeters is at least fourfold:
% (ii) separate these neutral particles from energy deposits from charged hadrons; 
% (iii) reconstruct and identify electrons and all accompanying Bremsstrahlung photons; and 


\subsubsection{Link algorithm}

A particle passing through the CMS detector can leave hits in multiple subdetectors, as was
illustrated in Fig.~\ref{fig:cms_slice}, and will thus most likely give rise to multiple PF
elements. There can be a track, one or more calorimeter clusters, or possibly a track in the muon
system. The linking algorithm is designed to link together all elements originating from a single
particle, thereby removing any possible double counting from different subsystems. The quality of a
given link is quantified by the distance between the linked elements. 

A link between a charged particle track and a calorimeter cluster is made by extrapolating the
track to the calorimeters, at a depth corresponding to the expected shower maximum. The track is
linked to a calorimeter cluster if the extrapolated track position falls within the cluster. The
link distance is defined as the distance in the $(\eta,\phi)$ plane between the extrapolated track
position and the cluster position.  
Calorimeter clusters originating from bremsstrahlung photons emitted by electrons are captured by
extrapolating tangents to a track at each intersection between the track and a tracker layer
toward the ECAL. If the extrapolated tangent position is within a cluster, the cluster is linked to
the track as well.

Links between calorimeter clusters are made when the position of a cluster in the higher granularity
calorimeter falls within the boundaries of a cluster in the less granular calorimeter. The link
distance is defined as before. 

A charged-particle track in the tracker and a muon track in the muon system are linked when a global
fit between the two tracks returns an acceptable $\chi^2$, the value of which is used as link
distance. When several of these `global muons' can be fit using a given muon track and several
tracker tracks, only the global muon that returns the smallest $\chi^2$ is retained. 


\subsubsection{Particle reconstruction and identification}

The blocks of linked elements are converted into a set of identified particles by the particle-flow
algorithm. The order of the particle reconstruction follows how clean the signature is. Muons are
reconstructed first, neutral hadrons last. The full particle reconstruction and identification
algorithm proceeds in this way for each block of linked tracks and clusters. 

\begin{enumerate}
  \item Each global muon is added to the list of PF muons if its momentum is compatible with the
momentum determined using tracker information only. The track is then removed from the block. 

  \item The algorithm then proceeds to reconstruct electrons, using a dedicated
method~\cite{Khachatryan:2015hwa}. A Gaussian-Sum Filter is used to refit the candidate electron
track, taking into account the possible energy loss by bremsstrahlung, and follow its trajectory to
the ECAL. Seeds for the time-consuming fitting procedure are chosen only from the subset of tracks
that pass certain identification criteria.
An electron is fully identified if its track matches with an ECAL cluster, and if it passes a set of
tracking and calorimeter requirements. The electron is then added to the PF electron
collection, and the associated electron track and ECAL clusters, including those from
bremsstrahlung, are removed from the block before further processing. 

  \item The remaining tracks are subject to tighter quality criteria, namely the relative
uncertainty on the transverse momentum should be smaller than the relative calorimeter energy
resolution for charged hadrons. The presence of photons and neutral hadrons will be inferred from a
detailed comparison of the track momenta and calorimeter energies. 

  \item For each HCAL cluster all associated charged hadron candidate tracks are found. If a track
traverses more than one HCAL cluster, it is assigned to the closest one. 
  The charged hadron candidate tracks associated with a given HCAL cluster are then matched with
the ECAL clusters. The closest ECAL cluster they traverse is assigned to the charged hadron
candidate. If the track passes through multiple ECAL clusters, those clusters are first ordered by
distance. They are added, one by one, to the charged hadron candidate for as long as the total
calorimetric energy is smaller than the momentum of the charged particle track. 

  \item If the total reconstructed calorimeter energy is significantly smaller than the total
charged particle momentum, there is an inconsistency, and a relaxed muon reconstruction is
performed.
Tracks that fail more stringent track quality criteria are subsequently removed as well. 

  \item If on the other hand the total track momentum is smaller than the total calorimeter energy,
then the remaining tracks are indeed consistent with stemming from charged hadrons. The tracks are
thus added to the list of PF charged hadrons, with the track momentum as charged hadron momentum.

  \item For cases where the track momentum is compatible with the calorimeter energy, the charged
hadron energy and momentum are refit, using both tracker and calorimeter information. This is of
particular interest for high \pt hadrons, where the calorimeters provide a better resolution. 
 
  \item When there is a substantial excess of calorimeter energy compared to the track momentum,
photons and possibly neutral hadrons are reconstructed. First, photons are reconstructed from the
ECAL clusters. If this cannot account for the full excess, neutral hadrons are reconstructed from
the remainder.  
 
  \item Finally, ECAL and HCAL clusters without matching tracks are reconstructed as photons and
neutral hadrons, respectively. 
\end{enumerate}

At the end of the PF sequence, we now have a list of identified particles which can be used to
reconstruct jets. The user can specify which particle types are included in the jet reconstruction. 
By default, isolated muons and electrons are not included.  


\subsubsection{Pileup mitigation techniques}

The presence of pileup causes extra energy deposits and tracks to be overlaid with those of the
hard interaction. This results in a degraded resolution, and less clean signatures.
Pileup vertices are usually separated in space from the vertex of interest. The very precise
tracker system allows these vertices to be reconstructed, and we can use the particle flow
framework to mitigate the effect of in-time pileup, using a technique called \textit{charged
hadron subtraction}~\cite{CMS-PAS-JME-14-001}, or also \texttt{PFNoPileUp}. 

Contamination from pileup events is reduced by discarding charged hadron PF candidates that are
associated to pileup vertices, prior to jet clustering and any further processing.   
The leading primary vertex of the event is the one with the largest value of $\sum
|\pt^{\mathrm{track}}|^2$. The pileup vertices are all other primary vertices for which the number
of degrees of freedom (d.o.f.) in the vertex fit is greater than four. 
Charged hadrons are assigned to a particular vertex according to the compatibility, expressed as
the $\chi^2/\mathrm{d.o.f.}$, of the track with the proto-vertex reconstructed without the
currently considered track. If $\chi^2/\mathrm{d.o.f.}<20$ for a given track-vertex combination,
the track is associated to that vertex. 
Charged hadron candidates with a track associated to a pileup vertex are removed. All other tracks,
even when not associated to a vertex, are retained. 

Since charged hadron subtraction relies on the tracker information, it can only be applied within
the tracker acceptance, $|\eta| < 2.5$. It has been shown that this technique is successful at
removing a large portion of pileup jets, in addition to removing a significant part of the pileup
contribution to jets from the hard interaction. This results in an improved energy and angular
resolution. 


% The average pileup energy due to neutral hadrons is computed
% event-by-event and subtracted from the energy when computing lepton isolation and jet energy.  The
% energy subtracted is  the average pileup energy per unit area (in $\Delta\eta \times \Delta\phi$)
% times the jet area~\cite{Fastjet1, Fastjet2}.




\subsection{Physics object identification \label{sec:event_objects}}

The event selection is an integral part of any physics analysis. It determines which events are
used, and thus what processes contribute to the data sample. This in turn drives how the
backgrounds are estimated, what the sensitivity will be, etcetera. 
An event selection is most easily described in terms of particles, \eg two electrons, no muons, at
least four jets, as this is the closest to how we think about a given process.  
The particle flow technique described in the previous section is very compatible with this approach,
given that it reconstructs a fully consistent set of identified particles out of the detector hits. 
However, a more thorough selection of the PF objects is needed in order to ensure that their
behaviour is understood, and to ensure that the selected events are not dominated by
misidentified particles, or detector artefacts. 
The physics object groups (POG's) within the CMS Collaboration are in charge of providing general
recommendations on how to define each object. These recommendations are based on extensive studies,
and are applicable for most analyses, thus reducing the workload for the analysis teams.
In the following paragraphs all the standard objects that will be used in the razor boost analysis
are discussed.


%%%%%%%%%%%%%%%%%%%%%%%%%%%%%%%%%%%%%%%%%%%%%%%%%%%%%%%%%%%%%%%%%%%%%%%%%%%%%%%%%%%%%%%%%%%%%%%%%

%%%%%%%%%%%%%%%%%%%%%%%%%%%%%%%%%%%
%%  Object identification
%%%%%%%%%%%%%%%%%%%%%%%%%%%%%%%%%%%



% The average pileup energy due to neutral hadrons is computed
% event-by-event and subtracted from the energy when computing lepton isolation and jet energy.  The
% energy subtracted is  the average pileup energy per unit area (in $\Delta\eta \times \Delta\phi$)
% times the jet area~\cite{Fastjet1, Fastjet2}.
% this corrects energy and momentum, not substructure
% TODO: move to jet and lepton sections

% 
% Missing transverse energy, which is used in the calculation of the razor variable $\mr$, is 
% defined to be the negative sum of the transverse momenta of all the particle flow objects in an
% event.  Loosely identified and isolated electrons with $\pt > 5$~\GeV and $|\eta| < 2.5$ and muons
% with $\pt > 5$\GeV and $|\eta| < 2.4$ are used both to suppress backgrounds in our signal region
%and
% in the definition of the control regions.  A tight definition of isolated leptons (electrons with
% $\pt > 10$~\GeV and $|\eta| < 2.5$ and muons with $\pt > 10$~\GeV and $|\eta| < 2.4$) defines a
% control region enriched in $\cPZ \rightarrow \ell \ell $ events, from which we estimate the
% systematic uncertainty in the predicted number of $\cPZ \rightarrow \nu \nu$ events in the signal
% region. Any electron candidates with $1.44 < |\eta| < 1.57$ are rejected since the transition
%region
% between barrel and endcap calorimeters is less well-instrumented.
% In order to suppress the decays of taus and other leptons that fail the loose selection, events
%that
% have isolated tracks with $\pt > 10$\GeV and track-primary vertex distance along the beam
%direction
% $dz < 0.05$ are rejected.

\subsubsection{Primary vertices \label{sec:object_vertex}}

We require at least one {\it good} primary vertex to be reconstructed in each event. 
This vertex should be associated with at least four charged-particle tracks. It should also lie
within 24\cm of the origin of the CMS coordinate system along the beam direction, and within 2\cm
in the plane transverse to the beam. 
These requirements, translated to the CMS nomenclature, are summarized in
Table~\ref{tab:object_vertex}.
In case there are multiple good vertices, we choose the vertex with the highest value of $\sum
\pt^2$ of associated tracks to be the leading primary vertex in the event. This vertex is
taken as a reference to reconstruct the event, e.g. to perform the track subtraction for pileup
removal, for which we use the charged hadron subtraction algorithm, as explained before.

\begin{table}[htdp]
\caption{Vertex selection criteria. \label{tab:object_vertex}}
\begin{center}
\begin{tabular}{l l}
\toprule
\texttt{\small isFake()} & $= 0$ \\
\texttt{\small ndof()} & $> 4$ \\
\texttt{\small z()} & $< 24\cm$ \\
\texttt{\small position.Rho()} & $< 2\cm$ \\
\bottomrule
\end{tabular}
\end{center}
\end{table}


\subsubsection{Jets \label{sec:object_jets}}

Most analyses are interested primarily in the quarks and gluon produced in the hard interaction, or
in the decay of heavy particles, such as top quarks or $\W$ bosons. 
However, through the process of parton showering and hadronization, the few initial quarks and
gluons turn into a multitude of hadrons. 
Hadrons from a given initial quark or gluon can usually be found close together, they form a
\textit{jet}. The proper description of jets, and the jet definitions that are used to reconstruct
them, relies on two properties: infrared, and collinear safety~\cite{Salam:2009jx}.
It is important that a jet definition returns the same set of final jets regardless of whether a
parton underwent a collinear or soft splitting. If this is not the case, i.e. the jet definition is
infrared or collinear unsafe, then one finds that divergencies in the theoretical computation of
jet cross sections do not vanish. 

A jet definition comprises two parts: the jet algorithm that defines in which order particles are
grouped together, and the recombination scheme that defines how to combine the momenta of the
to-be-merged particles. 
For the latter, the most common choice is to simply add the four-vectors of the particles, which
then gives rise to massive jets. 
For the jet algorithm there are many choices. Here I will focus solely on the anti-$k_\textrm{T}$
algorithm~\cite{antikt}, which is the default jet algorithm used by CMS.
As for most sequential recombination algorithms, one defines distances $d_{ij}$ between particles
$i$ and $j$ (or pseudojets if particles have been combined before),  and distances $d_{iB}$ between
particle $i$ and the beam.
The distance measures are in this case given by
\begin{align}
  d_{ij} &= \min \left(\frac{1}{p_{\mathrm{T,i}}^2}, \frac{1}{p_{\mathrm{T,j}}^2}\right)
\frac{\Delta R_{ij}^2}{R^2}, \\
  d_{iB} &= \frac{1}{p_{\mathrm{T,i}}^2},
\end{align}
where $\Delta R_{ij}^2 = (y_i - y_j)^2 + (\phi_i - \phi_j)^2$ and $R$ is a tuneable parameter
determining the size of the jets. The rapidity $y$ of a particle is given by,
\begin{equation}
  y = \frac{1}{2} \ln{\frac{ E + p_z }{ E - p_z }} .
\end{equation}
The jet clustering proceeds by identifying the smallest of all distances. If it is a $d_{ij}$, we
recombine particles $i$ and $j$, while if it is $d_{iB}$, we move $i$ from the list of particles to
the list of final jets. All distances are then recalculated and the procedure is repeated until no
particles are left.
The anti-$k_\textrm{T}$ algorithm results in mostly circular jets, reminiscent of the older cone jet
algorithms that are no longer used because they are not infrared and collinear safe.

The input to the jet clustering are the PF candidates that pass the charged hadron subtraction. 
The clustering itself is done with the anti-$k_\textrm{T}$ algorithm with size parameter $R=0.5$
(AK5), as implemented in \textsc{FastJet 3.0.1}~\cite{Cacciari:2011ma}.
We apply the standard loose identification criteria to the resulting jets, as defined by the
requirements listed in Table~\ref{tab:object_jets}. 

Unfortunately, the calorimeter response to incident particles is not uniform. It it, therefore, not
straightforward to translate the measured jet energy to the true particle energy, which is what we
want to use to do our analysis. A set of jet energy scale corrections -- scalings of the
jet four-momentum depending on jet \pt, $\eta$ and flavour -- are applied to both data and
simulation in order to achieve a proper mapping to the particle level. The difference between the
reconstructed and particle-level jet energy is called the \textit{offset} in what follows.
Jet energy corrections within CMS are taken care of in a sequential way, each level of correction
taking care of a different effect~\cite{JEC}. 

First, the residual effect from pileup is removed using the so-called L1 corrections. The effects
of charged hadrons from in-time pileup have already been largely reduced by the charged hadron
subtraction method. The effect of neutral particles and out-of-time pileup is removed at this stage
using a slightly modified version of the \textit{jet area method}~\cite{Fastjet1,Fastjet2}.
This method uses the effective area of the jets, $A$, multiplied by the average energy density
in the event, $\rho$, to calculate the energy to be subtracted from the jets.
Both real and simulated jets are first corrected with a $\pt$, $\eta$, and number of primary
vertices dependent offset correction determined in simulation. For data events, an additional
data/simulation scale factor is derived from ZeroBias data to correct for remaining $\eta$ dependent
discrepancies.
Figure~\ref{fig:JEC_L1} shows the size of the energy offset between reconstructed and particle level
jets, before and after the L1 corrections have been applied. A clear reduction of the overall offset
is observed. 

\begin{figure}[tpb]
  \centering
  \includegraphics[width=0.4\textwidth]{figures/eventreco_objects/OffMeantnpuRef_BB_ak5pfchs}
  ~
  \includegraphics[width=0.4\textwidth]{figures/eventreco_objects/OffMeantnpuRef_BB_ak5pfchsl1}
  \caption{The offset shown on the $y$-axis in these plots is defined as the difference in
transverse momentum for a reconstructed jet with added pileup and the same jet without pileup.
The lefthand side shows the offset as a function of the generated \pt of a jet before the L1
corrections have been applied, and the righthand side shows the offset after pileup
corrections. Different markers represent different levels of pileup. Figures taken from
Ref.~\cite{JEC_plots}.
  \label{fig:JEC_L1}}
\end{figure}


The second level of corrections, the L2 Relative corrections, are designed to make the jet response
flat in $\eta$. Since the simulation of the detector response is very detailed, see
Section~\ref{sec:event_simulation}, the jet response is in fact very well modelled in simulation,
which is why it is used for the bulk of the jet energy corrections. 
MC truth information is used to correct a jet at arbitrary $\eta$ relative to a jet
in the central area ($|\eta|<1.3$). 

\begin{figure}[tpb]
  \centering
\includegraphics[width=0.6\textwidth]
{figures/eventreco_objects/CorrectionVsEta_Overview_TDR_ak5pfl1_L2L3}
  \caption{ The size of the L2 and L3 corrections as a function of jet $\eta$ for three reference
transverse momentum values: 30\GeV (white hollow circles), 100\GeV (red squares) and 300\GeV (blue
circles). Figure taken from Ref.~\cite{JEC_plots2}.
  \label{fig:JEC_L23}}
\end{figure}


Then the L3 Absolute corrections, which flatten the jet response with respect to \pt, are applied.
They are derived from simulation, and correct the jet energy back to the particle level, such that
on average the \pt of a reconstructed jet matches that of a jet clustered using generator level
particles,
\begin{equation}
  <\pt(\mathrm{reco})_{\mathrm{corr}}> {=} <\pt(\mathrm{gen})>
\end{equation}
These are the final corrections applied to jets from simulated events. 

Data events are further corrected by the L2L3 Residual jet energy scale corrections to take care of
the small differences between data and simulation. These corrections are \pt and $\eta$ dependent,
and only correct the relative energy scale. The absolute energy scale was found to be well modelled
in the simulation. A dedicated, data-driven approach is employed, using data samples of dijet,
$\gamma+$jet, and $\cPZ+$jet events. 

% TODO Add information on jet corrections and pileup subtraction


% The jets are corrected for pile-up effects in a two step process.  
% First charged hadron particle-flow candidates that have been associated with a pile-up vertex are
% removed from the list of particles to be clustered using the {\tt PFNoPileUp} algorithm.  
% The jets are then clustered and corrected for the L2 and L3 corrections, taking into
% account the charged-hadron removal. 
% The remaining PU energy is subtracted by applying the event-by-event quantity $\pi \rho (\Delta
% R)^2$, where $\Delta R$ is the jet size and $\rho$ is the average density from PU events, as
% computed by {\tt FastJet} using only neutral hadron particle-flow candidates.  

After all corrections are applied, jets are required to have $\pt > 30\GeV$ and $|\eta| < 2.4$.  

  






\begin{table}[htdp]
\caption{Jet selection criteria. \label{tab:object_jets}}
\begin{center}
\begin{tabular}{l l}
\toprule
\pt & $> 30\GeV$ \\
$|\eta|$ & $< 2.4$ \\
\midrule
\texttt{\small neutralHadronEnergyFraction()} & $< 0.99$ \\
\texttt{\small neutralEmEnergyFraction()} & $< 0.99$ \\
\texttt{\small nConstituents()} & $> 1$ \\
\texttt{\small chargedHadronEnergyFraction()} & $> 0$ \\
\texttt{\small chargedMultiplicity()} & $> 0$ \\
\texttt{\small chargedEmEnergyFraction()} & $< 0.99$ \\
\bottomrule
\end{tabular}
\end{center}
\end{table}

The AK5 jets defined here will be used for most aspects of the razor boost analysis, except for the
reconstruction of boosted hadronic $\W$-candidates. 
Section~\ref{sec:boost_wtag} provides details on the dedicated jet treatment that is used for $\W$
tagging.

\subsubsection{B-Tagging \label{sec:object_btag}}

Jets originating from the hadronization of $\cPqb$ quarks can be distinguished from other jets,
initiated by gluons or light flavor quarks, due to the long lifetime of the $\cPqb$ quark. 
The non-prompt decay of the $B$ hadrons results in a secondary vertex, displaced with respect to
the primary vertex of the hard interaction. 

% TODO add more info on b-tagging algorithm

The ability to distinguish $\cPqb$ jets is especially important for new physics searches. Many new
physics models are associated with production of third generation quarks, whereas this is more rare
in the standard model. For many searches $\cPqb$ jet tagging is an essential tool in suppressing
the background from multijet or vector boson production. 

In the razor boost analysis $\cPqb$ tagging will also be employed. We will use the combined
secondary vertex (CSV) algorithm at two working points~\cite{btag7TeV,btag8TeV,BTagWP}, which are
shown on Table~\ref{tab:object_btag}. 
The Loose working point (CSVL), corresponding to a misidentification rate of $\sim$10\% and
efficiency of $\sim$85\%, will be used to veto $\cPqb$ jets, whereas the Medium working point
(CSVM), corresponding to a misidentification rate of $\sim$1\% and a typical efficiency of
$\sim$70\% , is used to select $\cPqb$ jets. 

\begin{table}[htdp]
\caption{Working points for the combined secondary vertex $\cPqb$ jet tagger.
\label{tab:object_btag}}
\begin{center}
\begin{tabular}{l l}
\toprule
Working point & Discriminator value \\
\midrule
Medium & $> 0.679$ \\
Loose & $> 0.244$ \\
\bottomrule
\end{tabular}
\end{center}
\end{table}
% 
% As will be explained in section~\ref{sec:selection}, we define our signal and control regions
% based on the number of b-tagged jets. 
% As the b-tagged jet multiplicity distribution is not exactly the same in data as in simulation, we
% need to apply appropriate Data/MC scale factors to the simulation. These scalefactors and their
% prescription have been provided by the BTag POG \cite{BTagSF1,BTagSF2}. 
% Whenever an explicit selection based on the number of b-tagged jets is made, the btag scale
% factors are applied to the simulation. 
% For more detailed information on the scale factors and their associated uncertainties, we refer to
% section~\ref{sec:btag_uncertainties}. 

% TODO Decide where to put the scale factor information

\subsubsection{Muons \label{sec:object_muon}}

Muons are identified using two different working points, a loose selection and a tight selection,
both of which will be detailed below. 


% Currently we mainly use the loose definition in the analysis, both for vetoing, and for selecting
% single muon events for the control regions enriched in TTJets and WJets. The tight selection is
% only used to define a control region enriched in $Z\rightarrow ll$ events, from which we derive a
% systematic uncertainty on the predicted number of $Z\rightarrow\nu\nu$ events in our signal
% region.

The \textbf{loose muon selection} that will be employed was developed especially for events with a
large amount of hadronic activity, where the standard identification criteria were observed to lose
efficiency, resulting in less background suppression when vetoing the presence of muons. 
The details and performance of this optimized selection is documented in
Ref.~\cite{CMS-AN2011-498}. 
The main feature is the use of a so-called \textit{directional} isolation.
The isolation of a particle is a measure of how far it is from other activity in the detector. The
leptons we are interested in, those originating in the hard interaction, are usually separated from
other activity, e.g. jets. This is not the case for misidentified muons or for muons from the decay
of heavy-flavour jets. Directional isolation is designed to have a better rejection of leptons from
these heavy-flavour jet decays, and is defined as
\begin{equation}
\overrightarrow{\mathrm{ISO}}(R) \equiv \sum_{\Delta R_{i} < R} \delta_{i}^{2}\pt{}_{i} ,
\end{equation}
where the sum is over all other particles $i$ within $\Delta R_{i}<R$ of the muon direction,
and $\delta_{i}$ is the angle between particle $i$ and the $\pt$-weighted centroid position
($\delta_{c}$) of all such particles in $(\eta,\phi)$ space. That is, if $\Delta\phi_i$ and
$\Delta\eta_i$ are respectively the difference in $\phi$ and $\eta$ angles between particle $i$ and
the muon, then:
\begin{eqnarray*}
\vec{e}_{i} & \equiv & \frac{1}{\sqrt{\Delta\phi_{i}^{2}+\Delta\eta_{i}^{2}}}\left(\begin{array}{c}
\Delta\phi_{i},\\
\Delta\eta_{i}
\end{array}\right),\\
\vec{\delta}_{c} & = & \sum_{\Delta R_{i}<R}\pt{}_{i}\vec{e}_{i},\\
\delta_{i} & = &
\angle(\vec{\delta}_{c},\vec{e}_{i})=\arccos(\vec{\delta}_{c}\cdot\vec{e}_{i}/|\vec{\delta}_{c}|),
\end{eqnarray*}
where $\vec{e}_{i}$ is the unit vector specifying particle $i$'s relative location in $(\eta,\phi)$
space with respect to the considered muon, as illustrated in Fig.~\ref{fig:object_directional_iso}.
Because of the weighting by $\delta_{i}^{2}$, the value for the directional isolation tends to be
larger for muons that are near the jet core, e.g. in case of leptonic $\cPqb$ decays, compared to
the more convential isolation definition which does not use this weighting. 

\begin{figure}[htpb]
  \centering
  \includegraphics[width=0.8\textwidth]{figures/eventreco_objects/directional_iso_cartoon}
  \caption{Illustration of ingredients used in the computation of directional isolation for a prompt
muon, denoted by a star, near some particles from a jet, denoted by points, in the $(\eta,\phi)$
plane. For prompt leptons $\delta_i$ tends to be small, especially for the high-\pt particles near
the core of the jet. Figure taken from Ref.~\cite{CMS-AN2011-498}.
  \label{fig:object_directional_iso}}
\end{figure}

Apart from the isolation, the identication criteria themselves are also altered from the standard
Loose Muon ID from the POG in order to further optimize the muon identification in environments
with large hadronic activity. 
Loose muons are reconstructed using either the global muon algorithm or the tracker-only
algorithm. 
Global muons are required to pass the {\tt GlobalMuonPromptTight} quality criteria,
and to have at least two muon chambers containing segments uniquely matched to its inner track. 
Tracker-only muons are required to pass the {\tt TMLastStationTight} criteria, which require the
muon to have compatible hits in the last muon chamber. 
All selected muons are then required to pass the selection listed in
Table~\ref{tab:object_loosemuon}. 
Some aspects of the selection depend on the muon $\pt$ and $\eta$; these are summarized in
Table~\ref{tab:object_loosemuon_cuts}.

\begin{table}[p]
\caption{Loose muon definition. }
\begin{center}
{\small
\begin{tabular}{l l}
\toprule
\pt & $> 5\GeV$ \\
$|\eta|$ & $< 2.4$ \\
\midrule
\texttt{\footnotesize innerTrack().hitPattern().numberOfLostHits()} & $\leq 1$ if $\pt < 20\GeV$ \\
                                                      & $\leq 4$ if $\pt \geq 20\GeV$ \\
$|\texttt{\footnotesize innerTrack().dxy(vertex.position())}|$ & $\pt$- and $\eta$-dependent\\
$|\texttt{\footnotesize muonBestTrack().dz(vertex.position())}|$ & $\pt$- and $\eta$-dependent\\
\midrule
$\overrightarrow{\mathrm{ISO}}(R=0.2)$ & $\pt$- and $\eta$-dependent \\
\bottomrule
\end{tabular}
}
\end{center}
\label{tab:object_loosemuon}
\end{table}

\begin{table}[p]
\caption{Details of the $\pt$ dependent thresholds employed in the loose muon selection.}
\begin{center}
  \begin{tabular}{l cccccc }
      \toprule
      Muon $\pt$  & $d_{xy} (\cm)$ & $d_{xy} (\cm)$ & $d_z (\cm)$ & $d_z (\cm)$ &
$\overrightarrow{\mathrm{ISO}}(0.2)$ &
$\overrightarrow{\mathrm{ISO}}(0.2)$ \\
      (\GeV) & Barrel & Endcap & Barrel & Endcap & Barrel & Endcap \\
      \midrule
      0 - 5          & 0.052 & 0.037 & 0.054 & 0.076 & 1.5  & 2    \\
      5 - 10         & 0.041 & 0.018 & 0.042 & 0.082 & 3    & 2.5  \\
      10 - 25        & 0.029 & 0.013 & 0.028 & 0.098 & 7    & 7.5  \\
      15 - 20        & 0.014 & 0.015 & 0.034 & 0.1   & 10.5 & 9    \\
      20 - 40        & 0.021 & 0.021 & 1     & 0.1   & 15.5 & 13.5 \\
      40 - 80        & 0.04  & 0.2   & 1     & 1     & 32.5 & 19   \\
      80 - 140       & 0.1   & 0.2   & 1     & 1     & 54.5 & 37   \\
      140 - 200      & 0.1   & 0.2   & 1     & 1     & 87   & 65.5 \\
      \bottomrule
    \end{tabular}
\end{center}
\label{tab:object_loosemuon_cuts}
\end{table}

 
The \textbf{tight muon selection} follows the recommendation from the Muon POG~\cite{MuonID}.
In addition to the identification criteria, we also require the tight muon to be isolated. 
Here we do not use directional isolation, but rather the more standard particle-based relative
isolation. 
This isolation, denoted $I_\mu$, is calculated using the PF candidates in a cone of size $\Delta R =
0.4$ around the muon. Charged-hadron candidates associated with pileup vertexes are not taken into
account in the calculation of the isolation. However, they are used to estimate the remaining
contribution to the isolation coming from neutral hadrons associated with pileup. This contribution
is then subtracted. 
The isolation definition is given by:
\begin{equation}
I_\mu = \frac{I_{Charged} + I_{Neutral} + I_{\gamma} - \Delta\beta\cdot I_{Charged}^{PU}}
             {\pt^\mu} , 
\label{eqn:iso}
\end{equation}
where $I_{Charged}$, $I_{Neutral}$, and $I_{\gamma}$ are computed as the sum of the \pt of the
charged hadrons, neutral hadrons and photons, respectively, in a cone of size $\Delta R = 0.4$
around the muon. The parameter $\Delta\beta$ is set to 0.5, and $I_{Charged}^{PU}$ is the estimated
contribution from pileup computed as the sum of the \pt of the charged hadrons associated with
pileup vertices.
The tight muon isolation requirement is $I_\mu < 0.15$.
A summary of the tight muon selection can be found in Table~\ref{tab:object_tightmuon}. 

\begin{table}[p]
\caption{Tight muon definition. }
\begin{center}
{\small
\begin{tabular}{l l}
\toprule
\pt & $> 10\GeV$ \\
$|\eta|$ & $< 2.4$ \\
\midrule
\texttt{\footnotesize isPFMuon()} & $= 1$ \\
\texttt{\footnotesize isGlobalMuon()} & $= 1$ \\
\texttt{\footnotesize globalTrack().normalizedChi2()} & $< 10$ \\
\texttt{\footnotesize globalTrack().hitPattern().numberOfValidMuonHits()} & $> 0$ \\
\texttt{\footnotesize track().hitPattern().trackerLayersWithMeasurement()} & $> 5$ \\
\texttt{\footnotesize innerTrack().hitPattern().numberOfValidPixelHits()} & $> 0$ \\
\texttt{\footnotesize numberOfMatchedStations()} & $> 1$ \\
$|\texttt{\footnotesize innerTrack().dxy(vertex.position())}|$ & $< 0.2\cm$ \\
$|\texttt{\footnotesize muonBestTrack().dz(vertex.position())}|$ & $< 0.5\cm$ \\
\midrule
$I_\mu =$ [\texttt{\footnotesize pfIsolationR04().sumChargedHadronPt()}& \\
\hspace{0.9cm} $+$ max(0., \texttt{\footnotesize pfIsolationR04().sumNeutralHadronPt()}  & \\
\hspace{2.7cm} $+$ \texttt{\footnotesize pfIsolationR04().sumPhotonPt()}  & \\
\hspace{2.7cm} $-$ 0.5 $\cdot$ \texttt{\footnotesize pfIsolationR04().sumPUPt()}) & \\
\hspace{0.9cm} ] / \pt & $< 0.15$ \\ 
\bottomrule
\end{tabular}
}
\end{center}
\label{tab:object_tightmuon}
\end{table}

 

\subsubsection{Electrons \label{sec:object_electron}}

Similar to the muon selection, we identify electrons using two different working points, a loose
selection, and a tight selection. 

% Currently we mainly use the loose definition in the analysis, both for vetoing, and for selecting
% single electron events for the control regions enriched in TTJets and WJets. The tight selection
% is only used to define a control region enriched in $Z\rightarrow ll$ events, from which we
% derive a systematic uncertainty on the predicted number of $Z\rightarrow\nu\nu$ events in our
% signal region.


The \textbf{loose electron selection} uses directional isolation as described in the previous
section, and fully documented in Ref.~\cite{CMS-AN2011-498}. A summary of the complete
loose electron selection is given in Table~\ref{tab:object_looseelectron}, with the details of
the $\pt$- and $\eta$-dependent requirements listed in Table~\ref{tab:object_looseelectron_cuts}. 

\begin{table}[p]
  \caption{Loose electron definition.}
  \begin{center}
  {\small 
    \begin{tabular}{l l l l}
      \toprule
      & Condition & Barrel & Endcap \\
      \midrule
      \pt & & $ > 5 \GeV$ & $> 5\GeV$ \\
      $|\eta|$ & & $ < 1.442$ & $1.556 - 2.5$ \\
      \midrule
      \texttt{\footnotesize gsfTrack().numberOfLostHits()} & $\pt < 20\GeV$ & $= 0$ & $= 0$ \\
      \texttt{\footnotesize gsfTrack().hitPattern().numberOfValidPixelHits()} & $\pt < 10\GeV$ &
$\geq 2$ & $\geq 1$ \\
      $|\texttt{\footnotesize gsfTrack().dz(vertex.position())}|$ & & \multicolumn{2}{l}{$\pt$- and
$\eta$-dependent}\\
      \midrule
      $\overrightarrow{\mathrm{ISO}}(R=0.3)$, calculated from charged particles only & &
\multicolumn{2}{l}{$\pt$- and $\eta$-dependent} \\
      $\overrightarrow{\mathrm{ISO}}(R=0.2)$, barrel only, calculated using all particles & &
\multicolumn{2}{l}{$\pt$- and $\eta$-dependent} \\
      \bottomrule
    \end{tabular}
    }
  \end{center}
  \label{tab:object_looseelectron} 
\end{table}


\begin{table}[p]
  \caption{Details of the $\pt$ dependent thresholds employed in the loose electron selection.}
  \begin{center}
  \begin{tabular}{ l ccccc }
      \toprule
      Electron $\pt$ & $d_z (\cm)$ & $d_z (\cm)$ &
$\overrightarrow{\mathrm{ISO}}(0.3,\textrm{charged})$ &
$\overrightarrow{\mathrm{ISO}}(0.3,\textrm{charged})$ & $\overrightarrow{\mathrm{ISO}}(0.2)$ \\
      (\GeV) & Barrel & Endcap & Barrel & Endcap & Barrel \\
      \midrule
      0 - 5          & 0.03 & 0.09 & 0.5  & 0.5  & 2    \\
      5 - 10         & 0.05 & 0.09 & 1.5  & 2.5  & 4.25 \\
      10 - 25        & 0.05 & 0.09 & 4.5  & 6.5  & 8.75 \\
      15 - 20        & 0.05 & 0.11 & 7.5  & 9    & 11   \\
      20 - 40        & 0.2  & 1    & 10   & 10.5 & 20.8 \\
      40 - 80        & 1    & 1    & 18.5 & 18.5 & 200  \\
      80 - 140       & 1    & 1    & 44   & 66.5 & 200  \\
      140 - 200      & 1    & 1    & 81.5 & 70   & 200  \\
      \bottomrule
    \end{tabular}
  \end{center}
  \label{tab:object_looseelectron_cuts}
\end{table}

The \textbf{tight electron selection} is in accordance with the recommendations of the EGamma POG
\cite{ElectronID}. A summary of the selection can be found in table~\ref{tab:object_tightelectron}.
We also require to electron to be isolated. The isolation is calculated using the PF candidates in a
cone of size $\Delta R = 0.3$ around the electron, and then corrected with an estimate of the
median energy from pileup as calculated with the {\tt FastJet} algorithm in a similar way to the
jet corrections explained in Sec.~\ref{sec:object_jets}. We require that this corrected isolation,
relative to the $\pt$ of the electron is less than 0.15.

\begin{equation}
I_e = \frac{ I_{Charged} + \max(0, I_{NeutralHad} + I_{\gamma} - A \rho ) }{\pt^e}
\end{equation}

% TODO: add more info on the pileup correction

\begin{table}[p]
\caption{Tight electron definition. }
\begin{center}
{\small
\begin{tabular}{l l l}
\toprule
& Barrel & Endcap \\
\midrule
\pt & $> 10\GeV$ & $> 10\GeV$\\
$|\eta|$ & $< 1.442$ & $1.556 - 2.5$ \\
\midrule
$|$\texttt{\footnotesize deltaEtaSuperClusterTrackAtVtx()}$|$ & $< 0.004$ & $< 0.005$ \\
$|$\texttt{\footnotesize deltaPhiSuperClusterTrackAtVtx()}$|$ & $< 0.030$ & $< 0.020$ \\
\texttt{\footnotesize sigmaIetaIeta()} & $< 0.010$ & $< 0.030$ \\
\texttt{\footnotesize hadronicOverEm()} & $< 0.120$ & $< 0.100$ \\
1.0/\texttt{\footnotesize ecalEnergy()} - \texttt{\footnotesize eSuperClusterOverP()/ecalEnergy()} &
$< 0.050$ &
$< 0.050$ \\
\texttt{\footnotesize gsfTrack().trackerExpectedHitsInner().numberOfHits()} & $\le 0$ & $\le 0$ \\
\texttt{\footnotesize passConversionVeto()} & $= 1$ & $= 1$ \\
$|\texttt{\footnotesize innerTrack().dxy(vertex.position())}|$ & $< 0.02\cm$ & $< 0.02\cm$\\
$|\texttt{\footnotesize gsfTrack().dz(vertex.position())}|$ & $< 0.1\cm$ & $< 0.1\cm$ \\
\midrule
$I_e$ & $<0.15$ & $< 0.15$ \\
\bottomrule
\end{tabular}
}
\end{center}
\label{tab:object_tightelectron}
\end{table}


\subsubsection{Isolated tracks \label{sec:object_isolatedtrack}}

In order to suppress the decays of both taus and other leptons that do not pass the loose
selection, we can veto events for which an isolated track is present~\cite{CMS-AN2013-089}. 
Isolated tracks are selected from the charged PF candidates with $\pt > 10\GeV$ and
longitudional track-primary vertex distance of $d_z < 0.05\cm$. They are required to have a
relative isolation in a cone of $\Delta R = 0.3$ of less than 0.1. 
In the razor boost analysis the isolated track veto will only be applied in the hadronic event
selections, and not in the control regions which require the presence of a lepton. 

\begin{table}[htdp]
\caption{Isolated track selection. }
\begin{center}
\begin{tabular}{l l}
\toprule
\pt & $> 10\GeV$ \\
\midrule
\texttt{charge()} & $> 0$ \\
$d_z({\rm PV, track})$ & $< 0.05\cm$ \\
$I_{\textrm{track}_i} = \frac{\sum_{j \neq i} \pt{}_j }{ \pt{}_i }$ & $< 0.1$ \\
\bottomrule
\end{tabular}
\end{center}
\label{tab:isolatedtrack}
\end{table}

\subsubsection{Missing transverse momentum \label{sec:object_met}}

The missing transverse momentum, \ETm, associated with a given event is computed as the negative
vector sum of the transverse momentum of all PF candidates $i$,
\begin{equation}
  \ETm = - \sum_i \pt^i .
\end{equation}
The corrections to the jet energy scale discussed above are propagated to the \ETm as well. 
Within CMS this type of missing transverse momentum is know as type-1 corrected \ETm.

No explicit selection will be placed on \ETm in the razor boost analysis selection, although it is
used in the definition of the razor variable $\rsq$, to be introduced in
Section~\ref{sec:boost_razor}.




\subsection{Event cleaning \label{sec:event_cleaning}}

The full CMS data taking and event reconstruction process is very intricate. Every now and then a
subdetector might not have behaved properly, or a reconstruction algorithm could have failed. 
Events affected by such failures need to be removed from the selection, as they can create
artificially high missing transverse momentum and would then end up in the signal region of many
supersymmetry searches. 
The following cleaning filters are applied:

\begin{itemize}
\item The {\tt EcalDeadCellTriggerPrimitiveFilter}, which removes events where dead cells in the
ECAL produce anomalous activity.
\item The {\tt hcalLaserEventFilter}, which removes events where the HCAL laser produces anomalous
activity.
%\item The {\tt hcalLaserEventFilter2012}, 
\item The {\tt trackingFailureFilter}, which removes events where the tracking algorithm does not
perform properly.
\item The {\tt CSCTightHaloFilter}, which removes events contaminated by beam halo.
\item The {\tt HBHENoiseFilter}, which removes events featuring large hadronic calorimeter noise.
\item The {\tt eeBadScFilter}, which removes events featuring high amplitude anomalous pulses due
to bad ECAL super-crystals.
\item The {\tt trkPOGFilters}, which remove events due to track reconstruction anomalies, such as
events with partly aborted track reconstruction and events affected by the Strip Tracker coherent
noise.
\item The {\tt primaryVertexFilter}, which removes events that do not have a good primary vertex.
\item The {\tt noscrapingFilter}, which removes events with a large multiplicity of low quality
tracks.
\end{itemize}

More details on these filters can be found in Ref.~\cite{metfilters}. The effect on the \ETm
distribution is shown in Fig.~\ref{fig:event_metcleaning}. There is a very clear reduction in the
\ETm tail in data, which brings data and simulation in agreement. 
The {\tt CSCTightHaloFilter} and {\tt HBHENoiseFilter} filters are not applied for simulated samples
that are passed through the fast CMS detector simulation because the necessary input collections are
not produced.

\begin{figure}[tpb]
  \centering
  \includegraphics[width=0.6\textwidth]{figures/eventreco_objects/METTail}
  \caption{ The PF \VEtmiss distribution for events passing a dijet selection without cleaning
filters applied (open markers), with cleaning filters applied (filled markers), and simulated
events (filled histograms). Figure taken from Ref.~\cite{Khachatryan:2014gga}.
  \label{fig:event_metcleaning}}
\end{figure}

\paragraph{HCAL noise filter}

In addition to the standard filters listed above, we also use an extra cleaning selection designed
to remove events with spurious HCAL noise originating in the outer barrel of the HCAL. 
Energy deposits in the HO are included in the computation of the missing transverse momentum using 
the particle flow algorithm, but are not included in the missing transverse energy
obtained from calorimeter information only. 
A selection requiring no substantial discrepancy between the two \ETm definitions is thus effective
at reducing the contribution of these noisy events. 

We reject events in which the PF missing transverse energy vector, $\VEtmiss(\text{PF})$, is
flipped with respect to the calorimeter based one, $\VEtmiss(\text{Calo})$. 
To accomplish this we compute the absolute value of the difference in polar angle,
 $|\Delta\phi_{\text{PF,Calo}}|$, taken in the range $[0,2\pi[$, and defined as
\begin{equation}
|\Delta\phi_{\text{PF,Calo}}| = \min \left ( \phi^{\text{PF}} - \phi^{\text{Calo}},   2\pi -
\phi^{\text{PF}} + \phi^{\text{Calo}} \right) ,\\
\end{equation}
with 
\begin{equation}
\phi^{\text{PF/Calo}} = \textrm{arctan} \left( \frac{\VEtmiss(\text{PF/Calo})|_y}
{\VEtmiss(\text{PF/Calo})|_x} \right) . \\
%\phi^{\text{Calo}} &= \textrm{arctan}\left( \frac{\VEtmiss(\text{Calo})|_y}
%{\VEtmiss(\text{Calo})|_x} \right) .
\end{equation}
Events for which $|\Delta\phi_{\text{PF,Calo}}|$ falls in a 1 radian window centred around $\pi$
are removed. 
\begin{equation}
\bigl| |\Delta\phi_{\text{PF,Calo}}| - \pi \bigr| < 1
\label{eqn:dphicut}
\end{equation}


\subsection{Event reweighting \label{sec:event_reweighting}}

The generation and simulation of events are tuned to mimic the data. However, the complete data
taking conditions, in particular the pileup profile, are not fully known before data
taking starts. It is thus impossible to mimic the data in all aspects. 
Furthermore, the event generation itself is also not perfect. 
Many details of the hadronization process are still unknown, and state of the art event
generators can only compute hard physics processes up to maximally NLO precision,
whereas data contains all orders.  
All these effects can lead to discrepancies between the observed data and the simulation.  

To correct for some of these imperfections, event reweighting prescriptions have been developed. I
have already mentioned some correction factors that should be applied to various objects, such as
the jet energy scale corrections for jets. 
In the next subsections I will cover the reweightings that have to be applied to the full event,
rather than a particular object. These include the corrections for mismodelling of the pileup
distribution, the initial state radiation, and the top quark \pt spectrum for the
$t\bar{t}$ simulation.

%%%%%%%%%%%%%%%%%%%%%%%%%%%%%
%% Event reweighting 
%%%%%%%%%%%%%%%%%%%%%%%%%%%%%

\subsubsection{Pileup reweighting \label{sec:event_pileup}}

The distribution of the number of pileup interactions is different in data with respect to
simulation. Given that the number of pileup interactions can have an influence on various aspects of
the reconstruction, such as the identification of primary vertices or lepton isolation, 
the simulated events should be reweighted such that their pileup distribution matches that of
data~\cite{pileup_twiki}.

The pileup distribution in data is provided centrally by the Physics Validation Group for each
data taking period. This distribution depends on the total \Pp\Pp inelastic cross section, sometimes
referred to as the \textit{minbias} cross section~\cite{Field:2012jv}. 
In simulation, the pileup distribution is taken from truth information, through the variable
\texttt{trueNumInteractions}. 
The pileup weights are computed as the ratio of the normalized pileup distributions in data and
simulation, and should be applied to all simulated events.
The distribution of the pileup in data and simulation, and the corresponding pileup weight is
shown in Figure~\ref{fig:pileup_comparison}. 

\begin{figure}[htpb]
 \centering
 \includegraphics[width=0.48\textwidth]{figures/eventreco_reweighting/pileup_comparison}
 ~
 \includegraphics[width=0.48\textwidth]{figures/eventreco_reweighting/pileup_weight_comparison}
\caption{[left] Comparison of the distribution of the true number of interactions in data and in
simulation. 
[right] Pileup weight as a function of the number of interactions. 
\label{fig:pileup_comparison}}
\end{figure}

% TODO decide whether to add the data mc comparison for the primary vertices

% 
% As a test of the performance of the pileup reweighting, we can check the agreement between data
% and
% simulation for the distribution of the number of good primary vertices ($PV$) at different
% selection
% levels. 
% We expect to find a reasonable, although not perfect agreement as the vertex reconstruction
% efficiency depends on many things. 
% This comparison is shown in figure~\ref{fig:comparison_PV}. 
% 
% \begin{figure}
%  \includegraphics[width=0.49\textwidth]{figures/Pileup/DataMC_PV_0Lb1Ll}
%  \includegraphics[width=0.49\textwidth]{figures/Pileup/DataMC_PV_g1Mb1Ll}
% \caption{Data/MC comparison plot of the number of good primary vertices after pileup reweighting
% for
% a control region enhanced in $W+$jets (left) and enhanced in $t\bar{t}+$jets (right).
% \label{fig:comparison_PV}}
% \end{figure}
% 


\subsubsection{ISR reweighting \label{sec:event_ISRreweighting}}

Searches for new physics often rely on an initial state boost of the produced system in order to
have experimental acceptance for the signature under consideration. This is especially important
for models featuring a compressed mass spectrum. A high-\pt ISR jet can be used to suppress
background, or the boost can raise the momentum of jets or leptons in the decay chain to a level
that is detectable.
A mismodelling of the initial state radiation, or uncertainty on the modelling, will thus
directly impact the interpretation of these searches. 

A study was performed to investigate how well the ISR is
modelled~\cite{Chatrchyan:2013xna,ISRreweighting} by evaluating the agreement between data and
simulation in the boost \pt for $\cPZ+$jets and $t\bar{t}$ events. 
For $\cPZ+$jets events the boost \pt was measured from the leptonic decay products of the $\cPZ$
boson. For $t\bar{t}$ events the ISR radiation was measured using the hadronic recoil system, which
is computed from all jets except for the $\cPqb$-tagged jets from the $t\bar{t}$ decay. 

It was found that the initial state radiation is not well modelled at high \pt. The mismodelling
can be corrected by applying a scale factor, with associated uncertainty, which was derived from the
observed disagreement. The scale factor depends on the \pt of the system recoiling against the ISR
jets. This system could be e.g. the $t\bar{t}$ system, the $\cPZ$ boson, or the $\tilde{g}\tilde{g}$
system for a SUSY event.  The uncertainty on this scale factor is taken to be the difference
between the scale factor and unity. 
The CMS SUSY group recommendeds to apply this ISR reweighting to all SUSY signal samples.
The prescription is summarized in Table~\ref{tab:ISRreweighting}. 

\begin{table}[htpb]
\caption{ISR reweighting prescription \label{tab:ISRreweighting}}
\begin{center}
\begin{tabular}{c c}
\toprule
\pt of recoiling & Scale factor \\ 
system (\GeV) & \\
\midrule
$\leq 120$ & $1.00 \pm 0.00$ \\
$120 - 150 $ & $0.95 \pm 0.05$ \\
$150-250$ & $0.90 \pm 0.10$ \\
$> 250$ & $0.80 \pm 0.20$ \\
\bottomrule
\end{tabular}
\end{center}
\end{table}

\subsubsection{Top quark \texorpdfstring{$\pt$}{pt} reweighting \label{sec:event_toppt_reweighting}}

Differential top-quark-pair cross section analyses have shown that the shape of the \pt spectrum of 
top quarks in data is softer than predicted by simulation~\cite{toppt,toppt_twiki}. 
To remedy this, events are reweighted based on the \pt of the generator level $t$ and $\bar{t}$
quarks in the $t\bar{t}$ simulation. 

The event weight, $w_{\rm TopPt}$, is computed as a function of the generated \pt of both the top
and anti-top quark
in the event: 
\begin{equation}
w_{\rm TopPt} = \sqrt{ SF_t \cdot SF_{\bar{t}} }
\end{equation}
\begin{equation}
SF(\pt^{gen}) = \exp(a + b\, \pt^{gen})
\end{equation}
with $a = 0.156$ and $b = -0.00137$.
The uncertainty associated with this reweighting is taken to be equal to the full size of the
reweighting, which gives for the up and down variations of the event weight:
\begin{align}
 +1~\sigma &: w_{\rm up} = w_{\textrm{TopPt}} * w_{\textrm{TopPt}}, \\
 -1~\sigma &: w_{\rm down} = 1 .
\end{align}

% 
% In figure~\ref{fig:TopPt} we show the Data/MC comparison for the $M_R$ and $R^2$ distribution in the
% $t\bar{t}+$jets control region (see section~\ref{sec:Tregion}) before and after applying the
% reweighting procedure. 
% We observe that this reweighting greatly improves the agreement between data and simulation. 
% Therefore we will always apply this reweighting to the $t\bar{t}+$jets simulated sample. 
% 
% \begin{figure}[htpb]
% \centering
% \includegraphics[width=0.49\textwidth]{
% figures/DataMC/DataMC_MR_g1Mbg1W1LlmT100_mdPhig0p5_width_noTopPt}
% \includegraphics[width=0.49\textwidth]{
% figures/DataMC/DataMC_R2_g1Mbg1W1LlmT100_mdPhig0p5_width_noTopPt}
% 
% \includegraphics[width=0.49\textwidth]{figures/DataMC/DataMC_MR_g1Mbg1W1LlmT100_mdPhig0p5_width}
% \includegraphics[width=0.49\textwidth]{figures/DataMC/DataMC_R2_g1Mbg1W1LlmT100_mdPhig0p5_width}
% \caption{[top] $M_R$ (left) and $R^2$ (right) distribution before applying the top \pt reweighting
%          [bottom] $M_R$ (left) and $R^2$ (right) distribution after applying the top \pt reweighting
% for the $T$ region as defined in section~\ref{sec:Tregion}
% \label{fig:TopPt}}
% \end{figure}
% 
% 
% 

