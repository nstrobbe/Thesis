\chapter*{Nederlandstalige samenvatting \markboth{NEDERLANDSTALIGE SAMENVATTING}{}}
\addcontentsline{toc}{chapter}{Nederlandstalige samenvatting}



Het Standaard Model van de deeltjesfysica beschrijft onze huidige kennis van de elementaire
deeltjes en hun interacties. Het werd ontwikkeld als een kwantumveldentheorie gedurende de laatste
vijftig jaar, en is grondig getest door een groot aantal experimenten, al dan niet aan een
deeltjesversneller.
Tot nu toe zijn er nog geen significante afwijkingen geobserveerd, en tot voor kort was het Higgs
boson het enige voorspelde deeltje dat nog niet gevonden was. Toen dat moeilijk te vangen deeltje
eindelijk ontdekt werd in 2012, was dit de ultieme overwinning voor het Standaard Model en het
mechanisme van de spontane elektrozwakke symmetriebreking. 

Ongeacht het grote succes van het Standaard Model, kan dit toch niet de definitieve theorie van de
natuur zijn. Hiervoor zijn verschillende redenen, in het bijzonder kan het Standaard Model de
zwaartekracht niet beschrijven. Het model bevat ook geen deeltjes die als kandidaat voor donkere
materie kunnen dienen, en er is ook geen verklaring voor donkere energie. Nu het Higgsdeeltje
ontdekt is, is het zogenaamde hi\"erarchie probleem ook heel relevant geworden. 

De kern van het hi\"erarchie probleem is de vraag waarom de electrozwakke schaal zo veel kleiner is
dan de Planckschaal. Kwantumcorrecties tot de massa van het Higgsdeeltje zijn namelijk kwadratisch
afhankelijk van de cutoff schaal die gebruikt wordt om de loop diagrammen te regularizeren. Als het
Standaard Model geldig is tot op het punt waar een theorie van kwantumgravitatie nodig is, dan is
deze cutoff schaal gelijk aan de Planckschaal. Dit zal dan op zijn beurt resulteren in een enorme
correctie van de Higgs boson massa. Technisch gezien is dit geen probleem, maar het betekent dat er
een zeer grote graad van finetuning aanwezig moet zijn tussen de naakte massa en de correctie om
tot de geobserveerde waarde van 125\GeV te leiden. Deze graad van finetuning wordt algemeen aanzien
als onnatuurlijk, en dus ongewenst. 

Verschillende modellen voor nieuwe fysica zijn ontwikkeld in een poging om deze en andere gebreken
op te lossen. De dag van vandaag is supersymmetrie de meest populaire theorie. Supersymmetrie voegt
een nieuwe fundamentele symmetrie toe, \'e\'en tussen fermionen en bosonen. Het resultaat hiervan
is de invoering van een superpartner deeltje voor elk bestaand deeltje in het Standaard Model.
Doorheen de geschiedenis van de deeltjesfysica hebben symmetrie\"en ons telkens opnieuw geleid tot
een dieper begrip van de natuur. De elegantie van supersymmetrie is dus zeker een aantrekkingspunt,
maar dit is niet het enige. 

Veel supersymmetrische modellen nemen aan dat de R-pariteit behouden is. Hieruit volgt dat het
lichtste supersymmetrische deeltje (LSP) stabiel is, en bovendien een kandidaat is voor donkere
materie. Supersymmetrie, en in het bijzonder de zogenaamde `natuurlijke supersymmetrie', kan ook
een meer natuurlijke verklaring voor het hi\"erachie probleem bieden, waarbij de finetuning
grotendeels verwijderd wordt. Opdat dit zou gebeuren moeten de superpartners van de top quark en het
gluon, de top squark en het gluino, relatief licht zijn, respectievelijk lichter dan 1\TeV en
1.5\TeV. Als deze deeltjes bestaan, zouden ze dus gevonden kunnen worden in de botsingen van de
LHC. 

De mogelijkheid dat de top squark licht kan zijn, diende als motivatie voor verschillende analyses
in de CMS en ATLAS experimenten, waarbij er specifiek gezocht wordt naar de directe productie van
top squark paren. De gevoeligheid van deze analyses verkleint wanneer de massa van de top squark
dat van het LSP benadert, of wanneer het verschil in hun massa's ongeveer de top quark massa
bedraagt. De razor boost analyse die in deze thesis voorgesteld wordt heeft als doel deze gaten in
de gevoeligheid op te vullen. In lijn met de voorspellingen van de natuurlijke supersymmetrie wordt
het bestaan van een relatief licht gluino verondersteld. 

In de razor boost analyse beschouwen we de paar-productie van gluinos, die vervolgens vervallen
naar een top squark en een top quark. De gluinos hebben een mass van ongeveer 1-1.5\TeV, en de top
squarks hebben een massa van enkele honderden \GeV. Het verval van de top squark hangt af van het
veronderstelde massaverschil tussen de top squark en het LSP. Voor kleine massaverschillen vervalt
de top squark naar een charm quark en het LSP, terwijl voor massaverschillen rond de top quark
massa de top squark naar een top quark en het LSP vervalt. De gebruikte modellen zijn zogenaamde
`simplified models', wat zoveel betekent als dat alle andere superpartners een heel grote massa
hebben en dus niet bijdragen tot de productie en het verval van de gluinos, top squarks en LSP's.
De razor boost analyse die in deze thesis beschreven is, is de eerste analyse binnen CMS die
expliciet de productie van top squarks via gluinos, en vervallend naar een charm quark en het LSP,
onderzoekt. Deze analyse zal dus nieuwe information verschaffen over supersymmetrie. 

De razor kinematische variabelen, $\mr$ en $\rsq$, zijn de belangrijkste variabelen voor de
razor boost analyse. Deze variabelen laten toe om een onderscheid te maken tussen de gewone
processen uit het Standaard Model, en de processen waarbij nieuwe, zware deeltjes geproduceerd
worden die vergezeld gaan van een hoeveelheid ontbrekende energie. De karakteristieke massaschaal
van de nieuwe deeltjes wordt geschat op twee manieren, door gebruik te maken van longitudinale
informatie enerzijds ($\mr$), en transversale informatie anderzijds ($\rsq$). Het gezochte signaal
zal verschijnen als een piek bovenop een exponentieel vallende achtergrond. De zoektocht naar
nieuwe fysica die hier is beschreven, wordt uitgevoerd in 25 gebieden in het tweedimensionale
($\mr$, $\rsq$) vlak. 

Dankzij het verondersteld grote massaverschil tussen de gluino en de top squark, zal de top quark
uit het verval van de gluino een grote boost krijgen. Als deze boost groot genoeg wordt, ongeveer
700\GeV, dan zullen de vervalproducten van de top quark gecollimeerd worden, en met elkaar
overlappen in de detector. Aangezien deze boost moeilijk te bereiken is met proton-proton botsingen
van 8\TeV, zullen in plaats daarvan gebooste $\W$ boson beschouwen. Door hun lagere massa,
overlappen de vervalproducten al met elkaar bij een boost van ongeveer 320\GeV. 
$\W$ bosonen die hadronisch vervallen worden ge\"identificeerd door gebruik te maken van technieken
die de substructuur van jets bestuderen, in het bijzonder `jet pruning' en `N-subjettiness'. De
eerste van die technieken laat toe om de massa van jets nauwkeurig te bepalen. De gezochte jets
moeten dus een massa hebben die overeenstemt met de bekende massa van een $\W$ boson.
De N-subjettiness observabelen testen of een jet compatibel is met het hebben van N subjets. Voor
de razor boost analyse zullen we dus vereisen dat de $\W$ boson kandidaten consistent zijn met de
hypothese van twee subjets. 


De selectie van het mogelijke signaal omvat de vereiste dat er minstens \'e\'en $\W$ boson
kandidaat is, en minstens \'e\'en jet die van een $\cPqb$ quark afkomstig is. We verwijderen alle
botsingen waarbij leptonen geproduceerd werden. De achtergrond van gekende processen uit
het Standaard Model wordt voor elk van de 25 regio's berekend via observaties in controlegebieden
en schaalfactoren die de relatie tussen het signaalgebied en de controlegebieden beschrijven. Er
werden drie controlegebieden gedefinieerd, $Q$, $W$ en $T$, die respectievelijk de multijet,
$t\bar{t}$ en $\W(\rightarrow \ell\nu)$+jets processen selecteren. 
Als gevolg van het tekort aan gesimuleerde data voor grote waarden van $\mr$ en $\rsq$, gebruiken
we globale schaalfactoren, ze zijn dus ge\"integreerd over het volledige ($\mr$, $\rsq$) vlak.


De methode om de achtergrond te berekenen is gebaseerd op een likelihood. De likelihood voor elke
signaalregio is een Poisson. De verwachtte achtergrondcomponenten in de
signaalgebieden, overeenstemmend met de multijet, $t\bar{t}$ en $\W(\rightarrow \ell\nu)$+jets
processen, worden gemodelleerd door een `prior distribution'. Deze distributie omvat de relaties
tussen de signaal- en controlegebieden, en all statistische en systematische onzekerheden. Alle
systematische onzekerheden worden gelijktijdig gevarieerd, wat tot gevolg heeft dat alle mogelijke
correlaties automatisch in rekening gebracht worden. 
De razor boost analyse is de eerste analyse binnen de supersymmetrie group van CMS die deze aanpak
gebruikt heeft. 

Het resultaat van de voorspelling van het aantal achtergrond-events in elke signaalregio werd
afgeleid van de prior, en is overeenstemming met de observatie in de data. We kunnen dus besluiten
dat er geen aanwijzingen zijn voor het bestaan van nieuwe fysica in de faseruimte die onderzocht
werd door de razor boost analyse. We kunnen deze resultaten ook gebruiken om limieten te zetten op
parameters van specifieke modellen. Zoals al eerder vermeld, beschouwen we twee modellen van gluino
paarproductie, waarbij de gluinos vervallen naar een top squark en een top quark. Voor het model
waarbij de top squark op zijn beurt vervalt naar een charm quark en het LSP, en het massaverschil
tussen de top squark en het LSP klein is ($\leq 80\GeV$), kunnen we de aanwezigheid van gluinos met
een massa tot 1\TeV uitsluiten, zolang het LSP een massa heeft die kleiner is dan 500\GeV. We kunnen
dit ook omdraaien en besluiten dat we top squarks met een massa tot ongeveer 500\GeV kunnen
uitsluiten, op voorwaarde dat ze vervallen tot een charm quark en het LSP, en dat er een gluino
aanwezig is in het spectrum met een massa van maximaal 1\TeV. 
Voor het andere model kunnen we een gelijkaardige vaststelling maken. Top squarks die vervallen
naar een top quark en het LSP, waarbij het massaverschil gelijk is aan de top quark massa, kunnen
uitgesloten worden als hun massa lager is dan ongeveer 450\GeV, en er gluinos bestaan met een
maximale massa van 850\GeV. 

Het zwakste punt van de razor boost analyse is het gebruik van globale schaalfactoren om de
relaties tussen signaalregio en controleregio's te beschrijven. Er kan namelijk een verschil zijn
in de vorm van de distributies die daardoor niet in rekening gebracht wordt. We hebben
verschillende tests uitgevoerd om te verifi\"eren dat de globale schaalfactoren voldoende geschikt
waren volgens de statistische precisie van de analyse. Als resultaat van deze testen werd er
besloten om een bijkomende onzekerheid van 33\% toe te passen op de multijet voorspelling. De
beslissing om globale schaalfactoren te gebruiken werd genomen omdat de statistische precisie van
de simulatie niet voldoende was om een aparte schaalfactor voor elke regio te defini\"eren. In de
toekomst kan dit onderdeel van de analyse dus verbeterd worden door meer gesimuleerde data te
voorzien in de regio's met hoge boost. 

De razor boost analyse gebruikte twee technieken die de analyse onderscheid van andere
supersymmetrie analyses: het identificeren van $\W$ bosonen met hoge boost, en de statistische
behandeling van de systematische onzekerheden. Deze beide technieken zullen erg waardevol zijn voor
toekomstige analyses die de dataset van 2015 zullen gebruiken. 

Door het verhogen van de botsingsenergie van 8\TeV naar 13\TeV zullen geproduceerde deeltjes meer
geboost worden. Het gebruik van jet substructuur kan dus een groot voordeel opleveren voor veel
analyses. Voor de razor boost analyse betekent de verhoogde energie dat de effici\"entie van de
selectie ook zal verhogen, meer dan wat verwacht zou worden van de verhoogde werkzame doorsnede
alleen. 

Het toevoegen van extra signaalgebieden kan ook helpen om de gevoeligheid aan meer verscheidene
modellen te verbeteren. Een eerste mogelijkheid is het invoegen van boosted top quark
identificatie. Door topologie\"en met boosted top quarks en boosted $\W$ bosonen te combineren,
kunnen meer modellen getest worden. Enerzijds zal het signaalgebied met boosted top quarks
optimaal zijn voor modellen met grote verschillen in massa tussen de verschillende deeltjes. 
Anderzijds is het gebruik van boosted $\W$ bosons gepast voor modellen met meer gemiddelde
massaverschillen. 
De tweede optie is het toevoegen van een signaalregio waarbij de aanwezigheid van een lepton
vereist wordt. $\W$ bosonen met een grote boost zullen vervallen naar leptonen met veel transversale
impuls. Deze leptonen kunnen zich bovendien dicht bij de $\cPqb$ jet van het top quark verval
bevinden, wat ervoor kan zorgen dat ze aan bepaalde isolatiecriteria niet voldoen. Een specifieke
behandeling van deze leptonen zal dus nodig zijn. 

