%%%%%%%%%%%%%%%%%%%%%
% Likelihood stuff
%%%%%%%%%%%%%%%%%%%%%

The statistical analysis of the observations,  $\{ N^S_i \}$, in the signal region is based on a
likelihood function, $L(\sigma)$, given by
\begin{align}
  L(\sigma) & \equiv  \int   \left[ \prod_{i=1}^M p(N^S_i | \sigma, {\cal L}, \theta_i)  \right] 
\pi(\theta) \, \pi({\cal L}) \, d\theta \, d{\cal L},
\label{eq:marginal}
\end{align}
where $\sigma$ is the total signal cross section, $M = 25$ is the number of bins, $N^S_i$ the
observed count in bin $i$, and the bin-by-bin parameters  $\epsilon$,  $b^S_{QCD}, b^S_{TTJ},
b^S_{\W\ell\nu}$, and $b^S_{oth}$ are  denoted collectively by $\theta$. 
The function $\pi({\cal L})$ is the integrated luminosity prior and $\pi(\theta)$ is an evidence
based prior constructed from observations in the control regions and the four global scale factors
$\kappa^{A/B}_{process}$ determined by simulated data. 
The parameter $\epsilon$ represents the $M$ signal efficiencies (including acceptance) for a given
signal model. Figure~\ref{fig:boost_flowchart} shows which control regions provide constraints on
the background parameters, $b^S_{process}$.
The likelihood per bin is taken to be
\begin{equation}
 p(N^S | \sigma, {\cal L}, \theta) = \textrm{Poisson}(N^S,  \epsilon \sigma {\cal L} + b^S_{QCD} +
b^S_{TTJ} + b^S_{\W\ell\nu} +  b^{S}_{oth}) .
\end{equation}

The integral in Eq.~(\ref{eq:marginal}) is approximated by Monte Carlo integration by sampling
 the priors $\pi({\cal L})$, and  $\pi(\theta)$. 
The priors for the expected integrated luminosity, ${\cal L}$, signal efficiencies, $\epsilon$, and 
simulated background counts, $b^{region}_{process, MC}$, are modelled with gamma densities,
\begin{align}
\textrm{gamma}(x, \gamma, \beta) &= \beta^{-1}(x/\beta)^{\gamma-1} \exp(-x / \beta) /
\Gamma(\gamma),
\label{eq:gamma}
\end{align}
in which the mode is set to $c$ and the variance to $\delta c^2$, 
where $c \pm \delta c$ denotes either the measured integrated luminosity, or for a given bin of
a given region and process, the simulated signal efficiency, or the simulated background count. This
yields the gamma density parameters,
\begin{align}
   \gamma &= [(k + 2) + \sqrt{(k+2)^2 - 4}]/2,\\
   \beta &= [\sqrt{c^2 + 4\delta c^2} - c]/2,
\end{align}
where $k = (c / \delta c)^2$.
For empty bins, we set $\gamma = 1$ and the bin value is constrained to zero by setting the $\beta$
parameter to $10^{-4}$.
 
For the signal efficiencies and backgrounds, the prior is modelled hierarchically,
\begin{align}
  \pi(\theta) = \int \pi(\theta | c ) \, \pi(c | \phi ) \pi(\phi) \, dc d\phi,
  \label{eq:prior}
\end{align}
where $c$ is a simulated count or efficiency in an $(\mr,  \rsq)$ bin and $\phi$ represents
parameters that characterize the independent sources of systematic uncertainty, described in
Section~\ref{sec:boost_systematics}. 
The integral in Eq.~(\ref{eq:prior}) is evaluated as follows: $\phi$ values are sampled from
$\pi(\phi)$, then $c$ values from $\pi(c | \phi)$, then $\theta$ values from $\pi(\theta | c)$. 
The sampling from $\pi(\phi)$ and $\pi(\theta|c)$ is straightforward because the functional forms
are known. However, the sampling of $c$ requires running the analysis multiple times.
The independent sources of systematic uncertainty are sampled simultaneously, which produces an
ensemble of sets of $(\mr, \rsq)$ histograms for the simulated backgrounds and efficiencies, for all
signals under consideration, that automatically incorporate all statistical dependencies
without the need to model them explicitly.  The ensemble of histograms is the output of the
procedure described in Section~\ref{sec:boost_systematics}. Thereafter, the sampling proceeds as
follows:
\begin{enumerate}
\item sample the integrated luminosity parameter;
\item sample the efficiency parameters, $\epsilon$, for every signal model;
\item sample the parameters $b^{region}_{process, MC}$ of the simulated background densities and sum
their values over the $M$ bins;
\item compute the $\kappa$ parameters from the appropriate background sums (for example,
$\kappa^{Q/S}_{QCD} =  \sum_i  b^Q_{QCD, MC, i} / \sum b^S_{QCD, MC, i}$);
\item scale each $\kappa$ value by a random Gaussian variate of unit mean and standard deviation
of 0.33 to account for additional uncertainty in $\kappa$ due to deficiencies in the simulated
 data, and 
\item sample the background parameters $b^S_{QCD}$, $b^S_{TTJ}$, and $b^S_{\W\ell\nu}$, from the 
Poisson models  of the control regions; for example, for region $Q$, we map  $\textrm{Poisson}(N^Q
, \kappa^{Q / S} b^S_{QCD} + b^Q_{oth})$ to a posterior density in $b^S_{QCD}$ using a flat prior
and sample $b^S_{QCD}$ from that density.
\end{enumerate}

In the absence of  a signal, we determine limits on the total signal cross section using the CLs
criterion~\cite{LHCCLs} and the test statistic $t_\sigma = 2 \ln [ L(\hat{\sigma}) /  L(\sigma)]$
when $0 \leq\hat{\sigma} \leq \sigma$, and $t_\sigma = 0$ when $\hat{\sigma} > \sigma$. 
Large values of $t_\sigma$ indicate incompatibility between the best fit hypothesis $\sigma^\prime 
= \hat{\sigma}$ and the entertained hypothesis $\sigma^\prime  = \sigma$. 
We calculate  the p-values $p_0 = \textrm{Prob}(t_\sigma > t_{\sigma, obs} | \sigma^\prime = 0)$ 
and $p_\sigma = \textrm{Prob}(t_\sigma > t_{\sigma, obs} | \sigma^\prime=\sigma)$, needed to
calculate $\textrm{CLs}(\sigma) = p_\sigma / p_0$,  by simulation. 
The quantity $t_{\sigma, obs}$ denotes the observed values of the test statistic, one for each
hypothesis $\sigma^\prime=\sigma$.
