% %%%%%%%%%%%%%%%%%%%%%%%%%%%
% % Systematic uncertainties
% %%%%%%%%%%%%%%%%%%%%%%%%%%%

 
The input to the statistical analysis is an ensemble of histograms in the $(\mr,\rsq)$ plane that 
incorporates systematic uncertainties in the simulated signal and background samples.  
The independent systematic effects, described below, are sampled simultaneously. 
This is one of the characteristics that sets this analysis apart. 
For each sampled systematic effect, the same zero mean, unit variance, Gaussian variate is used in
the calculation of the random shift of the systematic effect for all the signal and background
models. Likewise, the same randomly sampled parton distribution functions (PDFs) are used for all
signal and background models. 
In this way, the statistical dependencies among all bins of the signal and background models are
correctly, and automatically, modelled. The sampling of the systematic effects
is repeated several hundred times.  
In all cases, except for the PDFs, the systematic uncertainties are in the scale factors (SF) 
applied to the simulated samples to correct them for modelling deficiencies. 
In the next subsections I will discuss each source of systematic uncertainty in more detail. 


\subsection{Jet energy scale corrections}  

% TODO: might need to be moved partially to the reconstruction section

Jet energy corrections (JEC) are a set of tools that map the measured jet energy deposition in the
detector to the particle or parton level, as desired. 
Within CMS the JEC are applied in different sequential levels, correcting for different effects.
Each level of correction is a scaling of the four-momentum of a jet by a scale factor -- the
correction -- that depends on the jet \pt, $\eta$ and flavour.  

Uncertainties on JEC originate from various uncorrelated sources.   
Given $M$ such sources, the correction on the \pt of a jet becomes 
\begin{equation}
s(\pt, \eta, \alpha_i) = \sum_{i=1}^M \alpha_i S_i (\pt, \eta)
\end{equation}
where $S_i (\pt, \eta)$ is a \pt and $\eta$-dependent JEC uncertainty for a source $i$ and
$\alpha_i$s are weights randomly sampled from Gaussian distributions with zero mean and unit
width. 
Here, the $\alpha_i$ are different for each source, but are universal for each jet and each event.

In a much simpler approach JEC corrections can be computed as
\begin{equation}
s(\pt, \eta, \alpha) = \alpha S (\pt, \eta) = \alpha \sqrt {\sum_i^M s_i^2 (\pt, \eta) }
\end{equation}
using a total uncertainty $S (\pt, \eta)$ and a random number from a single Gaussian.  
The jet $p_T$ is corrected as:
\begin{equation}
\pt^{\rm corr} = \left(1 + s(\pt, \eta, \alpha)\right) \pt^{\rm orig}.
\end{equation}
This calculation is repeated $N$ times, each time, using a different number sampled from the
Gaussian.  
The overall effect of JEC uncertainty on the yield is obtained from the distribution of resulting
$N$ yields.  
The JEC uncertainties are also propagated to \ETm .


\subsection{Parton distribution functions} 

We use 100 randomly sampled sets of PDFs from {\tt NNPDF23\_lo\_as\_0130\_qed}~\cite{nnpdf}, {\tt
CT10}~\cite{Lai:2010vv}, and {\tt MSTW2008lo68cl}~\cite{Martin:2009iq}.  The samples for the latter
two are generated using the program {\tt hessian2replicas}, recently released with {\tt
LHAPDF6}~\cite{LHAPDF6}. Given a sampled set $i$, for PDF set $K$ and the PDF set $O$ with which the
events were simulated, events are reweighted using the scale factors, ${\rm SF}_{K, i} = w_{K, i} /
w_{O}$,
where the weights $w$ are products of the event-by-event PDFs for the colliding partons.

\subsection{Trigger efficiency}  

We take the uncertainty in each bin, as a function of $H_T$
and leading jet $\pt$, to be the maximum of the statistical uncertainty in the efficiency after
preselection and the difference between the efficiencies before and after preselection. 
% 
% The trigger efficiency as determined in section~\ref{sec:trigger} comes with an associated
% uncertainty coming from the statistical precision of the data samples used to perform the
% measurement.   Magnitudes of the plus and minus uncertainties were shown in
% figure~\ref{fig:trigger_efficiency} as functions of $H_T$ and leading jet \pt.  We compute an event
% weight based on the efficiency and uncertainty as follows:
% \begin{eqnarray}
% w_{\rm trig} & = & \epsilon_{\rm trig}(H_T, j_1p_T) + \sigma_{\rm trig} \delta\epsilon^+_{\rm
% trig}(H_T, j_1p_T), \,\,\, {\rm if} \,\, \sigma_{\rm trig} > 0 \\ 
% w_{\rm trig} & = & \epsilon_{\rm trig}(H_T, j_1p_T) + \sigma_{\rm trig} \delta\epsilon^-_{\rm
% trig}(H_T, j_1p_T), \,\,\, {\rm if} \,\, \sigma_{\rm trig} < 0 
% \end{eqnarray}
% where $\sigma_{\rm trig}$ is a random Gaussian number.	


\subsection{\texorpdfstring{$\cPqb$}{b} tagging} 
The $\cPqb$ tagging performance differs
between data and
simulation, and differs between the CMS full simulation (FullSim) and the parametric simulation
(FastSim).  The simulated events are therefore corrected by applying  jet flavour, \pt, and $\eta$
dependent data / FullSim and FullSim / FastSim $\cPqb$ tag scale factors.

\subsection{\texorpdfstring{$\W$}{W} tagging} 
The $\W$ boson tag efficiency, and the
misidentification
(or \textit{fake}) rate for $\W$ boson tag, $\W$ boson mass-tag, and $\W$ boson anti-tag differ
between data and simulation, as well as between FullSim and FastSim.  Data / FullSim and FullSim /
FastSim scale factors,  whose uncertainties are functions of jet $\pt$, are applied to the simulated
samples.  Some of these scale factors are derived specifically for this study  and are described
further in Section~\ref{sec:Wtag_SF}.

\subsection{Lepton identification} 

For electrons, we use \pt and $\eta$-dependent scale factors.
The 
uncertainties
are also \pt and $\eta$-dependent.  The corresponding uncertainties for muons are negligible.  

\subsection{Initial State Radiation} 
Deficiencies in the modelling of
initial-state-radiation (ISR) 
are corrected by reweighting~\cite{ISRerr} the signal samples using an event weight 
that depends on the \pt of the recoiling system.  The associated systematic uncertainty is equal to
the difference $1 - w_{ISR}$, where $w_{ISR}$ is the ISR event weight.  

\subsection{Top quark transverse momentum} 
Differential top-quark-pair cross section
analyses have
shown that the shape of the \pt spectrum of  top quarks in data is softer than
predicted~\cite{toppt}. To account for this, we reweight events  based on the \pt of the generator
level $t$ and $\bar{t}$ quarks in the $t\bar{t}$ simulation.  
The uncertainty associated with this reweighting is taken to be equal to the full size of the
reweighting.

\subsection{Pileup} 
Simulated events are reweighted so that their pileup (i.e.  vertex
multiplicity)
distributions  and the observed pileup distribution,  match. The minbias cross section is varied by
$\pm 5\%$, thereby changing the shape of pileup distribution and therefore the weights.  The
difference in weights is taken as a measure of the uncertainty in the pileup distribution.  

\subsection{QCD spectrum} 
The cross checks described in Section~\ref{sec:selection}
showed that
there is a 40\% uncertainty in the QCD multijet scale factor $\kappa$ between the signal and QCD
regions.  This uncertainty is accounted for by including an additional 33\% uncertainty to the
$\kappa$
parameters, as described in Section~\ref{sec:likelihood}.

\subsection{$\cPZ (\rightarrow \nu \nu)+$jets prediction} 
About 8\% of the background in
the signal
region is composed of $Z(\rightarrow\nu\nu)$+jets events. Since we require the presence of at least
one $\cPqb$ tagged jet, and given the known deficiency in modelling $\cPZ$ production in association
with heavy flavour, we include an extra systematic uncertainty in the $\cPZ(\rightarrow\nu\nu)$+jets
contribution.  This uncertainty is estimated using a data control region enriched in
$\cPZ(\rightarrow \ell \bar{\ell})+$jets, required to have exactly two tight leptons, same flavour
and opposite sign ($e$ or $\mu$), $60 < m_{ll} < 120\GeV$, at least one $\cPqb$ tagged jet, and at
least one $\W$ mass-tagged jet.  We estimate the uncertainty by first computing bin-by-bin data/MC
ratios in this control region.  Then, we take the uncertainty in the ratio in each bin as the
standard deviation of a Gaussian, normalized to  the number of events in that bin.  Finally, the
Gaussians from all bins are superposed, and the 
uncertainty is taken to be the magnitude of the 68\% band around a ratio of unity.

\subsection{Summary of separate systematic effects}

As noted before, all systematic effects are varied simultaneously. However, in order to see the
effect of each systematic uncertainty individually, each systematic effect $i$ is varied by one
standard deviation up and down.  
The effect on the background and signal samples in the signal region is shown in
Table~\ref{tab:bgsigsys}.  
The signal values are obtained from averaging over all mass points in the {\it T1ttcc} ($\Delta m =
25\GeV$) plane.  The PDF systematic uncertainties are obtained by running
over 100 different PDF set members, fitting a Gaussian to the efficiency distribution and taking the
width of the Gaussian.  
The last line in the table corresponds to the full sampling of the systematic uncertainties. To
obtain this value we again fit a Gaussian to the efficiency distribution obtained from the full
systematic sampling including 500 variations.  We note  that, although the effects of some of these
systematic uncertainties on the backgrounds are large, these do not influence our results greatly
because only the  ratios of simulated background counts enter the statistical analysis, not the
distributions themselves.  Therefore, most of the systematic effects cancel.  The dominant
systematic uncertainty arises from the parton distribution functions. The statistical precision of
the control regions is the leading uncertainty for the search bins at large $\mr$ or $\rsq$.

{
%\renewcommand{\arraystretch}{1.4}
\begin{table}[htpb]
\centering
\caption{Summary of $\pm 1 \sigma$ systematic uncertainties for the average signal count of all
 T1ttcc ($\Delta m=25\GeV$) signal points, and for the total background count in the signal
region, unless indicated otherwise, as determined from simulation.  \label{tab:bgsigsys}}
\vspace{1ex}
\begin{tabular}{l c c}
\toprule
Systematic Effect & Signal  & Background \\
\midrule
JEC & $ +2.2\% -2.1\%$   & $ +10.9\% -5.2\%$\\ 
Trigger & $ +1.1\% -3.3\%$ & $ +3.4\% -5.7\%$\\
b tag FullSim & $ +2.1\% -2.3\%$& $+3.9\% -4.0\%$\\
b tag FastSim & $ +1.2\% -1.3\%$& - \\
W tag efficiency Fullsim & $ +9.0\% -8.9\%$& $+4.6\% -4.6\%$\\
W tag efficiency FastSim & $ +2.2\% -2.2\%$& - \\
W tag fake rate FullSim & - & $ +1.4\% -1.4\%$ \\
W anti-tag fake rate FullSim ($Q$ region only) & - & $+2.6\% -2.6\%$ \\ 
W mass-tag fake rate FullSim ($W$ region only) & - & $+2.3\% -2.3\%$ \\ 
Electron ID ($T$ and $W$ region only) & - & $+0.2\% -0.2\%$ \\ 
Pileup & $ +0.5\% -0.5\%$ & $+1.0\% -1.1\%$\\
ISR & $ +6.6\% -6.6\%$ & - \\
Top \pt spectrum & - & $ -14.4\% ~ 20.5\%$ \\
$\cPZ\rightarrow\nu\nu+$ heavy flavour  & - & $+4.0\% -4.0\%$ \\
PDF & $20.7\%$ &  $10.7\%$ \\
\midrule
All &  $24.4\%$ &  $22.1\%$ \\
\bottomrule
\end{tabular}
\end{table}
}
 


% %%%%%%%%%%%%%%%%%%%%%%%%%%%%%%%%%%%%%%%%%%%%%%%%%%%%%%%%%%%%%%%%%%%%%%%%%%%%%%%%%%%%%%%%%%%%%%%%
% 
% \subsection{Jet energy corrections (JEC) \label{sec:JEC}}
% 

% 
% %%%%%%%%%%%%%%%%%%%%%%%%%%%%%%%%%%%%%%%%%%%%%%%%%%%%%%%%%%%%%%%%%%%%%%%%%%%%%%%%%%%%%%
% 
% \subsection{Parton distribution functions (PDF) \label{sec:PDFweights}}
% 
% In order to evaluate the systematics on parton distribution functions (PDFs), we use the three
% recommended PDF sets {\tt CT10}, {\tt MSTW2008lo68cl} and {\tt NNPDF23\_lo\_as\_0130\_qed} by
% PDF4LHC.  
% Since we take into account full correlations wihtin systematic variations, we need a way to sample
% randomly from the PDF uncertainties.  Of the three recommended PDF sets, NNPDF presents the PDF
% eigenvectors as a randomly distributed set, while the other groups provide eigenvectors obtained by
% varying the PDF fit parameters by $\pm 1$ standard deviation.  
% However, the recently developed {\tt LHAPDF6} offers a formal way to convert the latter sets into
% randomly distributed sets.  
% We have used the {\tt examples/hessian2replicas} code provided in {\tt LHAPDF6} to generate randomly
% distributed PDF sets with 100 members each for {\tt CT10} and {\tt MSTW2008lo68cl}.  
% 
% Given a member $i$ of a PDF set $K$, we calculate a scale factor for each event as the ratio of the
% PDF weight computed using the member $i$ and the weight computed using the central value of the
% original PDF used for generating the event
% \begin{equation}
% {\rm SF}^{\rm PDF}_{K}(i) = w^{\rm PDF}_{K}(i) / w^{\rm PDF}_{\rm orig}(\rm central)
% \end{equation} 
% All SM samples except for those generated with {\tt POWHEG} and all SMS signal samples were
% generated using {\tt CTEQ6.1}, while all {\tt POWHEG} samples were generated using {\tt CT10}.
% 
% The overal PDF uncertainty is then obtained from the distribution of ${\rm SF}^{\rm PDF}_{K}(i)$s
% obtained by a random selection of $K$s and $i$s.  
% 
% To study the effect of the different PDF sets on the MC counts used in the background estimation, we
% compute the MC counts using a random selection of PDF members from the three PDF sets. A Gaussian
% function is then fitted to these distributions. 
% Figures~\ref{fig:PDF_effect_on_bg_QCD}-\ref{fig:PDF_effect_on_bg_oth2} show the result of this
% procedure for the various considered MC counts. 
% There is a Gaussian distribution for each PDF set, and the arrow indicates the value obtained using
% the PDF that was used during the generation of the MC samples. From this we can see that the overall
% systematic uncertainty from parton distribution functions on the background counts is due to both
% the difference in the nominal values of the PDF sets, and the spread within the separate PDF sets. 
% Figure~\ref{fig:PDF_effect_on_sig} shows the equivalent for an example signal point. Here we see
% that the {\tt CT10} spread dominates the total uncertainty. 
% 
% \begin{figure}[p]
% \centering
% \includegraphics[width=0.49\textwidth]{figures/Systematics/PDF/h_S_QCD_MC}
% \includegraphics[width=0.49\textwidth]{figures/Systematics/PDF/h_Q_QCD_MC}
% \caption{Influence of different PDF sets on the MC counts entering $\kappa_{QCD}^{Q/S}$
% \label{fig:PDF_effect_on_bg_QCD}}
% \end{figure}
% 
% \begin{figure}[p]
% \centering
% \includegraphics[width=0.49\textwidth]{figures/Systematics/PDF/h_S_TTJ_MC}
% \includegraphics[width=0.49\textwidth]{figures/Systematics/PDF/h_T_TTJ_MC}
% \caption{Influence of different PDF sets on the MC counts entering $\kappa_{TTJ}^{T/S}$
% \label{fig:PDF_effect_on_bg_TTJ}}
% \end{figure}
% 
% \begin{figure}[p]
% \centering
% \includegraphics[width=0.49\textwidth]{figures/Systematics/PDF/h_S_Wlv_MC}
% \includegraphics[width=0.49\textwidth]{figures/Systematics/PDF/h_W_Wlv_MC}
% \caption{Influence of different PDF sets on the MC counts entering $\kappa_{Wlv}^{W/S}$
% \label{fig:PDF_effect_on_bg_Wlv}}
% \end{figure}
% 
% 
% \begin{figure}[p]
% \centering
% \includegraphics[width=0.49\textwidth]{figures/Systematics/PDF/h_S_oth_MC}
% \includegraphics[width=0.49\textwidth]{figures/Systematics/PDF/h_T_oth_MC}
% \caption{Influence of different PDF sets on the MC counts in the $S$ (left) and $T$ (right) region
% for the backgrounds that are taken directly from the simulation.
% \label{fig:PDF_effect_on_bg_oth1}}
% \end{figure}
% 
% \begin{figure}[p]
% \centering
% \includegraphics[width=0.49\textwidth]{figures/Systematics/PDF/h_W_oth_MC}
% \includegraphics[width=0.49\textwidth]{figures/Systematics/PDF/h_Q_oth_MC}
% \caption{Influence of different PDF sets on the MC counts in the $W$ (left) and $Q$ (right) region
% for the backgrounds that are taken directly from the simulation.
% \label{fig:PDF_effect_on_bg_oth2}}
% \end{figure}
% 
% \begin{figure}[p]
% \centering
% \includegraphics[width=0.7\textwidth]{figures/Systematics/PDF/h_S_T1ttcc_DM25_1000_300}
% \caption{Influence of different PDF sets on the signal efficiency for the T1ttcc signal point with
% $m_{\tilde{g}}=1000\GeV, m_{\tilde{t}_1}=325\GeV, m_{\tilde{\chi}_1^0}=300\GeV$.
% \label{fig:PDF_effect_on_sig}}
% \end{figure}
% 
% 
% %%%%%%%%%%%%%%%%%%%%%%%%%%%%%%%%%%%%%%%%%%%%%%%%%%%%%%%%%%%%%%%%%%%%%%%%%%%%%%%%%%%%%%
% 
% 
% \subsection{Trigger efficiency \label{sec:trigger_uncertainties}}
% 

% 
% 
% %%%%%%%%%%%%%%%%%%%%%%%%%%%%%%%%%%%%%%%%%%%%%%%%%%%%%%%%%%%%%%%%%%%%%%%%%%%%%%%%%%%%%%
% 
% 
% \subsection[btag scale factors]{$b$-tag scale factors \label{sec:btag_uncertainties}}
% 
% Performance in $b$-tagging differs between data and simulation, and differs further between FullSim
% and FastSim.  
% To account for the differences and correct simulated events to match data, we apply data-FullSim and
% FullSim-FastSim $b$-tag scale factors (SFs), which are given as the ratios of $b$-tagging
% efficiencies, where both efficiencies and SFs are functions of jet flavor, \pt, and $\eta$.  
% 
% There are several ways to reweight events using SFs \cite{BTagSF1}.  
% Here we choose a method where we consider the number of $b$ tagged jets in an event to be fixed to
% what is given by simulation.  
% For a given event, using the $b$-tagging efficiencies for truth $b$, $c$ and $udsg$ jets with the
% relevant tagging algorithm and working point, we first compute two probabilities to have our event
% give the number of $b$ jets it has as follows: 
% we compute a $P({\rm Sim})$ using efficiencies obtained from simulated MC events, and then we obtain
% a $P(\rm data)$ using efficiencies that represent data as
% \begin{eqnarray}
% P(Sim) & = & \prod_{i={\rm tagged}} \epsilon_i^{Sim} \prod_{j={\rm not tagged}} (1 -
% \epsilon_j^{Sim}) \\
% P(Data) & = & \prod_{i={\rm tagged}} \epsilon_i^{data} \prod_{j={\rm not tagged}} (1 -
% \epsilon_j^{data})
% \end{eqnarray}
% where 
% \begin{equation}
% \epsilon^{data} = {\rm SF}^{data}_{Sim} \epsilon^{Sim} .
% \end{equation}
% Our SM MC is simulated with FullSim, so we scale the efficiency with ${\rm SF}^{data}_{Sim} = {\rm
% SF}^{data}_{Full}$.  
% However, our signal samples have been generated using FastSim, so the overall scale factor becomes
% ${\rm SF}^{data}_{Sim} = {\rm SF}^{data}_{Full} {\rm SF}^{Full}_{Fast}$.
% The ratio of the two probabilities, $w = P(Data) / P(Sim)$, gives us the weight to scale the event.
%  
% The input efficiencies $\epsilon_j^{\rm Sim}$ are previously obtained from MC events.  
% 
% We take into account the uncertainties on the SFs as systematics.  
% To do this, we sample $n$ numbers $\sigma$ from a Gaussian with mean 0 and sigma 1 (we sample
% different $\sigma$s for FullSim and FastSim, and also for different jet truth flavors).  
% We then run our analysis $n$ times, where each time we use given $\sigma$s to compute
% $\epsilon^{data}$ as follows:
% \begin{eqnarray}
% \epsilon^{data} & = & ({\rm SF}^{data}_{Full} + \sigma_1 \, \delta {\rm SF}^{data}_{Full} )
% \epsilon^{Full} \\
% \epsilon^{data} & = & ({\rm SF}^{data}_{Full} + \sigma_1 \, \delta {\rm SF}^{data}_{Full} )({\rm
% SF}^{Full}_{Fast} + \sigma_2 \, \delta {\rm SF}^{Full}_{Fast} ) \epsilon^{Fast}.
% \end{eqnarray}
% For each of the $n$ runs, different $\epsilon^{data}$s will lead to different weights $w$ for a
% given event, and overall, the weighted yields will differ.  We obtain the systematic from the
% ensemble of the resulting $n$ yields.
% 
% In this study, we use CSVM and CSVL taggers, which are always used independent of each other.  
% Therefore, for a given event, we compute separate weights $w$ with CSVM and CSVL, and implement them
% in the relevant regions (e.g., we use the CSVM weight in the signal region and CSVL weight in the
% QCD region).
% 
% 
% %%%%%%%%%%%%%%%%%%%%%%%%%%%%%%%%%%%%%%%%%%%%%%%%%%%%%%%%%%%%%%%%%%%%%%%%%%%%%%%%%%%%%%
% 
% 
% \subsection[W tag scale factors]{$\PW$ tag scale factors \label{sec:Wtag_SF}}
% 
% 
% 
% %%%%%%%%%%%%%%%%%%%%%%%%%%%%%%%%%%%%%%%%%%%%%%%%%%%%%%%%%%%%%%%%%%%%%%%%%%%%%%%%%%%%%%
% 
% 
% \subsection{Lepton ID}
% 
% Our analysis requires a single loose lepton in the definitions of top and $W$ control regions.  We
% therefore apply lepton scale factors to our events and incorporate the lepton scale factor
% uncertainty as a systematic.  For electrons, we use the $p_T$ and $\eta$-dependent scale factors and
% uncertainties derived by the Egamma POG, which could be found in:
% {\small
% \begin{verbatim}
% https://twiki.cern.ch/twiki/bin/view/Main
%       /EGammaScaleFactors2012#2012_8_TeV_Jan22_Re_recoed_data
% \end{verbatim}}
% 
% The overall event weight is given as:
% \begin{equation}
% w_{eL} = {\rm SF_{eL}}(p_T, \eta) + \sigma_{\rm eL} \delta {\rm SF_{eL}}(p_T, \eta)
% \end{equation}
% where $\sigma_{\rm eL}$ is a random Gaussian number.
% 
% We ignore the scale factors and uncertainties for muons, since the scale factors are approximately 1
% and the uncertainties are negligible.
% 
% 
% %%%%%%%%%%%%%%%%%%%%%%%%%%%%%%%%%%%%%%%%%%%%%%%%%%%%%%%%%%%%%%%%%%%%%%%%%%%%%%%%%%%%%%
% 
% 
% \subsection{ISR reweighting}
% 
% For the signal samples the recommendation is to apply the so-called ISR reweighting recipe. The
% details were explained in section~\ref{sec:ISRreweighting}. 
% The size of the associated systematic uncertainty depends on the \pt of the recoiling system, and is
% equal to the difference $1 - w_{ISR}$, where $w_{ISR}$ is the weight associated to the ISR
% reweighting. 
% The one sigma up variation is thus always consistent with no reweighting. 
% 
% 
% %%%%%%%%%%%%%%%%%%%%%%%%%%%%%%%%%%%%%%%%%%%%%%%%%%%%%%%%%%%%%%%%%%%%%%%%%%%%%%%%%%%%%%
% 
% 
% \subsection{Top pT reweighting}
% 
% For more details on the top \pt reweighting recipe, we refer to
% section~\ref{sec:toppt_reweighting}. 
% The uncertainties associated with this reweighting are defined as follows:
% the one sigma down variation corresponds to no reweighting at all;
% the one sigma up variation corresponds to applying the reweighting twice. 
% 
% These uncertainties are propagated to the Data/MC ratio used in the estimation of the TTjets
% background. 
% 
% 
% %%%%%%%%%%%%%%%%%%%%%%%%%%%%%%%%%%%%%%%%%%%%%%%%%%%%%%%%%%%%%%%%%%%%%%%%%%%%%%%%%%%%%%
% 
% 
% \subsection{PU reweighting}
% 
% In order to account for uncertainties related to the PU reweighting, it is recommended to derive the
% up/down weights by varying the minbias cross section by +-5\%. This will result in a different shape
% for the data pileup, and thus in a different set of weights. 
% The difference in weights is used as the sigma of a gaussian from which we sample when doing the
% simultaneous sampling of all systematic uncertainties. 
% 
% %%%%%%%%%%%%%%%%%%%%%%%%%%%%%%%%%%%%%%%%%%%%%%%%%%%%%%%%%%%%%%%%%%%%%%%%%%%%%%%%%%%%%%
% 
% 
% \subsection{Shape systematics affecting the MC translation factors \label{sec:shape_systematics}}
% 
% In order to assess additional systematics related to the MC translation factors, we can use the two
% closure tests we explained in section~\ref{sec:Shape_sideband} and \ref{sec:Shape_QCD}. The level of
% closure in these tests can be used to derive an additional systematic uncertainty if it turns out
% that the closure is not covered by the sizeable statistical uncertainty and the other systematic
% uncertainties already considered.
% 
% To make it easier to use the results of the closure tests, we provide the results of the closure
% tests in a different format. Figures~\ref{fig:closure_green} and \ref{fig:closure_magenta} show each
% bin (containing data/prediction) in the 2D $M_R - R^2$ plane plotted next to each other, starting
% with the first $R^2$ strip for fixed $M_R$ and then moving along to the next strip for the next
% $M_R$ value. 
% 
% \begin{figure}[htpb]
% \centering
% \includegraphics[width=0.7\textwidth]{figures/ShapeSyst/closure_summary_0p5_green}
% \caption{Data/Prediction for each bin in the 2D $M_R-R^2$ plane for the closure test predicting the
% $\Delta\phi_{min}$ sideband of the S region. Uncertainties are statistical only.
% \label{fig:closure_green}}
% \end{figure}
% 
% \begin{figure}[htpb]
% \centering
% \includegraphics[width=0.7\textwidth]{figures/ShapeSyst/closure_summary_0p5_magenta}
% \caption{Data/Prediction for each bin in the 2D $M_R-R^2$ plane for the closure test predicting the
% high $\Delta\phi_{min}$ region of the Q control region. Uncertainties are statistical only.
% \label{fig:closure_magenta}}
% \end{figure}
% 
% Using the ratios between observed data and the prediction from the closure tests, we can derive a
% systematic uncertainty that we can apply on the MC ratios we use in the actual background
% estimation. 
% To do this we model each bin in the 2D plane as a Gaussian, taking the uncertainty in that bin as
% the width of the Gaussian, weight them with $\sqrt{N_{obs}}$ to give more weight to bins with higher
% precision and add all of them up. 
% From the summed distribution we can then compute the interval around 1 that contains approximately
% 68\% of the integral. 
% 
% The results of this procedure are shown in Figure~\ref{fig:closure_green_gauss} for the first
% closure test, and in Figure~\ref{fig:closure_magenta_gauss} for the second closure test. 
% 
% The 68\% range for the first closure test corresponds to a $\pm 20\%$ range, while for the second
% closure test it corresponds to $\pm 40\%$.
% These systematics are summarized in table~\ref{tab:shape_syst} for each considered MC ratio,
% $\kappa$.
% 
% As the total uncertainty on $\kappa_{QCD}^{Q/S}$ without including this closure test, is of the
% order of 25\%, we will inflate this uncertainty to 40\%, to cover for the shape difference observed
% in closure test two. 
% The observed uncertainty on the other $\kappa$'s is already sufficient. 
% 
% \begin{table}[htpb]
% \renewcommand*{\arraystretch}{1.4}
% \centering
% \caption{Summary of shape systematics \label{tab:shape_syst}}
% \begin{tabular}{|c|c|c|}
% \hline
% Translation factor & Closure test 1 & Closure test 2 \\
% \hline
% $\kappa_{TTJ}^{T/S}$ & 20\% & - \\
% $\kappa_{WJ}^{W/S}$  & 20\% & -  \\
% $\kappa_{QCD}^{Q/S}$ & 20\% & 40\%  \\
% $\kappa_{QCD}^{T/Q}$ & 20\% & 40\%  \\
% \hline
% \end{tabular}
% \end{table}
% 
% \begin{figure}[htpb]
% \centering
% \includegraphics[width=0.49\textwidth]{figures/ShapeSyst/closure_summary_0p5_green_gauss}
% \includegraphics[width=0.49\textwidth]{figures/ShapeSyst/closure_summary_0p5_green_gauss2}
% \caption{[left] We represent the agreement between data and prediction for the closure test
% predicting the $\Delta\phi_{min}$ sideband of the S region as a Gaussian pdf for each bin in the 2D
% $M_R-R^2$ plane. Each bin is shown as a normalized gaussian in a different shade of magenta. The sum
% of all Gaussians is depicted in black. 
% [right] Same as on left plot, but each separate component is now normalized to the weight it carries
% in the sum. 
% \label{fig:closure_green_gauss}}
% \end{figure}
% 
% \begin{figure}[htpb]
% \centering
% \includegraphics[width=0.49\textwidth]{figures/ShapeSyst/closure_summary_0p5_magenta_gauss}
% \includegraphics[width=0.49\textwidth]{figures/ShapeSyst/closure_summary_0p5_magenta_gauss2}
% \caption{[left] We represent the agreement between data and prediction for the closure test
% predicting the high $\Delta\phi_{min}$ region of the Q control region as a Gaussian pdf for each bin
% in the 2D $M_R-R^2$ plane. Each bin is shown as a normalized gaussian in a different shade of
% magenta. The sum of all Gaussians is depicted in black. 
% [right] Same as on left plot, but each separate component is now normalized to the weight it carries
% in the sum. 
% \label{fig:closure_magenta_gauss}}
% \end{figure}
% 
% 
% %%%%%%%%%%%%%%%%%%%%%%%%%%%%%%%%%%%%%%%%%%%%%%%%%%%%%%%%%%%%%%%%%%%%%%%%%%%%%%%%%%%%%%
% 
% 
% \subsection[Uncertainty on Znunu modeling]{Uncertainty on $Z\rightarrow\nu\nu$ modeling}
% 
% About 7-8\% of the background in our signal region is composed of $Z\rightarrow\nu\nu$+jets. As we
% are requiring the presence of at least one $b$-tagged jet, and we know that $Z$ production in
% association with heavy flavour is not well modeled, we need to apply an extra systematic uncertainty
% on the $Z\rightarrow\nu\nu$ +jets contribution. 
% 
% To derive this uncertainty, we define a control region in data, enriched in $Z\rightarrow l
% \bar{l}$, and as close as possible to the signal region. 
% The selection applied to this control region is: 
% \begin{itemize}
% \item at least one good vertex
% \item at least three jets
% \item $\pt(\textrm{jet}_1) > 200\GeV$
% \item $M_R > 800\GeV$ and $R^2 > 0.08$
% \item exactly two tight leptons, same flavour and opposite sign ($e$ or $\mu$)
% \item $60 < m_{ll} < 120$
% \item at least one $b$-tagged jet
% \item at least one $Y$ (mass-tagged jet)
% \end{itemize}  
% 
% Figure~\ref{fig:DataMC_ZCR} shows the Data/MC comparison for $M_R$ and $R^2$. 
% To derive an uncertainty using this comparison, we use the same procedure as explained in
% section~\ref{sec:shape_systematics}. 
% The results are shown in Figure~\ref{fig:1D_ZCR} and \ref{fig:ZCR_syst}. 
% Based on these results, we decide to put an additional uncertainty on the $Z\rightarrow\nu\nu$ cross
% section of 50\%. 
% 
% \begin{figure}[htpb]
% \centering
% \includegraphics[width=0.49\textwidth]{figures/DataMC/DataMC_MR_g1Mbg1Y2l0ol_width}
% \includegraphics[width=0.49\textwidth]{figures/DataMC/DataMC_R2_g1Mbg1Y2l0ol_width}
% \caption{Data/MC comparison for $M_R$ (left) and $R^2$ (right) in the $Z\rightarrow l \bar{l}$
% control region with at least one $b$-jet and at least one mass-tagged $W$-candidate. 
% \label{fig:DataMC_ZCR}}
% \end{figure}
% 
% \begin{figure}[htpb]
% \centering
% \includegraphics[width=0.7\textwidth]{figures/Zinv_DataMC_1D}
% \caption{Data/Simulation for each bin in the 2D $M_R-R^2$ plane for the $Z\rightarrow l \bar{l}$
% control region. Uncertainties are statistical only.
% \label{fig:1D_ZCR}}
% \end{figure}
% 
% \begin{figure}[htpb]
% \centering
% \includegraphics[width=0.49\textwidth]{figures/Zinv}
% \includegraphics[width=0.49\textwidth]{figures/Zinv2}
% \caption{[left] We represent the agreement between data and simulation for the $Z\rightarrow l
% \bar{l}$ control region as a Gaussian pdf for each bin in the 2D $M_R-R^2$ plane. Each bin is shown
% as a normalized gaussian in a different shade of magenta. The sum of all Gaussians is depicted in
% black. 
% [right] Same as on left plot, but each separate component is now normalized to the weight it carries
% in the sum. 
% \label{fig:ZCR_syst}}
% \end{figure}
% 
% 
% %%%%%%%%%%%%%%%%%%%%%%%%%%%%%%%%%%%%%%%%%%%%%%%%%%%%%%%%%%%%%%%%%%%%%%%%%%%%%%%%%%%%%%
% 
% 
% \subsection{Signal systematic uncertainties results \label{sec:signal_systematics}}
% 
% As explained before, we vary all systematics simultaneously, thus sampling the systematic parameter
% space in a coherent way, taking care of all correlations. 
% To get a sense of the magnitude of each individual systematic, we also varied them $1\sigma$ up and
% down. The results of this are reported in table~\ref{tab:signal_systematics}. We give the average
% value over all mass points in the T1ttcc DM25 plane, as well as the minimum and maximum variation
% found. 
% For the PDF systematics, we ran over 100 different PDF set members (as explained in
% section~\ref{sec:PDFweights}) and fitted a gaussian to the signal efficiency distribution. The width
% of that Gaussian is taken as the size of the systematic effect. 
% The last line in the table corresponds to the full systematics sampling. To obtain this value we
% again fit a gaussian to the signal efficiency distrubution obtained from the systematic sampling. In
% Appendix~\ref{app:signal_systematics} we show the effect of each systematic on the 2D SMS plane. 
% 
% As is visible from the above mentioned plots and table, the main source of systematic uncertainty
% for the signal is the uncertainty on the parton distribution functions. The uncertainty is 15-25\%
% depending on the mass point. 
% The second leading source is the uncertainty on the W tagging scale factor for FullSim, which is
% about 9\%. 
% Finally, for very compressed mass points, the uncertainty on the ISR modeling reached about 20\%,
% compared to 4-7\% for non-compressed spectra. 
% 
% {
%\renewcommand{\arraystretch}{1.4}
\begin{table}[htpb]
\centering
\caption{Summary of $\pm 1 \sigma$ systematic uncertainties for the average signal count of all
 T1ttcc ($\Delta m=25\GeV$) signal points, and for the total background count in the signal
region, unless indicated otherwise, as determined from simulation.  \label{tab:bgsigsys}}
\vspace{1ex}
\begin{tabular}{l c c}
\toprule
Systematic Effect & Signal  & Background \\
\midrule
JEC & $ +2.2\% -2.1\%$   & $ +10.9\% -5.2\%$\\ 
Trigger & $ +1.1\% -3.3\%$ & $ +3.4\% -5.7\%$\\
b tag FullSim & $ +2.1\% -2.3\%$& $+3.9\% -4.0\%$\\
b tag FastSim & $ +1.2\% -1.3\%$& - \\
W tag efficiency Fullsim & $ +9.0\% -8.9\%$& $+4.6\% -4.6\%$\\
W tag efficiency FastSim & $ +2.2\% -2.2\%$& - \\
W tag fake rate FullSim & - & $ +1.4\% -1.4\%$ \\
W anti-tag fake rate FullSim ($Q$ region only) & - & $+2.6\% -2.6\%$ \\ 
W mass-tag fake rate FullSim ($W$ region only) & - & $+2.3\% -2.3\%$ \\ 
Electron ID ($T$ and $W$ region only) & - & $+0.2\% -0.2\%$ \\ 
Pileup & $ +0.5\% -0.5\%$ & $+1.0\% -1.1\%$\\
ISR & $ +6.6\% -6.6\%$ & - \\
Top \pt spectrum & - & $ -14.4\% ~ 20.5\%$ \\
$\cPZ\rightarrow\nu\nu+$ heavy flavour  & - & $+4.0\% -4.0\%$ \\
PDF & $20.7\%$ &  $10.7\%$ \\
\midrule
All &  $24.4\%$ &  $22.1\%$ \\
\bottomrule
\end{tabular}
\end{table}
}
 
% 
% 
% %%%%%%%%%%%%%%%%%%%%%%%%%%%%%%%%%%%%%%%%%%%%%%%%%%%%%%%%%%%%%%%%%%%%%%%%%%%%%%%%%%%%%%
% 
% 
% \subsection{Background systematic uncertainties results \label{sec:background_systematics}}
% 
% To get a sense of the background systematic effects, we provide the $\pm 1\sigma$ variations for the
% separate systematic uncertainties for several MC event counts. 
% In table~\ref{tab:bg_systematics_S_tot} we show the systematic uncertainties on the total event
% count in the signal region. 
% This count is not used in the background estimation, but gives a good sense of the amplitude of the
% systematic effects. 
% 
% The main effect on the simulated background counts comes from the uncertainty on the top \pt
% spectrum, followed by the PDF uncertainties, the uncertainties related to the jet energy scale and
% the uncertainty on the $\cPW$ tagging efficiency. 
% Most of these effects cancel, however, since we use these counts only in ratios (the $\kappa$
% factors). 
% For the highest bins in the ($M_R,R^2$) plane, the background prediction is limited by the
% statistical precision in the control regions. 
% 
% \input{single_sys_summary_bg_h_S_tot_MC.tex}
% 
