\chapter{Introduction \label{chap:introduction}}

\textit{Curiosity.} Among all human traits, curiosity is the one that really drives progress and
innovation. Every child is born with it. It expresses itself as the relentless desire to learn about
the world, be it by eating dirt because you wondered what it tastes like, or by going through your
parents' cupboards even though that was not allowed. I am sure you can think of many more of these
examples. 
To me, this fundamental human trait is what science is all about. 

For centuries, scientists have investigated the world around them, pushing the boundaries of
our knowledge and coming to an ever better understanding of nature.
This exploration lead us to the biggest structures in the universe, the stars and galaxies, but
also to the smallest. First, atoms were discovered, thought at the time to be indivisible.
When Rutherford and Thomson showed that these atoms actually consist of a positive nucleus
surrounded by negative charges, that was a paradigm-shifting revelation. 

From then onward, theory and experiment worked hand-in-hand to come to the Standard Model of
particle physics. Many new particles were discovered during the past one hundred years or so. 
The electron, proton and neutron were the first to be discovered. Then a range of particles,
among which are the positron, pion and kaon, were identified in the study of cosmic rays. 
With the advent of particle accelerators, physicists were able to create and identify many more
particles. There were so many new particles in fact, without any apparent structure to them, that
they were called the `particle zoo'. Only when Gell-Mann and Zweig proposed that these particles
were composed of quarks, did pieces start falling into place. Experimental evidence of quarks was
found not much later. 

During the 1960s and 1970s, the theoretical formulation of the Standard Model took shape. 
The Standard Model describes the elementary particles and their interactions. In those days,
physicists already knew of all particles belonging to the first two families: the electron and
electron neutrino, the muon and muon neutrino, the up and down quarks, and the charm and strange
quarks. The tau lepton and bottom quark were discovered in the second half of the 1970s, indicating
the existence of a third family as well. If this family was to fit in the existing structure, two
more particles needed to exist: the tau neutrino, and the top quark. 
The Standard Model also predicted the existence of force carriers, in particular the gluon, and
the $\W$, $\cPZ$ and Higgs bosons. 
If the Standard Model was really to describe nature, then these particles needed to exist. And so
the hunt for these new particles began.

The gluon was discovered indirectly in 1979 at DESY, and the $\W$ and $\cPZ$ bosons were observed
in 1983 at CERN. The top quark was more difficult to find, it took until 1995 at Fermilab. The tau
neutrino was observed directly for the first time in 2000, also at Fermilab. The only missing part
was the elusive Higgs boson. In 2012, more than 40 years after it was first proposed, the Higgs
boson was finally discovered at the CERN LHC.

The Standard Model has proven to be very accurate in its predictions, both in terms of particle
content, and in behaviour of a multitude of processes. Until today, no significant deviations from
the Standard Model predictions have been found. However, there are several indications that the
current Standard Model is not the final word. Gravity is not included in the Standard Model for
example, nor can it provide a dark matter candidate. There is also no good theoretical explanation
for the mass of the Higgs boson. 

In an attempt to address these unresolved questions, many models of \textit{new physics} have been
proposed. One of the most popular models is supersymmetry (SUSY), which predicts the existence of
even more new particles, namely a superpartner for every Standard Model particle. 
Many searches for signs of new physics have been performed already, first at LEP and Tevatron, and
now at the LHC. Unfortunately, none of these searches have found any evidence.
Nevertheless, the general feeling within the experimental community is still one of optimism. Run 2
of the LHC is just around the corner, opening up a thus far unexplored energy domain. No matter what
will be found, our knowledge of the world around us will grow once more. 

Over the past couple years I participated in two searches for new physics with the CMS experiment at
the LHC. As a starting PhD student, I contributed to a search for the supersymmetric partner of the
top quark, first using the dataset at 7\TeV centre-of-mass energy, and later also at 8\TeV. 
Since it had become clear at that point that the more standard searches had not found any evidence
for new physics, the focus within the search groups shifted towards more dedicated searches to
explore all possible gaps in our sensitivity. With the discovery of the Higgs boson,
so-called natural supersymmetry also became a hot topic. 
The focus of these models is to provide a satisfying answer to question why the Higgs boson mass is
relatively light, also known as the hierarchy problem. 

In this thesis I present the \textit{razor boost} analysis, which was my main research topic for the
past two years. It especially targets natural SUSY models that were not well covered in previous
analyses, and therefore fits perfectly within the scope of the late Run 1 analyses. The razor boost
analysis uses the razor kinematic variables to search for signs of new physics in hadronic final
states including a highly boosted $\W$ boson. 

The thesis is structured in the following way: the first two chapters will briefly cover the
Standard Model and the need for new physics beyond the Standard Model. An introduction to
supersymmetry will be given in Chapter~\ref{chap:supersymmetry}, with emphasis on natural
supersymmetry and the phenomenological implications. Chapters~\ref{chap:LHC} and \ref{chap:CMS} will
provide some details on the LHC and the CMS experiment. The event generation, simulation, and
reconstruction will be discussed in Chapter~\ref{chap:event_generation}. Full details on the razor
boost analysis are presented in Chapter~\ref{chap:razorboost}, before concluding with a summary and
outlook in Chapter~\ref{chap:summary}.



