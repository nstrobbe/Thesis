%%%%%%%%%%%%%%%%%%%%%%%%%%%%
%% Event simulation
%%%%%%%%%%%%%%%%%%%%%%%%%%%%

Simulation of the CMS detector response is done in one of two ways: a `full' simulation
(FullSim) which is time-consuming but very accurate, or a `fast' simulation (FastSim), which is much
faster, but for which approximations have been made. The following subsections will explain the
basic concepts and use cases for both of these options. 

\subsection{CMS Full Simulation using Geant4 \label{subsec:fullsim}}

The purpose of the CMS full simulation~\cite{Banerjee:2007zz,Banerjee:2011zzc,Banerjee:2012ge} is to
provide a very accurate description of how particles interact with the CMS detector. These simulated
event samples can then be used to understand and demonstrate the power of analysis methods which
will later be applied to the real data. They can also be used to derive calibrations, efficiencies
and resolutions for high level physics objects in case the available data are not sufficient

The inputs to the detector simulation are the hadronized particles from the event generator. These
particles and their four-vectors are passed to the simulation software in the \textsc{HepMC}
format~\cite{Dobbs:2001ck}. 
The particles are then propagated through the CMS detector using the \GEANTfour toolkit~\cite{G4}.
\GEANTfour contains a large collection of electromagnetic and hadronic physics processes describing
the interaction of particles with material, and the resulting energy loss. Examples of this are
bremsstrahlung, photon conversion, nuclear interactions, multiple scattering, and showering. In
case new particles are produced in these interactions, they are also propagated further. 

A key component for the success of the simulation is the precise implementation of the detector
geometry, material budget, and magnetic field strength. Not only the active layers need to be
accounted for, but also the cooling, cabling, and support structures. 
Choices must be made for the level of detail to include in the simulation geometry in order to
optimize computation speed versus the correctness of the simulation.
Within the CMS software framework there is a single, common implementation of the full detector
geometry for both the simulation and the reconstruction, ensuring full compatibility.

The accuracy of the detector implementation has been thoroughly tested with dedicated test beam
data, cosmic muon data, and early LHC collision data. Overall, very good agreement was
found, and adjustments were made where necessary. 

Pileup interactions are also added at this stage of the processing chain. A library of simulated
hits of minimum bias events is prepared beforehand, and then used to overlay a number of extra
interactions onto the signal event according to a specified pileup scenario. 
As the bunch crossing time is shorter than the time needed for particles of a given event to fully
traverse the CMS detector, we also need to take into account \textit{out-of-time pileup}, which is
the effect of previous or subsequent bunch-crossings on the current event. Out-of-time pileup is
modelled by modifying the timing of the detector hits when overlaying a minimum bias interaction. 

The next step is the conversion of all energy depositions, from signal and pileup events, in the
sensitive detector volumes to electronic signals. Electronic noise is also included during this
\textit{digitization} step.
The output of the digitization is simulated data in a format identical to that of real collision
data read directly from the detector. The L1 and HLT decisions, and the objects used to arrive at
them, are also included in the simulated data. 
From this point onwards the simulated events go through the same reconstruction steps as the real
data, as will be explained in Section~\ref{sec:event_reconstruction}. 





\subsection{CMS Fast Simulation \label{subsec:fastsim}}

% explain what approximations are made, why it is necessary, especially in SUSY searches


The CMS fast simulation~\cite{fastsim,Rahmat:2012fs} is a faster alternative to the full simulation
explained above. 
It is intended to be used for physics analyses that require the generation of many
samples to span a wide phase-space region, e.g. the vast SUSY simplified model scans. 
A set of approximations is made, resulting in a speed
increase of about a factor twenty. 

The interactions simulated in FastSim are electron bremsstrahlung, photon conversion, charged
particle energy loss by ionization, charged particle multiple scattering, nuclear interactions, and
electron, photon and hadron showering. The various CMS subsystems are modelled in different ways,
with various levels of approximation. 

The tracking detector is the most complicated CMS subsystem, and the one for which the most
approximations are made, including for the geometry.
The tracker is modelled by 30 thin nested cylinders, and is assumed to be made of pure silicon,
uniformly distributed over each layer. The thickness of each layer in terms of interaction lengths
was tuned to reproduce the number of bremsstrahlung photons above a certain threshold as observed in
FullSim.
Charged particles are propagated between two detector surfaces according to the magnetic field, and
experience multiple scattering and energy loss by ionization. The intersections between the
trajectories and each tracker layer define the position of the simulated hits that are then
converted into reconstructed hits with a certain efficiency, which is determined from FullSim. 

The showers of electrons and photons which hit the electromagnetic calori\-meter are simulated as if
the ECAL were a homogeneous medium. This is a reasonable approximation because the ECAL crystals
are organized to have almost no gaps in between them. 
Electrons and photons at rapidity values not covered by the electromagnetic calorimeter ($|\eta| >
3$) are propagated directly to the forward hadron calorimeter.

Charged and neutral hadrons are propagated to the start of the ECAL, HCAL and HF after their
interactions with the tracker layers. Their energy response is derived from the full
simulation. First the energy is smeared according to energy resolution measured in FullSim for
single pions. Then, this smeared energy is distributed in the calorimeters using parameterized
longitudinal shower profiles. 

Muons propagate through the tracker, the calorimeters, the solenoid, and the muon chambers. 
Both muons coming directly from the main interaction vertex and those produced inside the
tracker from the decay of another particle are included. 
The calorimeter response is treated similarly to that of charged hadrons. In the muon systems the
only processes that are taken into account are multiple scattering and energy loss by ionization.

The reconstruction of FastSim tracker hits into tracks is done in a different, faster, way compared
to FullSim or data. The truth information is used to group hits together that come from the same
particle. Consequently, FastSim does not have fake tracks. Two hits in the same place are also not
merged to be one reconstructed hit, as would be the case for the real detector readout. 
Despite these simplifications, good agreement between FastSim and FullSim track reconstruction is
observed. 


