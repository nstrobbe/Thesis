\chapter{Event generation, simulation and reconstruction \label{chap:event_generation}}

\textcolor{red}{Outline to be finalized}

\section{What is an ``event''? \label{sec:event}}

% add picture of CMS event display
% add picture of hadronization etc for event

\section{Event generation \label{sec:event_generation}}

% add info on Madgraph, pythia, jet matching etc

\subsection{Matrix element generators}

\subsection{Parton shower and hadronization}

\subsection{Jet matching}


\section{Event simulation \label{sec:event_simulation}}

\subsection{CMS Full Simulation using Geant4 \label{subsec:fullsim}}

% add some info on how this is done, particle interaction with matter, ionization, 
% how it deals with long lived particles?

\subsection{CMS Fast Simulation \label{subsec:fastsim}}

% explain what approximations are made, why it is necessery, especially in SUSY searches

\section{Event reconstruction \label{sec:event_reconstruction}}

% Add subsection on each object with all the object definitions we use

% We select events with at least one interaction vertex, associated with at least 4 charged-particle
% tracks, that lies within 24~cm of the origin of the CMS coordinate system along the beam direction
% and 2~cm from the origin in the plane transverse to the beam. Owing to the high luminosity of the
% LHC, hard scattering events are typically accompanied by many additional events arising from the
% multiple proton-proton interactions that occur when the proton bunches cross.  The overlapping of
% events is referred to as pileup.  The primary vertex is identified as the vertex with the highest
% value of the $\sum \pt^2$ of the associated tracks.  Detector- and beam-related cleaning algorithms
% are used to remove events with detector noise that mimic events with high energy and large imbalance
% in transverse momentum.  


 

\subsection{Particle Flow technique \label{sec:event_reco_pf}}

% CMS reconstructs events using the particle flow (PF) method~\cite{PF}, which reconstructs particles
% (PF candidates) by combining information from the inner tracker, the calorimeters, and the muon
% system.  Each PF candidate is a ssigned to one of five object categories: muons, electrons, photons,
% charged hadrons, and neutral hadrons.  Contamination from pileup events is reduced by discarding
% charged PF candidates that are incompatible with having originated from the primary
% vertex~\cite{CMS-PAS-JME-14-001}.   The average pileup energy due to neutral hadrons is computed
% event-by-event and subtracted from the energy when computing lepton isolation and jet energy.  The
% energy subtracted is  the average pileup energy per unit area (in $\Delta\eta \times \Delta\phi$)
% times the jet area~\cite{Fastjet1, Fastjet2}.

\subsection{Object identification}

The event selection is an integral part of any physics analysis. It determines which events are
used, and thus what processes contribute to the data sample. This in turn drives how the
backgrounds are estimated, what the sensitivity will be, et cetera. 
An event selection is most easily described in terms of particles, e.g. two electrons, no muons, at
least four jets, as this is the closest to how we think about a given process.  
The particle flow technique is very compatible with this approach, given that it already
reconstructs particles out of the detector hits. 
However, a more thorough selection of the PF objects is needed in order to ensure that their
behaviour is understood, and to ensure that the selected events are not dominated by
misidentified particles, or detector artefacts. 
The physics object groups (POG's) within the CMS Collaboration are in charge of providing general
recommendations on how to define each object. 
In the following paragraphs all the standard objects that will be used in the razor boost analysis
will be discussed.

% 
% Missing transverse energy, which is used in the calculation of the razor variable $\mr$, is 
% defined to be the negative sum of the transverse momenta of all the particle flow objects in an
% event.  Loosely identified and isolated electrons with $\pt > 5$~\GeV and $|\eta| < 2.5$ and muons
% with $\pt > 5$\GeV and $|\eta| < 2.4$ are used both to suppress backgrounds in our signal region and
% in the definition of the control regions.  A tight definition of isolated leptons (electrons with
% $\pt > 10$~\GeV and $|\eta| < 2.5$ and muons with $\pt > 10$~\GeV and $|\eta| < 2.4$) defines a
% control region enriched in $\cPZ \rightarrow \ell \ell $ events, from which we estimate the
% systematic uncertainty in the predicted number of $\cPZ \rightarrow \nu \nu$ events in the signal
% region. Any electron candidates with $1.44 < |\eta| < 1.57$ are rejected since the transition region
% between barrel and endcap calorimeters is less well-instrumented.
% In order to suppress the decays of taus and other leptons that fail the loose selection, events that
% have isolated tracks with $\pt > 10$\GeV and track-primary vertex distance along the beam direction
% $dz < 0.05$ are rejected.


%%%%%%%%%%%%%%%%%%%%%%%%%%%%%%%%%%%%%%%%%%%%%%%%%%%%%%%%%%%%%%%%%%%%%%%%%%%%%%%%%%%%%%%%%%%%%%%%%

\subsubsection{Vertex Selection \label{sec:object_vertex}}

We require at least one {\it good} vertex to be reconstructed in each event, according to the
definitions given in Table~\ref{tab:object_vertex}.
The vertex with the highest value of $\sum \pt$ of associated tracks is chosen as the primary vertex
in the event. This vertex is taken as a reference to reconstruct the event, e.g. to perform the
track subtraction for pileup removal, for which we use the {\tt PFNoPileUp} algorithm.

\begin{table}[htdp]
\caption{Vertex selection criteria. \label{tab:object_vertex}}
\begin{center}
\begin{tabular}{l l}
\toprule
\texttt{isFake()} & $= 0$ \\
\texttt{ndof()} & $> 4$ \\
\texttt{z()} & $< 24$ \\
\texttt{position.Rho()} & $< 2$ \\
\bottomrule
\end{tabular}
\end{center}
\end{table}


\subsubsection{Jet selection \label{sec:object_jets}}

Jets are clustered from the set of PF candidates with \textsc{FastJet 3.0.1}~\cite{Cacciari:2011ma}
using the anti-$k_\textrm{T}$ algorithm~\cite{antikt} with size parameter $\Delta R=0.5$ (AK5).  
Jet corrections are applied as a function of jet $\pt$ and $\eta$ to account for the residual
effects of non-uniform detector response.  
The jet energies are corrected so that, on average, they match those of simulated particle-level
jets. 

% TODO Add information on jet corrections and pileup subtraction


% The jets are corrected for pile-up effects in a two step process.  
% First charged hadron particle-flow candidates that have been associated with a pile-up vertex are
% removed from the list of particles to be clustered using the {\tt PFNoPileUp} algorithm.  
% The jets are then clustered and corrected for the L2 and L3 corrections, taking into
% account the charged-hadron removal. 
% The remaining PU energy is subtracted by applying the event-by-event quantity $\pi \rho (\Delta
% R)^2$, where $\Delta R$ is the jet size and $\rho$ is the average density from PU events, as
% computed by {\tt FastJet} using only neutral hadron particle-flow candidates.  

After all corrections are applied, jets are required to have $\pt > 30$~\GeV and $|\eta| < 2.4$.  
We also apply the standard loose identification criterion, defined by the requirements listed in
Table~\ref{tab:object_jets}. 

\begin{table}[htdp]
\caption{Jet selection \label{tab:object_jets}}
\begin{center}
\begin{tabular}{l l}
\toprule
\pt & $> 30$ \\
$|\eta|$ & $< 2.4$ \\
\midrule
\texttt{neutralHadronEnergyFraction()} & $< 0.99$ \\
\texttt{neutralEmEnergyFraction()} & $< 0.99$ \\
\texttt{nConstituents()} & $> 1$ \\
\texttt{chargedHadronEnergyFraction()} & $> 0$ \\
\texttt{chargedMultiplicity()} & $> 0$ \\
\texttt{chargedEmEnergyFraction()} & $< 0.99$ \\
\bottomrule
\end{tabular}
\end{center}
\end{table}

The AK5 jets defined here will be used for most of the razor boost analysis, except to reconstruct
boosted hadronic $\W$-candidates. For this more details can be found in
Section~\ref{sec:boost_wtag}.

\subsubsection{B-Tagging \label{sec:object_btag}}

Jets originating from the hadronization of $\cPqb$ quarks can be distinguished from other jets,
initiated by gluons or light flavor quarks, due to the long lifetime of the $\cPqb$ quark. 
The non-prompt decay of the $B$ hadrons results in a secondary vertex, displaced with respect to
the primary vertex of the hard interaction. 

% TODO add more info on b-tagging algorithm

The ability to distinguish $\cPqb$ jets is especially important for new physics searches. Many new
physics models are associated with production of third generation quarks, whereas this is more rare
in the standard model. For many searches $\cPqb$ jet tagging is an essential tool in suppressing
the background from multijet or vector boson production. 

In the razor boost analysis $\cPqb$ tagging will also be employed. We will use the combined
secondary vertex (CSV) algorithm at two working points~\cite{btag7TeV,btag8TeV,BTagWP}, which are
shown on Table~\ref{tab:object_btag}. 
The Loose working point (CSVL), corresponding to a misidentification rate of $\sim$10\% and
efficiency of $\sim$85\%, will be used to veto $\cPqb$ jets, whereas the Medium working point
(CSVM), corresponding to a misidentification rate of $\sim$1\% and a typical efficiency of
$\sim$70\% , is used to select $\cPqb$ jets. 

\begin{table}[htdp]
\caption{Working points for the combined secondary vertex $\cPqb$ jet tagger.
\label{tab:object_btag}}
\begin{center}
\begin{tabular}{l l}
\toprule
Working point & Discriminator value \\
\midrule
Medium & $> 0.679$ \\
Loose & $> 0.244$ \\
\bottomrule
\end{tabular}
\end{center}
\end{table}
% 
% As will be explained in section~\ref{sec:selection}, we define our signal and control regions based
% on the number of b-tagged jets. 
% As the b-tagged jet multiplicity distribution is not exactly the same in data as in simulation, we
% need to apply appropriate Data/MC scale factors to the simulation. These scalefactors and their
% prescription have been provided by the BTag POG \cite{BTagSF1,BTagSF2}. 
% Whenever an explicit selection based on the number of b-tagged jets is made, the btag scale factors
% are applied to the simulation. 
% For more detailed information on the scale factors and their associated uncertainties, we refer to
% section~\ref{sec:btag_uncertainties}. 

% TODO Decide where to put the scale factor information

\subsubsection{Muons \label{sec:object_muon}}

Muons are identified using two different working points, a loose selection and a tight selection,
both of which will be detailed below. 


% Currently we mainly use the loose definition in the analysis, both for vetoing, and for selecting
% single muon events for the control regions enriched in TTJets and WJets. The tight selection is
% only used to define a control region enriched in $Z\rightarrow ll$ events, from which we derive a
% systematic uncertainty on the predicted number of $Z\rightarrow\nu\nu$ events in our signal
% region.

The \textbf{loose muon selection} that will be employed was developed especially for events with a
large amount of hadronic activity, where the standard identification criteria were observed to lose
efficiency, resulting in less background suppression when vetoing the presence of muons. 
The details and performance of this optimized selection is documented in
Ref.~\cite{CMS-AN2011-498}. 
The main feature is the use of a so-called \textit{directional} isolation.
The isolation of a particle is a measure of how far it is from other activity in the detector. The
leptons we are interested in, those originating in the hard interaction, are usually separated from
other activity, e.g. jets. This is not the case for misidentified muons or for muons from the decay
of heavy-flavour jets. Directional isolation is designed to have a better rejection of leptons from
these heavy-flavour jet decays, and is defined as
\begin{equation}
\overrightarrow{\mathrm{ISO}}(R) \equiv \sum_{\Delta R_{i} < R} \delta_{i}^{2}\pt{}_{i} ,
\end{equation}
where the sum is over all other particles $i$ within $\Delta R_{i}<R$ of the muon direction,
and $\delta_{i}$ is the angle between particle $i$ and the $\pt$-weighted centroid position
($\delta_{c}$) of all such particles in $(\eta,\phi)$ space. That is, if $\Delta\phi_i$ and
$\Delta\eta_i$ are respectively the difference in $\phi$ and $\eta$ angles between particle $i$ and
the muon, then:
\begin{eqnarray*}
\vec{e}_{i} & \equiv & \frac{1}{\sqrt{\Delta\phi_{i}^{2}+\Delta\eta_{i}^{2}}}\left(\begin{array}{c}
\Delta\phi_{i},\\
\Delta\eta_{i}
\end{array}\right),\\
\vec{\delta}_{c} & = & \sum_{\Delta R_{i}<R}\pt{}_{i}\vec{e}_{i},\\
\delta_{i} & = &
\angle(\vec{\delta}_{c},\vec{e}_{i})=\arccos(\vec{\delta}_{c}\cdot\vec{e}_{i}/|\vec{\delta}_{c}|),
\end{eqnarray*}
where $\vec{e}_{i}$ is the unit vector specifying particle $i$'s relative location in $(\eta,\phi)$
space with respect to the considered muon, as illustrated in Fig.~\ref{fig:object_directional_iso}.
Because of the weighting by $\delta_{i}^{2}$, the value for the directional isolation tends to be
larger for muons that are near the jet core, e.g. in case of leptonic $\cPqb$ decays, compared to
the more convential isolation definition which does not use this weighting. 

\begin{figure}[htpb]
  \centering
  \includegraphics[width=0.7\textwidth]{figures/eventreco_objects/directional_iso_cartoon}
  \caption{Illustration of ingredients used in the computation of directional isolation for a prompt
muon, denoted by a star, near some particles from a jet, denoted by points, in the $(\eta,\phi)$
plane. For prompt leptons $\delta_i$ tends to be small, especially for the high-\pt particles near
the core of the jet. Figure taken from Ref.~\cite{CMS-AN2011-498}.
  \label{fig:object_directional_iso}}
\end{figure}

Apart from the isolation, the identication criteria themselves are also altered from the standard
Loose Muon ID from the POG in order to further optimize the muon identification in environments
with large hadronic activity. 
Loose muons are reconstructed using either the global muon algorithm or the tracker-only
algorithm. 
Global muons are required to pass the {\tt GlobalMuonPromptTight} quality criteria,
and to have at least two muon chambers containing segments uniquely matched to its inner track. 
Tracker-only muons are required to pass the {\tt TMLastStationTight} criteria, which require the
muon to have compatible hits in the last muon chamber. 
All selected muons are then required to pass the selection listed in
Table~\ref{tab:object_loosemuon}. 
Some aspects of the selection depend on the muon $\pt$ and $\eta$; these are summarized in
Table~\ref{tab:object_loosemuon_cuts}.

\begin{table}[htbp]
\caption{Loose muon definition. }
\begin{center}
\begin{tabular}{l l}
\toprule
\pt & $> 5\GeV$ \\
$|\eta|$ & $< 2.4$ \\
\midrule
\texttt{innerTrack().hitPattern().numberOfLostHits()} & $\leq 1$ if $\pt < 20\GeV$ \\
                                                      & $\leq 4$ if $\pt \geq 20\GeV$ \\
$|\texttt{innerTrack().dxy(vertex.position())}|$ & $\pt$- and $\eta$-dependent\\
$|\texttt{muonBestTrack().dz(vertex.position())}|$ & $\pt$- and $\eta$-dependent\\
\midrule
$\overrightarrow{\mathrm{ISO}}(R=0.2)$ & $\pt$- and $\eta$-dependent \\
\bottomrule
\end{tabular}
\end{center}
\label{tab:object_loosemuon}
\end{table}

\begin{table}[htbp]
\caption{Details of the $\pt$ dependent thresholds employed in the loose muon selection.}
\begin{center}
  \begin{tabular}{l cccccc }
      \toprule
      Muon $\pt$  & $d_{xy}$ & $d_{xy}$ & $d_z$ & $d_z$ & $\overrightarrow{\mathrm{ISO}}(0.2)$ &
$\overrightarrow{\mathrm{ISO}}(0.2)$ \\
      (\GeV) & barrel & endcap & barrel & endcap & barrel & endcap \\
      \midrule
      0 - 5          & 0.052 & 0.037 & 0.054 & 0.076 & 1.5  & 2    \\
      5 - 10         & 0.041 & 0.018 & 0.042 & 0.082 & 3    & 2.5  \\
      10 - 25        & 0.029 & 0.013 & 0.028 & 0.098 & 7    & 7.5  \\
      15 - 20        & 0.014 & 0.015 & 0.034 & 0.1   & 10.5 & 9    \\
      20 - 40        & 0.021 & 0.021 & 1     & 0.1   & 15.5 & 13.5 \\
      40 - 80        & 0.04  & 0.2   & 1     & 1     & 32.5 & 19   \\
      80 - 140       & 0.1   & 0.2   & 1     & 1     & 54.5 & 37   \\
      140 - 200      & 0.1   & 0.2   & 1     & 1     & 87   & 65.5 \\
      \bottomrule
    \end{tabular}
\end{center}
\label{tab:object_loosemuon_cuts}
\end{table}

 
The \textbf{tight muon selection} follows the recommendation from the Muon POG~\cite{MuonID}.
In addition to the identification criteria, we also require the tight muon to be isolated. 
Here we do not use directional isolation, but rather the more standard particle-based relative
isolation. 
This isolation, denoted $I_\mu$, is calculated using the PF candidates in a cone of size $\Delta R =
0.4$ around the muon. Charged-hadron candidates associated with pileup vertexes are not taken into
account in the calculation of the isolation. However, they are used to estimate the remaining
contribution to the isolation coming from neutral hadrons associated with pileup. This contribution
is then subtracted. 
The isolation definition is given by:
\begin{equation}
I_\mu = \frac{I_{Charged} + I_{Neutral} + I_{\gamma} - \Delta\beta\cdot I_{Charged}^{PU}}
             {\pt^\mu} , 
\label{eqn:iso}
\end{equation}
where $I_{Charged}$, $I_{Neutral}$, and $I_{\gamma}$ are computed as the sum of the \pt of the
charged hadrons, neutral hadrons and photons, respectively, in a cone of size $\Delta R = 0.4$
around the muon. The parameter $\Delta\beta$ is set to 0.5, and $I_{Charged}^{PU}$ is the estimated
contribution from pileup computed as the sum of the \pt of the charged hadrons associated with
pileup vertices.
The tight muon isolation requirement is $I_\mu < 0.15$.
A summary of the tight muon selection can be found in Table~\ref{tab:object_tightmuon}. 

\begin{table}[htdp]
\caption{Tight muon definition. }
\begin{center}
\begin{tabular}{l l}
\toprule
\pt & $> 10\GeV$ \\
$|\eta|$ & $< 2.4$ \\
\midrule
\texttt{isPFMuon()} & $= 1$ \\
\texttt{isGlobalMuon()} & $= 1$ \\
\texttt{globalTrack().normalizedChi2()} & $< 10$ \\
\texttt{globalTrack().hitPattern().numberOfValidMuonHits()} & $> 0$ \\
\texttt{track().hitPattern().trackerLayersWithMeasurement()} & $> 5$ \\
\texttt{innerTrack().hitPattern().numberOfValidPixelHits()} & $> 0$ \\
\texttt{numberOfMatchedStations()} & $> 1$ \\
$|\texttt{innerTrack().dxy(vertex.position())}|$ & $< 0.2$ \\
$|\texttt{muonBestTrack().dz(vertex.position())}|$ & $< 0.5$ \\
\midrule
$I_\mu =$ [\texttt{pfIsolationR04().sumChargedHadronPt()}& \\
\hspace{0.9cm} $+$ max(0., \texttt{pfIsolationR04().sumNeutralHadronPt()}  & \\
\hspace{2.7cm} $+$ \texttt{pfIsolationR04().sumPhotonPt()}  & \\
\hspace{2.7cm} $-$ 0.5 $\cdot$ \texttt{pfIsolationR04().sumPUPt()}) & \\
\hspace{0.9cm} ] / \pt & $< 0.15$ \\ 
\bottomrule
\end{tabular}
\end{center}
\label{tab:object_tightmuon}
\end{table}

 

\subsubsection{Electrons \label{sec:object_electron}}
% 
% We identify electrons at two different working points; a tight selection and a loose selection. 
% Currently we mainly use the loose definition in the analysis, both for vetoing, and for selecting
% single electron events for the control regions enriched in TTJets and WJets. The tight selection is
% only used to define a control region enriched in $Z\rightarrow ll$ events, from which we derive a
% systematic uncertainty on the predicted number of $Z\rightarrow\nu\nu$ events in our signal region.
% 
% \subsubsection{Tight Selection \label{sec:tightele}}
% 
% \begin{table}[htdp]
% \caption{Tight electron definition}
% \begin{center}
% \begin{tabular}{|lll|}
% \hline
% \multicolumn{3}{|l|}{CMSSW collection label: } \\
% \multicolumn{3}{|l|}{CMSSW type: pat::Electron} \\
% \hline
% Variable/Method & \multicolumn{2}{l|}{Value} \\
% \hline
% pt() & $> 10$ & \\
% fabs(eta()) &  \multicolumn{2}{l|}{$< 2.5$ and not in $1.442 - 1.556 $} \\
% \hline
% ID & Barrel & Endcap \\
%      & $|\eta| < 1.479$ & $1.479 < |\eta| < 2.5$ \\
%      & isEB() $= 1$ & isEE() $= 1$ \\ 
% \hline
% deltaEtaSuperClusterTrackAtVtx() & $< 0.004$ & $< 0.007$ \\
% deltaPhiSuperClusterTrackAtVtx() & $< 0.060$ & $< 0.030$ \\
% sigmaIetaIeta() & $< 0.010$ & $< 0.030$ \\
% hadronicOverEm() & $< 0.120$ & $< 0.100$ \\
% 1.0/ecalEnergy() - eSuperClusterOverP()/ecalEnergy() ) & $< 0.050$ & $< 0.050$ \\
% gsfTrack().get().trackerExpectedHitsInner().numberOfHits() & $\le 1$ & $\le 1$ \\
% passConversionVeto() & $= 1$ & $= 1$ \\
% \hline
% \end{tabular}
% \end{center}
% \label{tab:tightele}
% \end{table}
% 
% 
% 
% The tight electron selection is taken from the EGamma POG \cite{ElectronID}. A summary of the
% selection can be found in table~\ref{tab:tightele}. 
% 
% We also require to electron to be isolated. The isolation is calculated using the particle-flow
% candidates in a cone of size 0.3 around the electron, and then corrected with an estimate of the
% median energy from pile-up as calculated with the {\tt FastJet} algorithm in a similar way to the
% jet corrections explained in Sec.~\ref{sec:jets}. We require that this corrected isolation, weighted
% by the $p_T$ of the electron is less than 0.15.
% 
% \begin{equation}
% I_e(p_T^e) = \left[ I_{Charged} + \max(0., (I_{NeutralHad} + I_{\gamma})) - A \rho \right] / p_T^e
% \qquad
% \end{equation}
% 
% \subsubsection{Loose electron selection \label{sec:looseele}}
% 
% As for the muons, we follow the stop working-group's recommendation for a loose-electron
% identification \cite{StopWG}. The exact details and performance of this selection is documented in
% Ref.~\cite{Ra2Top}. In summary, we require:
% \begin{itemize}
% \item $p_T > 5 \GeV$;
% \item $|\eta| < 2.5$ and ($|\eta| < 1.442$ OR $|\eta| > 1.556$) ; 
% \item No lost [expected] tracker hits for $\pt < 20\GeV$ electrons.
% \item At least 2 (1) pixel hits for $\pt < 10\GeV$ electrons in the barrel (endcap). 
% \item $\pt$- and $\eta$-dependent maximum $d_{z}$, the absolute difference between the electron and
% primary vertex $z$-positions.  
% \item $\pt$- and $\eta$-dependent maximum directional isolation sum with $R = 0.3$ and calculated
% with only charged particles.
% \item For electrons in the barrel only: an additional $\pt$- and $\eta$-dependent maximum
% directional isolation sum with $R = 0.2$ and calculated using all particles.
% \end{itemize}
% Details of the $\pt$- and $\eta$-dependent cuts can be found in Ref.~\cite{Ra2Top}, and a
% description of the directional isolation can be found in Sec.~\ref{sec:loosemuon}.
% 

\begin{table}[htb]
  \caption{Details of the $\pt$ dependent thresholds employed in the Loose Electron selection.}
  \begin{center}
  \begin{tabular}{ l ccccc }
      \hline
      Electron $\pt$ & $d_z$ & $d_z$ & charged iso & charged iso & iso \\
      (\GeV) & barrel & endcap & barrel & endcap & barrel \\
      \hline
      0 - 5          & 0.03 & 0.09 & 0.5  & 0.5  & 2    \\
      5 - 10         & 0.05 & 0.09 & 1.5  & 2.5  & 4.25 \\
      10 - 25        & 0.05 & 0.09 & 4.5  & 6.5  & 8.75 \\
      15 - 20        & 0.05 & 0.11 & 7.5  & 9    & 11   \\
      20 - 40        & 0.2  & 1    & 10   & 10.5 & 20.8 \\
      40 - 80        & 1    & 1    & 18.5 & 18.5 & 200  \\
      80 - 140       & 1    & 1    & 44   & 66.5 & 200  \\
      140 - 200      & 1    & 1    & 81.5 & 70   & 200  \\
      \hline
    \end{tabular}
  \end{center}
  \label{tab:object_looseelectron_cuts}
\end{table}
% 
\subsubsection{Isolated tracks \label{sec:object_isolatedtrack}}
% 
% In order to suppress the decays of both taus and other leptons that do not pass the loose selection,
% we employ an isolated track veto at particle-flow candidate level, following
% Ref.~\cite{StopSingleLepton}.  
% Isolated tracks are selected from the charged particle flow candidates with $p_T > 10\GeV$ and
% longitudional track-primary vertex distance of $dz < 0.05$.  
% 
% In the hadronic event selection, events with an isolated track will be vetoed.  
% 
% In event selections involving a tight lepton (none existent in the analysis at this point), isolated
% track veto can be used for vetoing events with a second lepton.  In this case, in order to ensure
% that the track under question does not belong to the tight lepton, $dR(\ell_{tight}, track) > 0.3$
% is required.  
% 
% \begin{table}[htdp]
% \caption{Isolated track selection.  See Section~\ref{sec:vertex} for primary vertex definition and
% Sections~\ref{sec:tightmuon} and~\ref{sec:tightele} for tight muon and electron definitions.}
% \begin{center}
% \begin{tabular}{|ll|}
% \hline
% \multicolumn{2}{|l|}{CMSSW collection label: pfNoPileUp} \\
% \multicolumn{2}{|l|}{CMSSW type: reco::PFCandidate} \\
% \hline
% Variable/Method & Value \\
% \hline
% pt() & $= 0$ \\
% charge() & $> 0$ \\
% $dz($PV, track$)$ & $< 0.05$ \\
% $\left[ \sum_{j \neq i} pt()_j \right] / pt()_i$ & $< 0.1$ \\
% \hline
% \multicolumn{2}{|l|}{For selections with tight leptons:} \\
% \hline
% $dR(\ell_{tight}, {\rm track})$ & $ > 0.3$ \\
% \hline
% \end{tabular}
% \end{center}
% \label{tab:isolatedtrack}
% \end{table}
% 
% We remove particle-flow candidates corresponding to an electron or muon passing the tight selection
% with a $\Delta R$ cut of 0.1. Isolated tracks are those particle-flow candidates coming from the
% primary vertex that are charged, have $p_T > 10$GeV and have relative isolation in a cone of size
% 0.3 less than 0.1. In the leptonic final states, the isolated track veto is only applied if the
% charge of the track is opposite to that of the tight lepton.


\subsubsection{Missing transverse momentum \label{sec:object_met}}

The missing transverse momentum, \ETm, associated with a given event is computed as the negative
vector sum of the transverse momentum of all PF candidates $i$,
\begin{equation}
  \ETm = - \sum_i \pt^i .
\end{equation}
The corrections to the jet energy scale discussed above are propagated to the \ETm as well. 
Within CMS this type of missing transverse momentum is know as type-1 corrected \ETm.

No explicit selection will be placed on \ETm in the razor boost analysis selection, although it is
used in the definition of the razor variable $\rsq$, to be introduced in
Section~\ref{sec:boost_razor}.



% \subsection{Pileup reweighting \label{sec:pileup}}
% 
% The distribution of the number of pileup interactions is different in Data vs MC. 
% As the number of pileup interactions can have an influence on various aspects of the reconstruction,
% we need to reweight the MC events to match the pileup distribution in data. 
% The method to do this is quite straightforward. 
% 
% \begin{enumerate}
% 
% \item Determine the pileup distribution in data by using a centrally provided script:
% {\footnotesize
% \begin{verbatim}
% pileupCalc.py -i Cert_190456-208686_8TeV_22Jan2013ReReco_Collisions12_JSON.txt 
%               -o data_pileup.root --inputLumiJSON=pileup_latest.txt 
%               --calcMode=true --minBiasXsec=69400 --maxPileupBin=100
% \end{verbatim}
% }
% In this, \texttt{pileup\_latest.txt} is taken directly from the central DQM area, as is the Lumi
% JSON. The Lumi JSON used is the one for the full Jan22 ReReco, corresponding to the data we are
% using, see section~\ref{sec:samples}. 
% 
% \item Determine the pileup distribution in simulation: \\ 
% For this we use a variable called \texttt{trueNumInteractions}. 
% The distribution of this variable for all MC events is used as the pileup distribution for Monte
% Carlo. 
% 
% \item Normalize the MC and data histograms to unit area, and divide them (Data/MC). The resulting
% histogram contains the event weights. 
% 
% \item Apply these event weights to each MC event, by picking up the value for
% \texttt{TrueNum\-Interactions} for that event, finding in what bin it falls and then applying the
% corresponding weight.
% 
% \end{enumerate}
% 
% The distribution of the pileup in data and simulation, and the corresponding pileup weight is shown
% in Figure~\ref{fig:DataMC_pileup} for the FullSim simulation and in figure~\ref{fig:DataMC_pileup52}
% for the FastSim simulation using CMSSW\_52X. 
% 
% \begin{figure}[p]
%  \centering
%  \includegraphics[width=0.7\textwidth]{figures/Pileup/compare_pileup_profile}
% \caption{Comparison plot of the true number of interactions in data and in MC. On the top the
% distribution of both are shown, on the bottom the ratio Data/MC. Left is linear scale, right is log
% scale. 
% \label{fig:DataMC_pileup}}
% \end{figure}
% 
% \begin{figure}[p]
%  \centering
%  \includegraphics[width=0.7\textwidth]{figures/Pileup/compare_pileup_profile_52X}
% \caption{Comparison plot of the true number of interactions in data and in MC processed with
% CMSSW\_52X. On the top the distribution of both are shown, on the bottom the ratio Data/MC. Left is
% linear scale, right is log scale. 
% \label{fig:DataMC_pileup52}}
% \end{figure}
% 
% As a test of the performance of the pileup reweighting, we can check the agreement between data and
% simulation for the distribution of the number of good primary vertices ($PV$) at different selection
% levels. 
% We expect to find a reasonable, although not perfect agreement as the vertex reconstruction
% efficiency depends on many things. 
% This comparison is shown in figure~\ref{fig:comparison_PV}. 
% 
% \begin{figure}
%  \includegraphics[width=0.49\textwidth]{figures/Pileup/DataMC_PV_0Lb1Ll}
%  \includegraphics[width=0.49\textwidth]{figures/Pileup/DataMC_PV_g1Mb1Ll}
% \caption{Data/MC comparison plot of the number of good primary vertices after pileup reweighting for
% a control region enhanced in $W+$jets (left) and enhanced in $t\bar{t}+$jets (right).
% \label{fig:comparison_PV}}
% \end{figure}


% \subsection[Top pt reweighting]{Top \pt reweighting \label{sec:toppt_reweighting}}
% 
% The TOP group has found in the normalized differential top-quark-pair cross section analysis that
% the shape of the \pt spectrum of the individual top quarks in data is softer than predicted by the
% various simulations while the available approx. NNLO prediction delivers a reasonable description.
% Based on this measurement, they have derived event scalefactors with associated systematic
% uncertainties to test the potential impact of the modelling of the top quark \pt spectrum
% \cite{TopPt}. 
% 
% The event weight is derived as a function of the generated \pt of both the top and anti-top quark in
% the event: 
% \begin{equation}
% w_{\textrm{TopPt}} = \sqrt{ SF_t * SF_{\bar{t}} }
% \end{equation}
% \begin{equation}
% SF(\pt^{gen}) = \exp(a + b \pt^{gen})
% \end{equation}
% with $a = 0.156$ and $b = -0.00137$.
% 
% The uncertainty on this reweighting is given by the following prescription: 
% \begin{align}
% +1~\sigma &: w = w_{\textrm{TopPt}} * w_{\textrm{TopPt}} \\
% -1~\sigma &: w = 1 
% \end{align}
% where $w$ is the weight to be applied to the event for the up/down variation. 
% 
% In figure~\ref{fig:TopPt} we show the Data/MC comparison for the $M_R$ and $R^2$ distribution in the
% $t\bar{t}+$jets control region (see section~\ref{sec:Tregion}) before and after applying the
% reweighting procedure. 
% We observe that this reweighting greatly improves the agreement between data and simulation. 
% Therefore we will always apply this reweighting to the $t\bar{t}+$jets simulated sample. 
% 
% \begin{figure}[htpb]
% \centering
% \includegraphics[width=0.49\textwidth]{
% figures/DataMC/DataMC_MR_g1Mbg1W1LlmT100_mdPhig0p5_width_noTopPt}
% \includegraphics[width=0.49\textwidth]{
% figures/DataMC/DataMC_R2_g1Mbg1W1LlmT100_mdPhig0p5_width_noTopPt}
% 
% \includegraphics[width=0.49\textwidth]{figures/DataMC/DataMC_MR_g1Mbg1W1LlmT100_mdPhig0p5_width}
% \includegraphics[width=0.49\textwidth]{figures/DataMC/DataMC_R2_g1Mbg1W1LlmT100_mdPhig0p5_width}
% \caption{[top] $M_R$ (left) and $R^2$ (right) distribution before applying the top \pt reweighting
%          [bottom] $M_R$ (left) and $R^2$ (right) distribution after applying the top \pt reweighting
% for the $T$ region as defined in section~\ref{sec:Tregion}
% \label{fig:TopPt}}
% \end{figure}

% 
% \subsection{ISR reweighting \label{sec:ISRreweighting}}
% 
% As recommended, we apply the ISR reweighting recipe developed in the SUSY group
% \cite{ISRreweighting} to all our simulated signal samples.
% Each event is reweighted with an event weight, depending on the \pt of the system that is recoiling
% against the ISR jet(s). 
% For the T1ttcc simplified model this system is the system of the two gluinos. For T2tt this is the
% system of the two stop squarks. 
% In table~\ref{tab:ISRreweighting} we list the scale factor to be applied for the different \pt
% ranges. 
% The difference between that scale factor and 1 is taken to be the one standard deviation
% uncertainty. 
% 
% \begin{table}[htpb]
% \centering
% \caption{ISR reweighting prescription \label{tab:ISRreweighting}}
% \begin{tabular}{|c|c|}
% \hline
% \pt of recoiling system & Scale factor \\ \hline
% $\leq 120$\GeV & 1 \\
% $120 - 150 $\GeV & 0.95 \\
% $150-250$\GeV & 0.9 \\
% $> 250$\GeV & 0.8 \\
% \hline
% \end{tabular}
% \end{table}

\subsection{Event Cleaning \label{sec:event_cleaning}}
% 
% We follow the recommended procedure set out by the JetMET POG, and apply a series of event
filters.
% These are:
% 
% \begin{itemize}
% \item The {\tt EcalDeadCellTriggerPrimitiveFilter}, which removes events where dead cells in the
% ECAL produce anomalous activity.
% \item The {\tt hcalLaserEventFilter}, which removes events where the HCAL laser produces anomalous
% activity.
% \item The {\tt hcalLaserEventFilter2012}, 
% \item The {\tt trackingFailureFilter}, which removes events where the tracking algorithm does not
% perform properly.
% \item The {\tt CSCTightHaloFilter}, which removes events contaminated by beam halo.
% \item The {\tt HBHENoiseFilter}, which removes events featuring large hadronic calorimeter noise.
% \item The {\tt eeBadScFilter}, which removes events featuring high amplitude anomalous pulses due
to
% bad ECAL super-crystals.
% \item The {\tt trkPOGFilters}, which remove events due to track reconstruction anomalies, such as
% events with partly aborted track reconstruction and events affected by the Strip Tracker coherent
% noise.
% \item The {\tt primaryVertexFilter}, which removes events that do not have a good primary vertex.
% \item The {\tt noscrapingFilter}, which removes events with a large multiplicity of low quality
% tracks.
% \end{itemize}
% 
% More details on these filters can be found in Ref.~\cite{metfilters}. For MC samples that are
% passed
% through the fast CMS detector simulation, the {\tt CSCTightHaloFilter} and {\tt HBHENoiseFilter}
% filters are not applied, as their input collections are not produced in {\sc FastSim} productions.
% 
% In addition to the filters listed above, we also use a cleaning cut to remove events with spurious
% HCAL noise. This is explained in more detail in section~\ref{sec:noise}. 

\subsubsection{HCAL Noise}\label{sec:noise}

% Starting with the 52X releases of CMSSW, the reconstruction of particle-flow jets and particle-flow
% MET  
% includes energy measured in the CMS Hadron Outer Calorimeter (HO), while the reconstruction of 
% calorimetric jets and calorimetric MET does not. 
% 
% Events may receive a large, anomalous contribution to the PF MET from HO energy. 
% In these events, the PF MET differs from the calo MET substantially, 
% even in the absence of muons (which also contribute to the PF MET but not the calo MET).
% Since no filter has yet been developed,
% we implement a cleaning cut to remove these events. In the following PF MET 
% always refers to type-1 corrected PF MET, while CALO MET refers to the global muon corrected calo
% MET.
% 
% We cut events in which the PF MET vector $\MET^{,\textrm{PF}}$ is flipped with respect 
% to the calo MET vector $\MET^{,\textrm{CALO}}$. 
% To accomplish this we compute the absolute value of the difference in polar angle
%  $|\Delta\phi_{\textrm{PF,CALO}}|$, taken to in the range $[0,2\pi)$, and defined as
% \begin{equation}
%  \phi^{\textrm{PF}} = \textrm{arctan}\left( \frac{\MET^{,\textrm{PF}}_y}{\MET^{,\textrm{PF}}_x}
% \right) ,~~~~
%  \phi^{\textrm{CALO}} = \textrm{arctan}\left(
% \frac{\MET^{,\textrm{CALO}}_y}{\MET^{,\textrm{CALO}}_x} \right)
% \end{equation}
% \begin{equation}
% |\Delta\phi_{\textrm{PF,CALO}}| = \min \left ( \phi^{\textrm{PF}} - \phi^{\textrm{CALO}},   2\pi -
% \phi^{\textrm{PF}} + \phi^{\textrm{CALO}} \right)
% \end{equation}
% 
% Then we cut events in which $|\Delta\phi_{\textrm{PF,CALO}}|$ falls in a 1 radian ($\sim57$ degree)
% window centred around $\pi$. This cut is applied to all boxes to remove anomalous events with large
% PF MET.
% \begin{equation}
% ||\Delta\phi_{\textrm{PF,CALO}}| -\pi | < 1
% \label{eqn:dphicut}
% \end{equation}