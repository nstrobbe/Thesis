%%%%%%%%%%%%%%%%%%%%%%%%%%%%
%% Event generation 
%%%%%%%%%%%%%%%%%%%%%%%%%%%%

% add info on Madgraph, pythia, jet matching etc

In this section I will explain in more detail the various techniques employed by the most common
event generators. In particular, I will focus on \textsc{Madgraph} and \textsc{Pythia}, the programs
that generated the events for most of the processes used in the Razor Boost analysis, presented in
Chapter~\ref{chap:razorboost}. 
The following sections are based on Refs.~\cite{Campbell:2006wx,Salam:2010zt,Buckley:2011ms}. 

\subsection{Matrix element generators \label{sec:event_matrix_element_generators}}

% explain basics of how madgraph works internally 
% narrow width approx?

% From P. Skands:
% Using PDFs extracted using higher-order matrix elements in lower-order calculations, as, e.g.,
% when using NLO PDFs as input to an LO calculation. In principle, the higher-order PDFs are
% better constrained and the difference between, e.g., an NLO and an LO set should formally be
% beyond LO precision, so that one might be tempted to simply use the highest-order available
% PDFs for any calculation. However, as described in section 2.4, it is often possible to partly
% absorb
% higher-order terms into lower-order coefficients. In the context of PDFs, the fit parameters
% of lower-order PDFs will effectively attempt to “compensate” for missing higher-order contributions
% in the matrix elements. To the extent those higher-order contributions are universal, this
% is both desirable and self-consistent. However, this will only give an improvement when used
% with matrix elements at the same order as those used to extract the PDFs. It is therefore quite
% possible that NLO PDFs used in conjunction with LO matrix elements give a worse agreement
% with data than LO PDFs do


% There is a wide range of programs available, most notably ALPGEN [17, 18],
% the COMPHEP package [19, 20] and MADGRAPH [21, 22]. All of these programs
% implement the calculation of the diagrams numerically and provide a suitable phase
% space over which they can be integrated. ALPGEN uses an approach which is not based
% on a traditional Feynman diagram evaluation [23], whereas the other two programs rely
% on more conventional methods such as the helicity amplitudes evaluation of HELAS [24]
% in MADGRAPH.
% Although in principle these programs can be used to calculate any tree-level
% prediction, in practice the complexity of the process that may be studied is limited
% by the number of particles that is produced in the final state. This is largely due to the
% factorial growth in the number of Feynman diagrams that must be calculated. Even in
% approaches which do not rely directly on the Feynman diagrams, the growth is still as a
% power of the number of particles. For processes which involve a large number of quarks
% and gluons, as is the case when attempting to describe a multi-jet final state at a hadron
% collider such as the Tevatron or the LHC, an additional concern is the calculation of
% colour matrices which appear as coefficients in the amplitudes [25].


\subsection{Parton shower}

% discuss underlying event tunes etc
% discuss string model for hadronization


% For the case of parton showers from the initial state,
% the evolution proceeds backwards from the hard scale of the process to the cutoff scale,
% with the Sudakov form factors being weighted by the parton distribution functions at
% the relevant scales.

% In the parton showering process, successive values of an evolution variable t, a
% momentum fraction z and an azimuthal angle φ are generated, along with the flavours
% of the partons emitted during the showering. The evolution variable t can be the
% virtuality of the parent parton (as in PYTHIA versions 6.2 and earlier and in SHERPA),
% E 2 (1 − cos θ), where E is the energy of the parent parton and θ is the opening angle
% between the two partons (as in HERWIG) , or the square of the relative transverse
% momentum of the two partons in the splitting (as in PYTHIA 6.3). The HERWIG
% evolution variable has angular ordering built in, angular ordering is implicit in the
% PYTHIA 6.3 [85] evolution variable, and angular ordering has to be imposed after the
% fact for the PYTHIA 6.2 evolution variable. Angular ordering represents an attempt to
% simulate more precisely those higher order contributions that are enhanced due to soft
% gluon emission (colour coherence). Fixed order calculations explicitly account for colour
% coherence, while parton shower Monte Carlos that include colour flow information model
% it only approximately.
% Note that with parton showering, we in principle introduce two new scales, one for
% initial state parton showering and one for the shower in the final state. In the PYTHIA
% Monte Carlo, the scale used is most often related to the maximum virtuality in the
% hard scattering, although a larger ad hoc scale, such as the total centre-of-mass energy,
% can also be chosen by the user. The HERWIG showering scale is determined by the
% specific colour flow in the hard process and is related to the invariant mass of the colour
% connected partons.
% the Sudakov form
% factor gives the probability for a parton to evolve from a harder scale to a softer scale
% without emitting a parton harder than some resolution scale, either in the initial state
% or in the final state. Sudakov form factors form the basis for both parton showering and
% resummation
% A Sudakov
% form factor will depend on: (1) the parton type (quark or gluon), (2) the momentum
% fraction x of the initial state parton, (3) the hard and cutoff scales for the process and
% (4) the resolution scale for the emission.



The fixed-order matrix-element MC programs discussed in the previous section provide a powerful
combination of accuracy and flexibility as long as you want to calculate infrared and collinear safe
observables -- such as jets, $\W$ or $\cPZ$ bosons, but not pions, kaons, et cetera -- and
don’t need to study regions of phase space that involve disparate physical scales. An example of
the latter could be requiring a heavy boson to have a \pt much smaller than its mass, leading to
large coefficients at all orders in the perturbative expansion.
These defects are related to the presence of soft and collinear divergences in the calculations.
Real life does not diverge, however. We thus need a different approach to tackle the soft and
collinear part of the phase space. This approach is the parton shower. 

Parton shower algorithms, such as the one implemented in \textsc{Pythia}, describe the evolution in
momentum transfer from the high scales associated with the hard process down to the low scales, of
order 1\GeV, associated with the confinement of the partons it describes into hadrons. 
In analogy with bremsstrahlung of photons in QED, a parton (quark or gluon) with high momentum will
have some probability to radiate a gluon. This gluon can then radiate more gluons, or it can split
in a $q\bar{q}$ pair. This process repeats itself until the energy of the quarks and gluons becomes
too low, and hadronization begins. Hadronization is a non-perturbative process, but fortunately it
is universal, i.e. it does not depend on the hard interaction, but only on the partons at the low
scale after the parton shower.

The probabilities for the various parton splittings are encompassed in the splitting functions,
$P_{j\leftarrow i}$, which were already mentioned briefly in Section~\ref{sec:event_pdfs}.
Let us first introduce the variable $t$ as
\begin{equation}
  t = \ln\frac{Q^2}{\Lambda^2},
\end{equation}
with $\Lambda$ the QCD scale. We then find for the differential
\begin{equation}
  \text{d}t = \text{d}\ln Q^2 = \frac{\text{d}Q^2}{Q^2}.
\end{equation}
We can view $t$ as a kind of time in the evolution of the parton shower. The smaller $t$, and
thus the lower the scale, the further along in the shower process we are. 
In terms of the variable $t$, we can write the differential probability for a parton $i$ to branch
into any parton $j$ with momentum fraction $z$ in the following way,
\begin{equation}
  \text{d}\mathcal{P}_i = \sum_j \frac{\alpha_S}{2\pi} P_{j\leftarrow i}(z)\text{d}t\text{d}z,
\label{eq:splitting}
\end{equation}
with the different splitting functions in the collinear limit given by
\begin{align}
  P_{q\leftarrow q}(z) &= C_F \frac{1 + z^2}{1 - z}, & 
  P_{g\leftarrow q}(z) &= C_F \frac{1 + (1-z)^2}{z}, \\
  P_{g\leftarrow g}(z) &= C_A \frac{z^4 + 1 + (1-z)^4}{z(1-z)}, &
  P_{q\leftarrow g}(z) &= T_R (z^2 + (1-z)^2), 
\end{align}
where $C_F$ and $C_A$ are color factors and $T_R$ is a constant depending on the definition of
$\alpha_S$. 
There are two sets of divergencies that occur in the computation of the branching probability: when
the radiated parton becomes extremely soft, or when it becomes collinear with the original parton. 
The cases where this occurs can in fact not be resolved in any physical measurement. Two exactly
collinear partons look exactly like one parton with the same total momentum. We should thus impose
a resolution criterion. Often the chosen criterion is that the relative transverse momentum
between the two partons is larger than some cutoff scale $Q_0$. Imposing this cutoff, then results
in a finite resolvable emission probability. Because the total probability of something happening
has to be unity, we can find the probability to not have a resolvable emission as one minus the
resolvable emission probability. In this way we have avoided computing the divergent pieces, which
would have to be added to the divergent loop-correction to the hard process in order to cancel.

Since Eq.~\ref{eq:splitting} is a completely general expression that does not depend on the hard
process, we can iterate it, using it on a parton resulting from the hard process to generate
one branching and then treating the new final state as the hard process, generating another
splitting from it, and so on. In what follows we will discuss how this shall be done in practice.
%utiziling the trick of unit total probability to avoid computing divergent pieces, and thus
%implicitly including high-order corrections into our results. 

The integral of the branching probability over all allowed $z$ values, according to the
particular resolution criterion imposed, and for a given $t$ value, is defined as
\begin{equation}
  \mathcal{I}_{j\leftarrow i}(t) = \int dz \frac{\alpha_S}{2\pi} P_{j\leftarrow i}(z)
\end{equation}
The naive probability that a resolved branching occurs during a small range of $t$ values, $\delta
t$, is given by 
\begin{equation}
 \sum_j \mathcal{I}_{j\leftarrow i}(t) \delta t,
\end{equation}
where we did not take into account anything which could have happened during the parton shower,
before that time. The probability for no resolved emission to occur is then simply given by $1 -
\sum_j \mathcal{I}_{j\leftarrow i}(t) \delta t$. 
If the evolution of parton $i$ starts at $t_{\text{max}}$, then the probability that the parton
has not yet branched later in the shower, when $t < t_{\text{max}}$, is given by
the product of the probabilities that it did not branch in any of the small intervals $\delta t$
between $t$ and $t_{\text{max}}$. In other words, letting $\delta t \rightarrow 0$, the no-branching
probability at time $t$, given starting point $t_{\text{max}}$, exponentiates, and is given by
\begin{equation}
  \mathcal{P}_{\text{no-branching}}(t_{\text{max}},t) = \exp 
  \left\{ - \int_t^{t_{\text{max}}} dt' \sum_j \mathcal{I}_{j\leftarrow i}(t') \right\} .
\end{equation}
The actual differential probability that the first resolved branching of parton $i$ occurs at `time'
$t$, which is the actual question we wish to answer, is thus given by
\begin{align}
  \frac{\text{d}\mathcal{P}_i}{\text{d}t} &= -
\frac{\mathcal{P}_{\text{no-branching}}(t_{\text{max}},t)}{\text{d}t} \\
 &= \left( \sum_j \mathcal{I}_{j\leftarrow i}(t)\right) \exp \left\{ - \int_t^{t_{\text{max}}} dt'
\sum_j \mathcal{I}_{j\leftarrow i}(t) \right\}, \label{eq:prob_first_branch}
\end{align}
where the first factor in Eq.~\ref{eq:prob_first_branch} is the naive probability mentioned above,
and the second term is an exponential suppression, similar to that found in the formula for
radioactive decay, to account for the fact that if a parton has already branched at $t'$, it can no
longer branch at $t$. 
This exponential factor, the probability to not branch above a certain scale, here
contained in the variable $t$, is called the Sudakov form factor, $\Delta_i(t_max,t)$. 

Implementing this in a Monte Carlo program is conceptually straightforward. 
First, a random number $r$ is sampled uniformly between 0 and 1. Then the value for $t$ such that 
$\Delta_i(t_max,t) = r$ is determined. If the solution is above the cutoff $t_0$, corresponding to
the resolution $Q_0$, then a resolvable branching is generated with scale $t$, otherwise the shower
evolution is terminated. In case a resolvable branching is to be generated, a $z$ value is chosen
according to the splitting functions $P_{j\leftarrow i}(z)$, and then the algorithm is started
again. 
Of course, in practice one has to take into account several complications. 
Different approaches exist for deciding what the initial scale $t_{\text{max}}$ should be.
This scale has to match the hard interaction, and could thus be the largest virtuality in the hard
scatter, but could also be the centre-of-mass energy. 
The Sudakov form factor is also not necessarily easily invertible analytically, which can dealt
with by using the so-called veto-algorithm. 
Apart from final state
showers, such as explained here, the initial state also undergoes showering. There it is important
to properly match the parton shower with the PDF treatment, as well as ensure on-shell partons that
take part in the hard interaction.
For all details on how this is fully implemented in the \textsc{Pythia} shower routine, I refer to
the manual~\cite{Sjostrand:2006za}. 

At the end of the parton shower procedure, we end up with many more partons than we had directly
after the hard interaction, all which should be described at the low scale, via non-perturbative
models. The most widely used hadronization model will be discussed in
Section~\ref{sec:event_hadronization}. 

\subsection{Jet matching}

% motivation (double counting)
% MLM matching 
% xqcut, qcut

% Parton-shower Monte Carlo programs do a good job of describing most of the features of common
% events,
% including the hadron-level detail that is essential for the correct simulation of detector effects
% on event
% reconstruction. Another nice feature of theirs is that events have equal weight, just as with real
% data.
% A drawback of parton-shower Monte Carlos is that, because they rely on the soft and collinear
% approximation, they do not necessarily generate the correct pattern of hard large-angle radiation.
% This
% can be important, e.g., if you’re simulating backgrounds to new-physics processes, for which often
% the
% rare, hard multi-jet configurations are of most interest. In contrast, fixed-order programs do
% predict these
% configurations correctly

% As we have discussed previously in Section 3.5, parton showers provide an excellent
% description in regions which are dominated by soft and collinear gluon emission. On the
% other hand, matrix element calculations provide a good description of processes where
% the partons are energetic and widely separated and, in addition, include the effects of
% interference between amplitudes with the same external partons. But, on the other
% hand, the matrix element calculations do not take into account the interference effects
% in soft and collinear gluon emissions which cannot be resolved, and which lead to a
% Sudakov suppression of such emissions.
% Clearly, a description of a hard interaction which combines the two types of
% calculations would be preferable. For this combination to take place, there first needs
% to be a universal formalism that allows the matrix element calculation to “talk” to the
% parton shower Monte Carlo. Such a universal formalism was crafted during the Les
% Houches Workshop on Collider Physics in 2001 and the resulting “Les Houches Accord”
% is in common use [90]. The accord specifies an interface between the matrix element
% and the parton shower program which provides information on the parton 4-vectors, the
% mother-daughter relationships, and the spin/helicities and colour flow. It also points to
% intermediate particles whose mass should be preserved in the parton showering. All of
% the details are invisible to the casual user and are intended for the matrix element/parton
% shower authors.
% Some care must be taken however, as a straight addition of the two techniques
% would lead to double-counting in kinematic regions where the two calculations overlap.
% There have been many examples where matrix element information has been used to
% correct the first or the hardest emission in a parton shower. There are also more general
% techniques that allow matrix element calculations and parton showers to each be used
% in kinematic regions where they provide the best description of the event properties and
% that avoid double-counting. One such technique is termed CKKW [91]

% With the CKKW technique, the matrix element description is used to describe
% parton branchings at large angle and/or energy, while the parton shower description is
% used for the smaller angle, lower energy emissions. The phase space for parton emission is
% thus divided into two regions, matrix element dominated and parton shower dominated,
% using a resolution parameter dini. The argument of αS at all of the vertices is chosen
% to be equal to the resolution parameter di at which the branching has taken place and
% Sudakov form factors are inserted on all of the quark and gluon lines to represent the
% lack of any emissions with a scale larger than dini between vertices. The di represent
% a virtuality or energy scale. Parton showering is used to produce additional emissions
% at scales less than dini . For a typical matching scale, approximately 10% of the n-jet
% cross section is produced by parton showering from the n-1 parton matrix element; the
% rest arises mostly from the n-parton matrix element. A schematic representation of the
% CKKW scheme is shown in Figure 26 for the case of W + jets production at a hadron-
% hadron collider. A description of a W + 2 jet event in the NLO formalism is also shown
% for comparison.
% The CKKW procedure provides a matching between the matrix element and parton
% shower that should be correct to the next-to-leading-logarithm (NLL) level. There are,
% however, a number of choices that must be made in the matching procedure that do
% not formally affect the logarithmic behaviour but do affect the numerical predictions,
% on the order of 20–30%. The CKKW procedure gives the right amount of radiation but
% tends to put some of it in the wrong place with the wrong colour flow. Variations that
% result from these choices must be considered as part of the systematic error inherent in
% the CKKW process. This will be discussed further in Section 5.
% For the CKKW formalism to work, matrix element information must in principle
% be available for any number n of partons in the final state. Practically speaking, having
% information available for n up to 4 is sufficient for the description of most events at the
% Tevatron or LHC. The CKKW formalism is implemented in the parton shower Monte
% Carlo SHERPA [92] and has also been used for event generation at the Tevatron and
% LHC using the Mrenna-Richardson formalism [94]. An approximate version of CKKW
% matching (the “MLM approach” + ) is available in ALPGEN 2.0 [18]

\subsection{Hadronization \label{sec:event_hadronization}}


% Real events consist not of partons but of hadrons. Since we have no idea how to calculate the
% transition between partons and hadrons, Monte Carlo event generators resort to ‘hadronization’
% models.
% One widely-used model involves stretching a colour ‘string’ across quarks and gluons, and breaking
% it up into hadrons [89, 90]. For a discussion of the implementation of this ‘Lund’ model in the MC
% program PYTHIA, with further improvements and extensions, Ref. [86] and references therein provide
% many details. Another model breaks each gluon into a qq¯ pair and then groups quarks and anti-quarks
% into colourless ‘clusters’, which then give the hadrons. This cluster type hadronization is
% implemented in
% the HERWIG event generator [87,88,91] and recent versions of SHERPA. Both approaches are illustrated
% in Fig. 26
% Hadronization models involve a number of ‘non-perturbative’ parameters. The parton-shower
% itself involves the non-perturbative cutoff Q0. These different parameters are usually tuned to data
% from
% the LEP experiments. The quality of the description of the data that results is illustrated in Fig.
% 27.


