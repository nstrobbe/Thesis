\chapter{Event generation, simulation and reconstruction \label{chap:event_generation}}

\section{What is an ``event''? \label{sec:event}}

% add picture of CMS event display
% add picture of hadronization etc for event

\section{Event generation \label{sec:event_generation}}

% add info on Madgraph, pythia, jet matching etc

\section{Event simulation \label{sec:event_simulation}}

\subsection{CMS Full Simulation using Geant4 \label{subsec:fullsim}}

% add some info on how this is done, particle interaction with matter, ionization, 
% how it deals with long lived particles?

\subsection{CMS Fast Simulation \label{subsec:fastsim}}

% explain what approximations are made, why it is necessery, especially in SUSY searches

\section{Event reconstruction \label{sec:event_reconstruction}}

% Add subsection on each object with all the object definitions we use

We select events with at least one interaction vertex, associated with at least 4 charged-particle
tracks, that lies within 24~cm of the origin of the CMS coordinate system along the beam direction
and 2~cm from the origin in the plane transverse to the beam. Owing to the high luminosity of the
LHC, hard scattering events are typically accompanied by many additional events arising from the
multiple proton-proton interactions that occur when the proton bunches cross.  The overlapping of
events is referred to as pileup.  The primary vertex is identified as the vertex with the highest
value of the $\sum \pt^2$ of the associated tracks.  Detector- and beam-related cleaning algorithms
are used to remove events with detector noise that mimic events with high energy and large imbalance
in transverse momentum.  

\subsection{Particle Flow technique}

CMS reconstructs events using the particle flow (PF) method~\cite{PF}, which reconstructs particles
(PF candidates) by combining information from the inner tracker, the calorimeters, and the muon
system.  Each PF candidate is a ssigned to one of five object categories: muons, electrons, photons,
charged hadrons, and neutral hadrons.  Contamination from pileup events is reduced by discarding
charged PF candidates that are incompatible with having originated from the primary
vertex~\cite{CMS-PAS-JME-14-001}.   The average pileup energy due to neutral hadrons is computed
event-by-event and subtracted from the energy when computing lepton isolation and jet energy.  The
energy subtracted is  the average pileup energy per unit area (in $\Delta\eta \times \Delta\phi$)
times the jet area~\cite{Fastjet1, Fastjet2}.

\subsection{Object identification}

Jets are clustered with \textsc{FastJet 3.0.1}~\cite{Cacciari:2011ma} using the anti-$k_\textrm{T}$
algorithm~\cite{antikt} with size parameter $\Delta R=0.5$ (AK5).  Jet corrections are applied as a
function of jet $\pt$ and $\eta$ to account for the residual effects of non-uniform detector
response.  
The  jet energies are corrected so that, on average, they match those of simulated particle-level
jets. After correction, jets are required to have $\pt > 30$~\GeV and $|\eta| < 2.4$.  We use the
combined secondary vertex algorithm~\cite{btag7TeV,btag8TeV} to identify jets arising from $\cPqb$
quarks. The medium tagging condition, which yields a $\cPqb$  jet misidentification rate of
$\sim$1\% and a typical efficiency of $\sim$70\%, is used to select $\cPqb$ jets. The loose tagging
condition, with a misidentification rate of $\sim$10\% and efficiency of $\sim$85\%, is used to veto
$\cPqb$ jets.  

Missing transverse energy, which is used in the calculation of the razor variable $\mr$, is 
defined to be the negative sum of the transverse momenta of all the particle flow objects in an
event.  Loosely identified and isolated electrons with $\pt > 5$~\GeV and $|\eta| < 2.5$ and muons
with $\pt > 5$\GeV and $|\eta| < 2.4$ are used both to suppress backgrounds in our signal region and
in the definition of the control regions.  A tight definition of isolated leptons (electrons with
$\pt > 10$~\GeV and $|\eta| < 2.5$ and muons with $\pt > 10$~\GeV and $|\eta| < 2.4$) defines a
control region enriched in $\cPZ \rightarrow \ell \ell $ events, from which we estimate the
systematic uncertainty in the predicted number of $\cPZ \rightarrow \nu \nu$ events in the signal
region. Any electron candidates with $1.44 < |\eta| < 1.57$ are rejected since the transition region
between barrel and endcap calorimeters is less well-instrumented.
In order to suppress the decays of taus and other leptons that fail the loose selection, events that
have isolated tracks with $\pt > 10$\GeV and track-primary vertex distance along the beam direction
$dz < 0.05$ are rejected.

