%%%%%%%%%%%%%%%%%%%%%%%%%%%%
%% Event generation 
%%%%%%%%%%%%%%%%%%%%%%%%%%%%

% add info on Madgraph, pythia, jet matching etc

\cite{Campbell:2006wx}

\subsection{Matrix element generators \label{sec:event_matrix_element_generators}}

% explain basics of how madgraph works internally 
% narrow width approx?

% From P. Skands:
% Using PDFs extracted using higher-order matrix elements in lower-order calculations, as, e.g.,
% when using NLO PDFs as input to an LO calculation. In principle, the higher-order PDFs are
% better constrained and the difference between, e.g., an NLO and an LO set should formally be
% beyond LO precision, so that one might be tempted to simply use the highest-order available
% PDFs for any calculation. However, as described in section 2.4, it is often possible to partly
% absorb
% higher-order terms into lower-order coefficients. In the context of PDFs, the fit parameters
% of lower-order PDFs will effectively attempt to “compensate” for missing higher-order contributions
% in the matrix elements. To the extent those higher-order contributions are universal, this
% is both desirable and self-consistent. However, this will only give an improvement when used
% with matrix elements at the same order as those used to extract the PDFs. It is therefore quite
% possible that NLO PDFs used in conjunction with LO matrix elements give a worse agreement
% with data than LO PDFs do


% There is a wide range of programs available, most notably ALPGEN [17, 18],
% the COMPHEP package [19, 20] and MADGRAPH [21, 22]. All of these programs
% implement the calculation of the diagrams numerically and provide a suitable phase
% space over which they can be integrated. ALPGEN uses an approach which is not based
% on a traditional Feynman diagram evaluation [23], whereas the other two programs rely
% on more conventional methods such as the helicity amplitudes evaluation of HELAS [24]
% in MADGRAPH.
% Although in principle these programs can be used to calculate any tree-level
% prediction, in practice the complexity of the process that may be studied is limited
% by the number of particles that is produced in the final state. This is largely due to the
% factorial growth in the number of Feynman diagrams that must be calculated. Even in
% approaches which do not rely directly on the Feynman diagrams, the growth is still as a
% power of the number of particles. For processes which involve a large number of quarks
% and gluons, as is the case when attempting to describe a multi-jet final state at a hadron
% collider such as the Tevatron or the LHC, an additional concern is the calculation of
% colour matrices which appear as coefficients in the amplitudes [25].


\subsection{Parton shower and hadronization}

% discuss underlying event tunes etc
% discuss string model for hadronization


% 
% By the use of the parton showering process, a few partons produced in a hard
% interaction at a high energy scale can be related to partons at an energy scale close to
% ΛQCD . At this lower energy scale, a universal non-perturbative model can then be used
% to provide the transition from partons to the hadrons that are observed experimentally.
% This is possible because the parton showering allows for the evolution, using the DGLAP
% formalism, of the parton fragmentation function. The solution of this DGLAP evolution
% equation can be rewritten with the help of the Sudakov form factor, which indicates the
% probability of evolving from a higher scale to a lower scale without the emission of a
% gluon greater than a given value. For the case of parton showers from the initial state,
% the evolution proceeds backwards from the hard scale of the process to the cutoff scale,
% with the Sudakov form factors being weighted by the parton distribution functions at
% the relevant scales.

% In the parton showering process, successive values of an evolution variable t, a
% momentum fraction z and an azimuthal angle φ are generated, along with the flavours
% of the partons emitted during the showering. The evolution variable t can be the
% virtuality of the parent parton (as in PYTHIA versions 6.2 and earlier and in SHERPA),
% E 2 (1 − cos θ), where E is the energy of the parent parton and θ is the opening angle
% between the two partons (as in HERWIG) , or the square of the relative transverse
% momentum of the two partons in the splitting (as in PYTHIA 6.3). The HERWIG
% evolution variable has angular ordering built in, angular ordering is implicit in the
% PYTHIA 6.3 [85] evolution variable, and angular ordering has to be imposed after the
% fact for the PYTHIA 6.2 evolution variable. Angular ordering represents an attempt to
% simulate more precisely those higher order contributions that are enhanced due to soft
% gluon emission (colour coherence). Fixed order calculations explicitly account for colour
% coherence, while parton shower Monte Carlos that include colour flow information model
% it only approximately.
% Note that with parton showering, we in principle introduce two new scales, one for
% initial state parton showering and one for the shower in the final state. In the PYTHIA
% Monte Carlo, the scale used is most often related to the maximum virtuality in the
% hard scattering, although a larger ad hoc scale, such as the total centre-of-mass energy,
% can also be chosen by the user. The HERWIG showering scale is determined by the
% specific colour flow in the hard process and is related to the invariant mass of the colour
% connected partons.
% the Sudakov form
% factor gives the probability for a parton to evolve from a harder scale to a softer scale
% without emitting a parton harder than some resolution scale, either in the initial state
% or in the final state. Sudakov form factors form the basis for both parton showering and
% resummation
% A Sudakov
% form factor will depend on: (1) the parton type (quark or gluon), (2) the momentum
% fraction x of the initial state parton, (3) the hard and cutoff scales for the process and
% (4) the resolution scale for the emission.


% 
% The programs we’ve discussed so far, known as ‘Matrix Element Monte Carlos’ provide a powerful
% combination of accuracy and flexibility as long as you want to calculate IR and collinear safe
% observables
% (jets, W’s, Z’s, but not pions, kaons, etc.), don’t mind dealing with wildly fluctuating positive
% and
% negative weights, and don’t need to study regions of phase space that involve disparate physical
% scales.
% All these defects are essentially related to the presence of soft and collinear divergences. Yet we
% know that real life does not diverge. So it is natural to wonder whether we can reinterpret the
% divergences
% of perturbation theory physically. It turns out that the right kind of question to ask is “what is
% the
% probability of not radiating a gluon above some (transverse momentum) scale kt”. Starting from a qq¯
% system, using the results of Section 2, we know that to O (αs) the answer in the soft and collinear
% limit
% goes as(64)
% It so happens that in the soft and collinear limit, this result is easy to work out not just at
% first order, but
% at all orders, giving simply the exponential of the first order result.
% .
% Whereas Eq. (64) had a ‘bare’ infinity if one took kt → 0, Eq. (65) is simply bounded to be between
% 0
% and 1.
% The quantity ∆(kt
% , Q) is known as a Sudakov form factor. We’ve been very approximate in the
% way we’ve calculated it, neglecting for example the running of the coupling (αs should be placed
% inside
% the integral and evaluated at the scale Eθ) and the treatment of hard collinear radiation (the dE/E
% integral should be replaced with the full collinear splitting function), but these are just
% technical details.
% The importance of the Sudakov form factor is that it allows us to easily calculate the distribution
% in
% transverse momentum kt1 of the gluon with largest transverse momentum in an event:
% dP
% dkt1
% =
% d
% dkt1
% ∆(kt1, Q). (66)
% This distribution is easy to generate by Monte Carlo methods: take a random number r from a
% distribution
% that’s uniform in the range 0 < r < 1 and find the kt1 that solves ∆(kt1, Q) = r. Given kt1 we also
% need to generate the energy for the gluon, but that’s trivial. If we started from a qq¯ system (with
% some
% randomly generated orientation), then this gives us a qqg¯ system. As a next step one can work out
% the
% Sudakov form factor in the soft/collinear limit for there to be no emission from the qqg¯ system as
% a whole
% above some scale kt2 (< kt1) and use this to generate a second gluon. The procedure is then repeated
% over and over again until you find that the next gluon you would generate is below some
% non-perturbative
% cutoff scale Q0, at which point you stop. This gives you one ‘parton shower’ event.
% This is essentially the procedure that’s present in the shower of PYTHIA 8 [82] and the pt ordered
% option of PYTHIA 6.4




\subsection{Jet matching}

% motivation (double counting)
% MLM matching 
% xqcut, qcut

% Parton-shower Monte Carlo programs do a good job of describing most of the features of common
% events,
% including the hadron-level detail that is essential for the correct simulation of detector effects
% on event
% reconstruction. Another nice feature of theirs is that events have equal weight, just as with real
% data.
% A drawback of parton-shower Monte Carlos is that, because they rely on the soft and collinear
% approximation, they do not necessarily generate the correct pattern of hard large-angle radiation.
% This
% can be important, e.g., if you’re simulating backgrounds to new-physics processes, for which often
% the
% rare, hard multi-jet configurations are of most interest. In contrast, fixed-order programs do
% predict these
% configurations correctly

% As we have discussed previously in Section 3.5, parton showers provide an excellent
% description in regions which are dominated by soft and collinear gluon emission. On the
% other hand, matrix element calculations provide a good description of processes where
% the partons are energetic and widely separated and, in addition, include the effects of
% interference between amplitudes with the same external partons. But, on the other
% hand, the matrix element calculations do not take into account the interference effects
% in soft and collinear gluon emissions which cannot be resolved, and which lead to a
% Sudakov suppression of such emissions.
% Clearly, a description of a hard interaction which combines the two types of
% calculations would be preferable. For this combination to take place, there first needs
% to be a universal formalism that allows the matrix element calculation to “talk” to the
% parton shower Monte Carlo. Such a universal formalism was crafted during the Les
% Houches Workshop on Collider Physics in 2001 and the resulting “Les Houches Accord”
% is in common use [90]. The accord specifies an interface between the matrix element
% and the parton shower program which provides information on the parton 4-vectors, the
% mother-daughter relationships, and the spin/helicities and colour flow. It also points to
% intermediate particles whose mass should be preserved in the parton showering. All of
% the details are invisible to the casual user and are intended for the matrix element/parton
% shower authors.
% Some care must be taken however, as a straight addition of the two techniques
% would lead to double-counting in kinematic regions where the two calculations overlap.
% There have been many examples where matrix element information has been used to
% correct the first or the hardest emission in a parton shower. There are also more general
% techniques that allow matrix element calculations and parton showers to each be used
% in kinematic regions where they provide the best description of the event properties and
% that avoid double-counting. One such technique is termed CKKW [91]

% With the CKKW technique, the matrix element description is used to describe
% parton branchings at large angle and/or energy, while the parton shower description is
% used for the smaller angle, lower energy emissions. The phase space for parton emission is
% thus divided into two regions, matrix element dominated and parton shower dominated,
% using a resolution parameter dini. The argument of αS at all of the vertices is chosen
% to be equal to the resolution parameter di at which the branching has taken place and
% Sudakov form factors are inserted on all of the quark and gluon lines to represent the
% lack of any emissions with a scale larger than dini between vertices. The di represent
% a virtuality or energy scale. Parton showering is used to produce additional emissions
% at scales less than dini . For a typical matching scale, approximately 10% of the n-jet
% cross section is produced by parton showering from the n-1 parton matrix element; the
% rest arises mostly from the n-parton matrix element. A schematic representation of the
% CKKW scheme is shown in Figure 26 for the case of W + jets production at a hadron-
% hadron collider. A description of a W + 2 jet event in the NLO formalism is also shown
% for comparison.
% The CKKW procedure provides a matching between the matrix element and parton
% shower that should be correct to the next-to-leading-logarithm (NLL) level. There are,
% however, a number of choices that must be made in the matching procedure that do
% not formally affect the logarithmic behaviour but do affect the numerical predictions,
% on the order of 20–30%. The CKKW procedure gives the right amount of radiation but
% tends to put some of it in the wrong place with the wrong colour flow. Variations that
% result from these choices must be considered as part of the systematic error inherent in
% the CKKW process. This will be discussed further in Section 5.
% For the CKKW formalism to work, matrix element information must in principle
% be available for any number n of partons in the final state. Practically speaking, having
% information available for n up to 4 is sufficient for the description of most events at the
% Tevatron or LHC. The CKKW formalism is implemented in the parton shower Monte
% Carlo SHERPA [92] and has also been used for event generation at the Tevatron and
% LHC using the Mrenna-Richardson formalism [94]. An approximate version of CKKW
% matching (the “MLM approach” + ) is available in ALPGEN 2.0 [18]

\subsection{Hadronization}


% Real events consist not of partons but of hadrons. Since we have no idea how to calculate the
% transition between partons and hadrons, Monte Carlo event generators resort to ‘hadronization’
% models.
% One widely-used model involves stretching a colour ‘string’ across quarks and gluons, and breaking
% it up into hadrons [89, 90]. For a discussion of the implementation of this ‘Lund’ model in the MC
% program PYTHIA, with further improvements and extensions, Ref. [86] and references therein provide
% many details. Another model breaks each gluon into a qq¯ pair and then groups quarks and anti-quarks
% into colourless ‘clusters’, which then give the hadrons. This cluster type hadronization is
% implemented in
% the HERWIG event generator [87,88,91] and recent versions of SHERPA. Both approaches are illustrated
% in Fig. 26
% Hadronization models involve a number of ‘non-perturbative’ parameters. The parton-shower
% itself involves the non-perturbative cutoff Q0. These different parameters are usually tuned to data
% from
% the LEP experiments. The quality of the description of the data that results is illustrated in Fig.
% 27.


