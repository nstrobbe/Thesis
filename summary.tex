\chapter{Summary and discussion \label{chap:summary}}

The Standard Model of particle physics encapsulates our current knowledge of the elementary
particles and their interactions. It was developed as a quantum field theory
over the past fifty years and has been tested thoroughly by many different collider and
non-collider experiments. So far, no significant deviations have been observed, and until recently
the Higgs boson was the only particle predicted to exist that had not been found. 
When that elusive particle was finally discovered in 2012, this was the ultimate victory for the
Standard Model, and the mechanism of spontaneous electroweak symmetry breaking. 

Regardless of all its successes, the Standard Model cannot be the final theory describing the
workings of nature. There are several reasons for this, the principal one being that gravity is
as yet not included. The Standard Model also cannot explain dark matter, or dark energy. 
With the discovery of the Higgs boson, the hierarchy problem has become a pertinent issue. 

At the core of the hierarchy problem is the question why the electroweak scale is so much smaller
than the Planck scale. Radiative corrections to the Higgs boson mass are quadratically dependent on
the ultraviolet cutoff scale that is used to regulate the loop diagrams. If the Standard Model is
valid up to the point where we need a theory of quantum gravity, then this cutoff is exactly the
Planck scale. This then results in a huge correction to the Higgs boson mass. Within the Standard
Model this is technically not an issue, but it would require an extremely large amount of
finetuning between the bare Higgs mass and the correction to result in the observed Higgs boson
mass of 125\GeV. This amount of finetuning is generally viewed to be unnatural, and therefore
unwanted. 

Several models beyond the Standard Model have been developed in an attempt to address these and
other issues. The most popular extension of the Standard Model nowadays is supersymmetry.
Supersymmetry adds another symmetry to the theory, a symmetry between fermions and bosons, which
results in the introduction of a superpartners for every Standard Model particle. 
Many times before, symmetries have guided physicists to a deeper understanding of nature. 
The elegance of supersymmetry is thus definitely part of the appeal, but it is not the only nice
feature. 

Many supersymmetric models assume the conservation of R-parity, which leads to a lightest
supersymmetric particle (LSP) that is stable, and is a possible dark matter candidate. 
Supersymmetry, and in particular `natural supersymmetry', can provide a much more natural solution
to the hierarchy problem, removing most of the finetuning. For this to happen the supersymmetric
partners of the top quark and the gluon, the top squark and the gluino, must be relatively light,
less than about $1\TeV$ and $1.5\TeV$, respectively. These particles, if they exist, should thus be
accessible at the collision energies of the LHC. 


The possibility that the top squark could be light has motivated several dedicated searches by the
CMS and ATLAS collaborations for the direct production of top squark pairs. 
The sensitivity of many of these searches diminishes when the mass of the top squark approaches that
of the LSP, or when the mass difference between the top squark and the LSP is comparable to the top
quark mass. The razor boost analysis presented in this thesis aims to fill these gaps in the
sensitivity by using the naturalness argument and assuming the existence of an accessible gluino. 


In the razor boost analysis, we consider gluino pair production in which the gluino decays to a top
squark and a top quark. In the models considered, the gluino is assumed to have a mass around
1-1.5\TeV and the lighter top squark has a mass of a few hundred \GeV.
The considered top squark decay depends on the assumed mass difference between top squark and LSP.
For small mass differences, the top squark decays to a charm quark and the LSP, while for mass
differences around the top quark mass the top squark decays to a top quark and the LSP. 
The models used are simplified models, meaning that all other possible supersymmetric particles
have a very large mass and are thus entirely decoupled from the gluino, top squark and LSP. 
The razor boost analysis described in this thesis is the first analysis within CMS to explicitly
probe the gluino-mediated production of top squarks decaying to a charm quark and the LSP, and
will thus provide new information on natural supersymmetry. 


The razor boost analysis uses the razor kinematic variables $\mr$ and $\rsq$ as main search tool.
These variables can discriminate processes with new heavy particles and missing transverse
momentum from Standard Model processes. The characteristic mass scale of the new physics is
estimated in two ways, using longitudinal information ($\mr$) and transverse information ($\rsq$). 
Signal events are expected to appear as a peak over an exponentially falling background. 
The search for new physics is performed in 25 search bins across the high $\mr$ - high $\rsq$
region. 

Owing to the significant mass gap presumed to exist between the gluino and the top squark, the
top quark from the gluino to top squark decay will receive a large boost. 
When the boost is large enough, around 700\GeV, the top quark decay products will merge. Since this
boost is hard to reach with proton-proton collisions at 8\TeV, we opted to consider boosted $\W$
bosons instead, for which merged decay products arise when the boost is around 320\GeV. 
The hadronically decaying boosted $\W$ bosons are identified using jet substructure techniques, in
particular jet pruning and N-subjettiness. Pruned jets allow the proper reconstruction of the
jet mass, which should be around the $\W$ boson mass for merged jets from a $\W$ boson. 
N-subjettiness is a very useful observable to check the compatibility of a jet having N subjets.
For the razor boost analysis we require that the boosted $\W$ boson candidates should be consistent
with having two subjets. 


The signal region selection requires the presence of at least one boosted $\W$ boson candidate and
one jet tagged as originating from a $\cPqb$ quark. We also require the presence of three jets, and
veto events with leptons. 
The background from Standard Model processes in the 25 signal region bins is estimated using
observations in data control regions and scale factors, calculated from simulated data, that
relate the number of events in one region to that in another. 
Three control regions, $Q$, $W$ and $T$, are defined to select high-purity samples of multijet,
$\W (\rightarrow \ell \nu)$+jets and $t\bar{t}$ processes, respectively. 
The scale factors are global scale factors, meaning that they are integrated over the full
($\mr$,$\rsq$) space. The reason for this is the lack of simulated events at high $\mr$ and high
$\rsq$. 

The background estimation method uses a likelihood-based approach. 
The likelihood for each search bin is modelled as a Poisson. The expected background
components in the signal bins, corresponding to the $t\bar{t}$, $\W (\rightarrow \ell \nu)$+jets and
multijet processes, are constrained using a prior distribution, which is the translation of the
relationships between signal and control regions, and incorporates all statistical and systematic
uncertainties. 
The systematic uncertainties are sampled simultaneously, which ensures that any correlations are
taken into account automatically. The razor boost analysis is the first SUSY analysis within CMS
that uses this complete treatment of the systematic uncertainties. 

The background prediction result for each signal bin is derived from the prior distribution. 
The results are found to be in agreement with the observation in data. We can thus conclude that
there is no evidence of new physics in the phase space that was probed by the razor boost analysis. 
These results can also be used to constrain the allowed parameter space for more specific new
physics scenarios. As mentioned earlier, we consider two simplified models of gluino pair
production, with the gluino decaying to a top squark and a top quark. For the case where the top
squark and the LSP have a small mass difference ($\leq 80\GeV$), the top squark is assumed to decay
to a charm quark and the LSP. In this model, we can exclude gluinos up to about 1\TeV for LSP masses
up to about 500\GeV. Or equivalently, we can exclude top squarks decaying to charm and LSP with
masses up to about 500\GeV, provided that there is a gluino with mass less than about 1\TeV. 
Similarly, for the model in which the top squark decays to a top quark and the LSP, with a mass
difference of 175\GeV between them, we have excluded top squarks with a mass up to about 450\GeV
if the gluino mass is less than 850\GeV.


The weakest point of the razor boost analysis is the use of global scale factors to translate
between signal and control regions, as there could be shape differences that are not accounted for.
A set of closure tests was performed to check whether the global scale factors were adequate for
the precision achieved in the analysis. We found that an extra uncertainty of 33\% needed to be
applied to the multijet prediction in order to cover for possible shape differences. 
Once applied, there was no further evidence of a shape difference within uncertainties. 
The decision to use global scale factors was driven by the lack of statistical precision
in the simulated samples. This can thus be improved upon in the future by having more simulated
events in the boosted regime. 


The razor boost analysis features two techniques that set it apart from other SUSY analyses: the
use of boosted $\W$ boson tagging, and the statistical treatment of the systematic uncertainties. 
Both of these techniques will be very valuable for analyses with the 13\TeV dataset to be
collected in 2015. 


Because of the increase in centre-of-mass energy from 8 to 13\TeV, we expect the produced particles
to be more boosted in general. The use of jet substructure techniques will thus be a powerful
addition to many analyses. For the razor boost analysis the raised energy will mean that the
signal efficiency will also become larger, boosting the reach of the analysis beyond what would be
expected from the increased cross section alone. 

For the razor boost analysis at 13\TeV, adding extra channels could help increase the sensitivity
to wider range of models. The first option, which I believe will be very beneficial, is to include
boosted top tagging. Having a well-thought out strategy that combines final states with both boosted
$\W$ boson and boost top quarks will ensure that a wide range of mass spectra can be probed. The
boosted top signal region will be optimal for very large mass splittings, whereas the boosted $\W$
region will be more targeted towards medium mass splittings. 
Secondly, the leptonic channels could be added to the analysis. Boosted $\W$ bosons that decay
leptonically, will feature leptons with very large transverse momentum. These leptons can be very
close to the $\cPqb$ jet from the top quark decay, and could thus fail the isolation requirements.
A dedicated treatment of these leptons is thus necessary. 

















