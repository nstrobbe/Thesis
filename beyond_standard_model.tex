\chapter{The need for physics beyond the Standard Model \label{chap:beyond_standard_model}}

Although the Standard Model has succeeded in predicting and explaining a plethora of physics
processes, it cannot be the ultimate theory describing Nature. In
Section~\ref{sec:missing_elements}, I will list some open questions for which the Standard Model
does not provide an answer. A very pertinent question regarding the Higgs boson mass is discussed
in Section~\ref{sec:hierarchy_problem}.
Any new theory that attempts to solve these issues should revert back to
the Standard Model for the energy domains that have been so thoroughly tested over the past
decades. In this respect we can view the Standard Model as an effective low-energy theory, just
like Newton's laws of motion follow from a low-energy approximation of special relativity. 
Most of the proposed new physics theories are, therefore, extensions of the Standard Model. 
A number of possibilities are introduced in Section~\ref{sec:extensions_standard_model}.

\section{Open questions \label{sec:missing_elements}}

The open questions can roughly be divided in two categories. The first category comprises the
experimental observations that are not covered at all by the SM, while the second category concerns
characteristics of the SM for which we have no fundamental explanation. 
I provide a non-exhaustive list below.

\paragraph{Category 1}
\begin{itemize}
  \item \textbf{Gravity} is not included in the Standard Model. The main reason for this is that
a satisfactory microscopic theory of gravity has not been formulated yet. Some advances have been
made, \eg supergravity
theories~\cite{VanNieuwenhuizen:1981ae,freedman2012supergravity,Nastase:2011aa}, but it is not yet
at the point where it can be unified with the rest of the Standard Model. 

  \item The latest Planck results~\cite{Planck:2015xua} state that about 26\% of the energy budget
of the universe is covered by \textbf{dark matter}, compared to less than about 5\% for the ordinary
matter that is described by the Standard Model. Currently, the Standard Model does not contain any
particle that could be a dark matter candidate. Such a particle should be stable, neutral, weakly
interacting and have a reasonably large mass. 

  \item In addition to the presence of dark matter, we have very strong indications that
the remaining 69\% of the universe is \textbf{dark energy}, and drives the expansion of the
universe. The Planck results also indicate that this dark energy is consistent with the assumption
of a cosmological constant. The SM again does not provide an explanation for this, in fact, any
attempt to compute this cosmological constant in terms of vacuum energy leads to a mismatch of
around 100 orders of magnitude. 

  \item The universe is almost entirely made up of matter, rather than antimatter. Assuming that
equal amounts were created in the Big Bang, the SM cannot explain this \textbf{matter-antimatter
asymmetry}. As postulated by Sakharov~\cite{Sakharov}, there are three necessary conditions for a
baryon asymmetry to exist: charge (C) and charge-parity (CP) violation; the absence of thermal
equilibrium; and at least one baryon number violating process. Within the SM there is a small amount
of CP violation, \eg in the decay of the $K^0$ meson. However, even if the other two conditions were
satisfied, there would not be enough CP violation to explain the observed matter-antimatter
discrepancy. 
  
  \item Neutrinos of different flavour have been observed to
oscillate~\cite{Abe:2008aa,Abe:2014ugx,Agafonova:2014ptn}. These \textbf{neutrino oscillations} can
only occur if at least two of the three neutrino types have mass. The neutrino mass eigenstates
($\nu_1, \nu_2, \nu_3$) are then superpositions of the flavour eigenstates ($\nu_e, \nu_\mu,
\nu_\tau$).
No measurements of the absolute masses have been made so far, but the squared mass differences are
known. 
In the Standard Model the neutrinos are massless. Adding a mass to the neutrinos can be
done~\cite{Klinkhamer:2011aa}, but the question remains whether they are normal Dirac fermions, or
Majorana fermions (\ie their own antiparticle). 
\end{itemize}

\paragraph{Category 2}

\begin{itemize}
  \item Why is the Higgs boson mass only 125\GeV? Radiative corrections would automatically drive
this mass up to very large scales. This is often referred to as the hierarchy problem.
As we will see later, this issue forms part of the motivation for the search presented in this
thesis, and will thus be covered in some more detail in Section~\ref{sec:hierarchy_problem}. 

  \item Why are there three families of fermions? Careful study of the lineshape of the $\cPZ$
boson has shown that there is no fourth family with light neutrinos~\cite{Decamp:1989fr}. Could
there be an extra family with heavier neutrinos?

  \item Why do the fermions have the masses, \ie couplings to the Higgs boson, they have? And why
is there such a wide range of masses, \ie from $0.511\MeV$ for the electron up to 173\GeV for the
top quark, a difference of 5 orders of magnitude. This is sometimes referred to as the fermion mass
hierarchy problem. 

  \item Are baryon and lepton number conserved? In the Standard Model these are accidental
symmetries, without an underlying reason such as a local gauge symmetry. There is thus no
compelling reason to assume that baryon and lepton number are conserved quantities. 
  
  \item Why is the $\mu^2$ parameter in the Higgs potential ($\mu^2 \Phi^\dagger \Phi + \lambda
(\Phi^\dagger \Phi)^2$) negative? Within the SM this is an assumption that is made, without
underlying motivation other than that it is needed to trigger electroweak symmetry breaking.
\end{itemize}


\section{The hierarchy problem \label{sec:hierarchy_problem}}

At the core of the hierarchy problem are the different mass scales that are present in the
universe. The Standard Model does a very good job explaining phenomena at the electroweak scale of
$\mathcal{O}(100\GeV)$. We know that when we reach the Planck scale,
$\mathcal{O}(\text{10}^{\text{19}}\GeV)$, the SM can no longer be the complete theory, as 
quantum gravity effects will then need to be included. 

Observed particle masses are a combination of the bare, tree-level mass, and all radiative
corrections from additional loop diagrams. The loop momenta are cut off at the scale where we
believe the theory to be no longer valid, in this case the Planck scale. 
For fermion masses these corrections are only logarithmically dependent on the high cutoff scale,
as they are protected by chiral symmetry. Gauge bosons are similarly protected by the local gauge 
symmetry. The Higgs boson, being a scalar, does not have any protection, and therefore the
radiative corrections introduce a quadratic dependence on the cutoff scale. 

The one-loop corrections to the Higgs boson mass arise from the diagrams shown in
Fig.~\ref{fig:oneloopdiagrams}. The fermionic loop correction arises from the Yukawa interaction
between the Higgs boson and the fermions. The relevant part of the Lagrangian (Eq.~\ref{eq:yukawa})
is given for a generic fermion as,
\begin{equation}
  \mathcal{L}_{f\bar{f}H} = - \frac{\lambda_f}{\sqrt{2}} H f \bar{f} .
\end{equation}
When one computes the fermionic one-loop diagram, we find for the correction to the Higgs boson
mass~\cite{Djouadi:2005gj}, 
\begin{equation}
  \Delta m_H^2 = N_f \frac{\lambda_f^2}{8\pi^2} \left[ - \Lambda^2 + 6 m_f^2 \log
\left(\frac{\Lambda}{m_f}\right) - 2 m_f^2 \right] + \mathcal{O}\left(\frac{1}{\Lambda^2}\right) ,
\end{equation}
where the quadratic dependence on the cutoff scale $\Lambda$ is explicitly visible. Taking this
cutoff to be the Planck scale results in a correction that is more than 30 orders of magnitude
larger than the Higgs boson mass squared itself. To still achieve an observable mass of 125\GeV, the
bare mass and the correction would thus have to cancel to an extremely high precision. The inclusion
of the vector boson and Higgs self-coupling loops do not change this overall behaviour. 
The requirement for this large amount of finetuning is viewed to be unnatural, and many models of
new physics will use the \textit{naturalness} argument in their favour. 

\begin{figure}[t]
\begin{tikzpicture}[scale=1.2]
  \node[void] (in) at (0,0)  {};
  \node[void] (v1) at (1,0) {};
  \node[void] (v2) at (2,0) {};
  \node[void] (out) at (3,0) {};
  \draw[spin0] (in) -- node [above]{$H$} (v1) ;
  \draw (1.5,0) circle (0.5);
  \node at (1.5,0.7) {$f$};
  \draw[spin0] (v2) -- node [above]{$H$} (out);
\end{tikzpicture}
\begin{tikzpicture}[scale=1.2]
  \node[void] (in) at (0,0)  {};
  \node[void] (v1) at (1,0) {};
  \node[void] (v2) at (2,0) {};
  \node[void] (out) at (3,0) {};
  \draw[massvect] (v1) arc(180:0:0.5) ;
  \node at (1.5,0.8) {$W,Z$};
  \draw[spin0] (in) -- node [above]{$H$} (v1) -- (v2) -- node [above]{$H$} (out);
\end{tikzpicture}
\begin{tikzpicture}[scale=1.2]
  \node[void] (in) at (0,0)  {};
  \node[void] (v1) at (1,0) {};
  \node[void] (v2) at (2,0) {};
  \node[void] (out) at (3,0) {};
  \draw[massvect] (1.5,0.45) circle (0.5);
  \node at (1.5,1.3) {$W,Z$};
  \draw[spin0] (in) -- node [above]{$H$} (v1) -- (v2) -- node [above]{$H$} (out);
\end{tikzpicture}
\begin{tikzpicture}[scale=1.2]
  \node[void] (in) at (0,0)  {};
  \node[void] (v1) at (1,0) {};
  \node[void] (v2) at (2,0) {};
  \node[void] (out) at (3,0) {};
  \draw[spin0] (1.5,0.5) circle (0.5);
  \node at (1.5,1.3) {$H$};
  \draw[spin0] (in) -- node [above]{$H$} (v1) -- (v2) -- node [above]{$H$} (out);
\end{tikzpicture}
\caption{One-loop quantum corrections to the Higgs boson mass. From left to right: contribution
from the Yukawa interaction; two contributions from the gauge interaction; contribution from the
Higgs self-interaction.
\label{fig:oneloopdiagrams}}
\end{figure}

A way to remove this quadratic dependence on the cutoff scale is to introduce extra particles in
the theory, with properties such that the loop behaviour is opposite to the Standard Model
particles. It is straight-forward to show that we can cancel the fermionic loops by introducing
extra scalar particles. 
Assuming that there are $N_S$ new scalar particles, with mass $m_S$, trilinear coupling $v\lambda_S$
and quadrilinear coupling $\lambda_S$, we find as additional contribution to the
one-loop correction to the Higgs mass:
\begin{equation}
  \Delta m_H^2 =  \frac{N_S\lambda_S}{16\pi^2} \left[ - \Lambda^2 + 2 m_S^2 \log
\left(\frac{\Lambda}{m_S}\right)\right] - \frac{\lambda_S^2 N_S}{16 \pi^2} v^2 \left[ -1 + 2
\log\left(\frac{\Lambda}{m_S}\right) \right] + \mathcal{O}\left(\frac{1}{\Lambda^2}\right) .
\end{equation}
By assuming $\lambda_f^2 = - \lambda_S$ and $N_S = 2 N_f$, we find upon adding both
contributions, and using Eq.~\ref{eq:fermion_masses},
\begin{equation}
  \Delta m_H^2 = \frac{\lambda_f^2 N_f}{4\pi^2} \left[ \left(m_f^2-m_S^2\right)
\log\left(\frac{\Lambda}{m_S}\right) + 3 m_f^2 \log\left(\frac{m_S}{m_f}\right) \right] .
\end{equation}
All quadratically divergent terms have vanished. Introducing scalar particles with the
appropriate couplings has thus technically solved the hierarchy and naturalness problem. 
If in addition $m_S = m_f$, then the logarithmically divergent terms vanish as well. 

The divergencies introduced by the other loop diagrams in Fig.~\ref{fig:oneloopdiagrams} can also be
resolved by the introduction of new particles, fermions in this case, that have just the right
couplings to the Higgs boson. In this way all divergent contributions to the Higgs mass will
vanish, and no large finetuning is needed. 
%As we will see shortly, supersymmetry introduces extra particles that behave exactly in this
%way. 


\section{Extensions of the Standard Model \label{sec:extensions_standard_model}}


In an attempt to address some of the afore-mentioned open questions, numerous models of
beyond-the-SM (BSM) physics have been developed. With the discovery of the Higgs boson, some of
these are now ruled out~\cite{Cheng:2007bu}. Examples are the Higgsless models such as the most
basic incarnation of technicolour, or the models which predict a very large Higgs boson mass, such
as certain Composite Higgs models where the Higgs mass would be related to some new strong dynamics
at high scales. 
Nevertheless, several viable models still remain. A subset of these are presented in the following
sections.

\subsection{Supersymmetry \label{sec:supersymmetry}}

Supersymmetric models impose a new symmetry, supersymmetry (SUSY), that relates fermions and
bosons. Given that the razor boost analysis was developed with supersymmetry in mind and provides
interpretations in a SUSY context, I will discuss supersymmetric models in a separate chapter
(Chapter~\ref{chap:supersymmetry}) and only provide a brief motivation here. 

One of the nice features of SUSY is that it provides a solution to the hierarchy problem. 
The reason is that each known SM particle comes with a supersymmetric partner that differs in
spin by 1/2, and has the same mass. The structure of the couplings is also exactly as we suggested
in Section~\ref{sec:hierarchy_problem}, resulting in the removal of the quadratically divergent
terms in the mass correction. In practice we have not observed any such partner particles. Hence,
their masses cannot be equal to the corresponding SM particles, and SUSY must be broken somehow. 
If the breaking mechanism is such that the quadratic divergencies still cancel, but not necessarily
the logarithmic ones, then the hierarchy problem can still be solved. 

SUSY also triggers electroweak symmetry breaking in a dynamic way. There is thus no need to
explicitly assume a negative $\mu^2$ in the Higgs potential. 
A very generic SUSY Lagrangian would allow interactions leading to proton decay. To avoid this,
R-parity is introduced. If R-parity is conserved, then the lightest supersymmetric particle (LSP) is
stable, as it cannot decay without violating R-parity. This LSP can be a dark matter candidate, as
it would be heavy and only weakly interacting. 

For all these reasons, and more, SUSY is nowadays by far the most popular extension of the Standard
Model. Hopes are high to find hints of the existence of supersymmetric particles during the
upcoming Run 2 of the LHC. 

\subsection{Little Higgs scenarios \label{sec:little_higgs}}

In Little Higgs theories~\cite{Cheng:2007bu,Reuter:2012sd,Schmaltz:2005ky} the electroweak scale is
stabilized in a natural way. 
The Higgs boson is viewed as a pseudo-Goldstone boson of a new global symmetry that is broken,
both spontaneously and explicitly, by new physics around the 10\TeV scale. 
The Lagrangian contains two sets of interactions that explicitly break the symmetry, in addition to
the symmetric part $\mathcal{L}_0$
\begin{equation}
  \mathcal{L} = \mathcal{L}_0 + \lambda_1 \mathcal{L}_1 + \lambda_2 \mathcal{L}_2 .
\end{equation}
The Higgs boson would be an exact massless Goldstone boson if both couplings $\lambda_1$ and
$\lambda_2$ vanish, and can only acquire a mass if both of them are present. This means that
the corrections to the Higgs mass are suppressed by two loops w.r.t. the cutoff scale, and as such
the hierarchy problem only appears around a scale of 10\TeV, compared to 1\TeV for the SM. 
Similarly to supersymmetry, new particles are postulated to exist, but they should have the same 
spin as the known SM particles. 
Many choices for the new global symmetry can be made, resulting in slightly different model
predictions. To avoid having a big impact on electroweak precision observables, T-parity is usually
introduced. This parity ensures that the new particles have to be produced in pairs, as is the case
for R-parity in SUSY, which means they only impact the observables at loop-level. 

\subsection{Extra dimensions \label{sec:extra_dimensions}}

A key assumption in models of extra spatial dimensions~\cite{ArkaniHamed:1998rs}, is that the
electroweak scale is the only fundamental short distance scale. The loop corrections to the Higgs
mass are thus cut off at the electroweak rather than the Planck scale, resulting in a much less
finetuned model.  
The weakness of gravity is explained by assuming that gravity permeates these new dimensions, while
the gauge interactions do not. 
The Planck scale in ($4+n$) dimensions is assumed to be of the order of the electroweak scale.  
The effective Planck scale at large distances (larger than the size $R$ of the extra dimensions)
then becomes $M_{\text{Pl}(4+n)}\cdot R^n$. For $n\geq 2$, the needed size of the extra dimensions
is sub-millimetre, a scale where the current understanding of gravity has not been tested yet. 
Models of extra dimensions can be tested in the high-energy collisions at the
LHC~\cite{Chatrchyan:2011fq}. One could detect excited gravitons, which preferentially decay to two
high energy photons, or one could look for missing energy when particles disappear into the extra
dimensions. 
